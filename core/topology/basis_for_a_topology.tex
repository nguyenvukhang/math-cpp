\subsection{Basis for a topology}\label{b81839f}

\Proposition{Collection generated from a basis is a topology}\label{ee1b725}

Let $X$ be a given set, and let $\mathcal B$ be a \href{e896402}{basis} for a
topology on $X$.

Let a collection $\mathcal T$ be \href{e6b5306}{generated} by $\mathcal B$.
Then $\mathcal T$ is a \href{cc8eb8b}{topology} on $X$.

\begin{proof}
  Let $\mathcal T$ denote the collection of all subsets of $X$ generated by
  basis $\mathcal B$. It follows from \href{e6b5306}{definition} that
  \begin{equation*}
    \mathcal T=\Set{U\subseteq X}{\forall x\in U,\ \exists B\in\mathcal B\text{ \ s.t. } x\in B\text{ and }B\subseteq U}\Tag{*}
  \end{equation*}

  We now have to show that $\mathcal T$ is indeed a \href{cc8eb8b}{topology}.

  The empty set $\emptyset$ is among those generated, since it satisfies the
  generating condition in $(*)$ vacuously. Hence, $\emptyset\in\mathcal T$.

  By definition of a \href{e896402}{basis}, for all $x\in X$, there is a $B\in
  \mathcal B$ such that $x\in B$ and $B\subseteq X$. Hence $X$ satisfies the
  generating condition in $(*)$ too, so $X\in\mathcal T$.

  Now let us take an indexed family $\{U_\alpha\}_{\alpha\in J}$, of elements
  of $\mathcal T$ and show that
  $$
    U:=\bigcup_{\alpha\in J}U_\alpha
  $$

  belongs to $\mathcal T$. Given $x\in U$, there is an index $\alpha$ such that
  $x\in U_\alpha$. Since $U_\alpha\in\mathcal T$, \href{e896402}{there is} a
  basis element $B$ such that $x\in B\subseteq U_\alpha$. Then $x\in B$ and
  $B\subseteq U$, and so by $(*)$, we have that $U\in\mathcal T$.

  Now let us take two elements $U_1,U_2\in\mathcal T$ and show that $U_1\cap
  U_2$ belongs to $\mathcal T$. Given $x\in U_1\cap U_2$, we have that $x\in
  U_1$ and $x\in U_2$. \href{e896402}{By definition}, there exists a basis
  element $B_1$ such that $x\in B_1\subseteq U_1$. Then $x\in B_1$ and
  $B_1\subseteq U_1$. Similarly, there is a basis element $B_2$ such that $x\in
  B_2$ and $B_2\subseteq U_2$. So then $x\in B_1\cap B_2$.

  \href{e896402}{By definition} again, there is a basis element $B_3$ such that
  $B_3\subseteq B_1\cap B_2$ and $x\in B_3$. Clearly, $B_3\subseteq U_1\cap
  U_2$, and hence we have $x\in B_3\subseteq U_1\cap U_2$. It follows that
  $U_1\cap U_2$ satisfies the generating condition in $(*)$, and hence $U_1\cap
  U_2\in\mathcal T$.

  Finally, we show by induction that any finite intersection $U_1\cap\ldots\cap
  U_n$ of elements of $\mathcal T$ is in $\mathcal T$. This fact is trivial for
  $n=1$; we suppose it true for $n-1$ and prove it for $n$. Now,
  $$
    U_1\cap\ldots\cap U_n=(U_1\cap\ldots\cap U_{n-1})\cap U_n
  $$

  By hypothesis, $U_1\cap\ldots\cap U_{n-1}$ belongs to $\mathcal T$, and by
  the result just proved, the intersection of two elements (in this case,
  $U_1\cap\ldots\cap U_{n-1}$ and $U_n$) in $\mathcal T$ belongs to $\mathcal
  T$.

  Thus we have checked that $\mathcal T$ as generated by $\mathcal B$ in $(*)$
  satisfies all \href{cc8eb8b}{conditions} required to be a topology.
\end{proof}

\Lemma{Topology is the collection of all unions of elements in a basis}\label{cd21899}
%+Munkres Lemma 13.1

Let $X$ be a set, and $\mathcal B$ be a \href{e896402}{basis} for a
\href{cc8eb8b}{topology} $\mathcal T$ on $X$. Then $\mathcal T$ equals the
collection of all unions of elements of $\mathcal B$.

This effectively means that every \href{c4490f8}{open set} in $X$ can be
expressed as a union of basis elements.

\begin{proof}
  Given a collection of elements of $\mathcal B$, \href{e6b5306}{they are also}
  elements of $\mathcal T$. By \autoref{ee1b725}, $\mathcal T$ is a topology
  \href{cc8eb8b}{and so} their union is in $\mathcal T$.

  Conversely, given $U\in\mathcal T$, choose for each $x\in U$ an element $B_x$
  of $\mathcal B$ such that $x\in B_x\subseteq U$. By
  \href{e6b5306}{construction}, such a $B_x$ exists for each $x$. Then we have
  $U=\bigcup_{x\in U}B_x$, so $U$ equals a union of elements of $\mathcal B$.
\end{proof}

\Lemma{Criterion for basis (topology)}\label{bc13024}
%+Obtaining a basis from a given topology
%+Criterion for a topology's generating basis
%+Munkres Lemma 13.2

Let $(X,\mathcal T)$ be a \href{de3c1b1}{topological space}. Suppose that
$\mathcal C$ is a collection of \href{c4490f8}{open sets} of $X$ such that for
each open set $U$ of $X$ and each $x\in U$, there is an element $C$ of
$\mathcal C$ such that $x\in C\subseteq U$. Then $\mathcal C$ is a
\href{e896402}{basis} for $\mathcal T$.

\begin{proof}
  We must show that $\mathcal C$ is a \href{e896402}{basis}.

  Given $x\in X$, \href{cc8eb8b}{since} $X$ itself is an open set, there is by
  hypothesis a $C\in\mathcal C$ such that $x\in C\subseteq X$.

  Let $C_1,C_2\in\mathcal C$, and let $x\in C_1\cap C_2$. Since $C_1$ and $C_2$
  are open, \href{cc8eb8b}{so is} $C_1\cap C_2$. Therefore, there exists by
  hypothesis a $C_3\in\mathcal C$ such that $x\in C_3\subseteq C_1\cap C_2$.

  With the last two results, we've shown that $\mathcal C$ is a basis.

  Next, we must show that the topology $\mathcal T'$ generated by $\mathcal C$
  equals the topology $\mathcal T$.

  First, note that if $U\in\mathcal T$ and $x\in U$, then there is by
  hypothesis a $C\in\mathcal C$ such that $x\in C\subseteq U$. It follows from
  \href{e896402}{definition} that $U\in\mathcal T'$. Conversely, if
  $W\in\mathcal T'$, then by \autoref{cd21899}, $W$ equals to a union of
  elements in $\mathcal C$. Since (by construction) each element of $\mathcal
  C$ belongs to $\mathcal T$, $W$ is in fact a union of elements in $\mathcal
  T$. \href{cc8eb8b}{Since} $\mathcal T$ is a topology, we have $W\in\mathcal
  T$.
\end{proof}

\Lemma{Fineness comparison using bases}\label{fb92a68}
%+Munkres Lemma 13.3

Let $\mathcal B$ and $\mathcal B'$ be \href{e896402}{bases} for
\href{cc8eb8b}{topologies} $\mathcal T$ and $\mathcal T'$ respectively, on a
given set $X$. Then the following are equivalent:
\begin{enumerati}
  \item $\mathcal T'$ is \href{e04d168}{finer} than $\mathcal T$.
  \item For each $x\in X$ and each basis element $B\in\mathcal B$ containing
        $x$, there is a basis element $B'\in\mathcal B'$ such that $x\in
        B'\subseteq B$.
\end{enumerati}

\begin{proof}
  ($\implies$) We are given $x\in X$ and $B\in\mathcal B$, with $x\in B$. Now
  $B\in\mathcal T$ by \href{e6b5306}{definition}, and $\mathcal
  T\subseteq\mathcal T'$ by (i). Therefore, $B\in\mathcal T'$. Since $\mathcal
  T'$ is generated by $\mathcal B'$, there is an element $B'\in\mathcal B'$ such
  that $x\in B'\subseteq B$.

  ($\impliedby$) Given an element $U\in\mathcal T$, we wish to show that
  $U\in\mathcal T'$ (to show that $\mathcal T\subseteq\mathcal T'$). Let $x\in
  U$. Then \href{e6b5306}{by definition} there is a $B\in\mathcal B$ such that
  $x\in B\subseteq U$. Then by assumption there is also a $B'\in\mathcal B'$
  such that $x\in B'\subseteq B(\subseteq U)$. Hence for each $x\in U$, we have
  a $B'\in\mathcal B'$ such that $x\in B'\subseteq U$, \href{e6b5306}{showing
  that} $U\in\mathcal T'$.
\end{proof}

\Lemma{Comparing topologies $\mathbb R$, $\mathbb R_\ell$, and $\mathbb R_K$}\label{e1bd9d5}
%+Munkres Lemma 13.4

The topologies of \href{cefed4f}{$\R_\ell$} and \href{ee5d783}{$\R_K$} are
strictly \href{e04d168}{finer} than the \href{ad37a51}{standard topology on
$\R$}, but are not comparable with one another.

\begin{proof}
  \def\T{\mathcal T}\def\Tl{\mathcal T_\ell}\def\TK{\mathcal T_K}

  Let $\T$, $\Tl$, $\TK$ be the topologies of $\R$, $\R_\ell$, and $\R_K$
  respectively.

  \proofp{$\T\subsetneq\Tl$} Given a basis element $(a,b)$ for $\T$ and a point
  $x$ of $(a,b)$, the basis element $[x,b)$ for $\Tl$ contains $x$ and lies in
  $(a,b)$. On the other hand, given the basis element $[x,d)$ for $\Tl$, there
  is no open interval $(a,b)$ that contains $x$ and lies in $[x,d)$.

  \proofp{$\T\subsetneq\TK$} Given a basis element $(a,b)$ for $\T$ and a point
  $x$ of $(a,b)$, this same interval is a basis element for $\TK$ that contains
  $x$. On the other hand, given the basis element $B=(-1,1)\setminus K$ for
  $\TK$ and the point $0$ of $B$, there is no open interval that contains $0$
  and lies in $B$.

  \proofp{$\Tl$ and $\TK$ are not comparable} Given a basis element $[a,b)$ for
  $\Tl$, there is no basis element in $\TK$ that contains $a$ and lies in
  $[a,b)$.

  Given a basis element $[a,b)$ for $\Tl$. Assume on the contrary that there is
  a basis element $(\alpha,\beta)$ for $\TK$ such that $a\in(\alpha,\beta)$.
  Then $\alpha<a$, and we have $\alpha\notin[a,b)$. Hence
  $(\alpha,\beta)\not\subset[a,b)$. The same argument holds if we consider the
  basis element $(\alpha,\beta)\setminus K$ for $\TK$. Hence there is no basis
  element for $\TK$ that contains $a$ and lies in $[a,b)$, and thus
  $\Tl\not\subset\TK$.

  Given a basis element $B=(-1,1)\setminus K$ for $\TK$, and the point $0$ of
  $B$, assume on the contrary that there exists a basis element $[a,b)$ for
  $\Tl$ such that $0\in[a,b)$. Then $a\leq0<b$. Then by the
  \href{d845856}{Archimedean Property}, there exists $n_b\in\N$ such that
  $0<1/n_b<b$. So then $n_b\notin B$ but $n_b\in[a,b)$, thus we have that
  $[a,b)\not\subset B$, and so $\TK\not\subset\Tl$.

  Hence, $\Tl$ and $\TK$ are not comparable.
\end{proof}

\Proposition{Collection generated from a subbasis is a topology}\label{dc96413}

Let $\mathcal S$ be a \href{aba7b48}{subbasis} for some topology on a given set
$X$. Then the collection \href{d1d3329}{generated} by $\mathcal S$ is indeed a
\href{cc8eb8b}{topology}.

\begin{proof}
  \def\B{\mathcal B}\def\S{\mathcal S}

  It \href{cd21899}{suffices} to show that the collection $\B$ of all finite
  intersections of elements of $\S$ is a \href{e896402}{basis}.

  Given any $x\in X$, \href{aba7b48}{by construction} there is an $S\in\S$ such
  that $x\in S$. Trivially, $S$ is part of the collection of all finite
  intersections of elements in $\S$, so $S\in\B$.

  Now assume that $x\in B_1\cap B_2$, where $B_1,B_2\in\B$. By assumption,
  $B_1$ and $B_2$, belonging to $\B$, are finite intersections of elements of
  $\S$. So then $B_1\cap B_2$ is also a finite intersection of elements of
  $\S$, and hence $B_1\cap B_2\in\B$.

  Hence $\B$ is indeed a \href{e896402}{basis} for a topology, thus completing
  the proof.
\end{proof}
