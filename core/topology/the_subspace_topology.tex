\subsection{The Subspace Topology}\label{c73c655}

\Remark{The subspace topology is indeed a topology}\label{b9e513c}

Let $(Y,\mathcal T_Y)$ be a \href{cddfbd8}{subspace} of a given
\href{de3c1b1}{topological space} $(X,\mathcal T)$. Then $\mathcal T_Y$ is a
\href{cc8eb8b}{topology}.

\begin{proof}
  \def\T{\mathcal T}
  Recall that by \href{cddfbd8}{definition}, $\T_Y=\Set{Y\cap U}{U\in\T}$

  Then clearly $\emptyset\in\T_Y$ because $\emptyset\in\T$ and
  $Y\cap\emptyset=\emptyset$. Similarly, $Y\in\T_Y$, because $X\in\T$ and
  $Y\cap X=Y$.

  It is also closed under finite intersections:
  $$
    (U_1\cap Y)\cap\ldots\cap(U_n\cap Y)=(U_1\cap\ldots\cap U_n)\cap Y
    \with{U_i\in\T\text{, for }i=\iter1n}
  $$

  and closed under arbitrary unions:
  $$
    (U_1\cap Y)\cup(U_2\cap Y)\cup\ldots=(U_1\cup U_2\cup\ldots)\cap Y
    \with{U_i\in\T\text{, for }i=1,2,\ldots}
  $$
\end{proof}

\Lemma{Element-wise intersection of basis yields basis for subspace topology}\label{a4e8ce7}
%+Munkres Lemma 16.1

If $\mathcal B$ is a \href{e896402}{basis} for the \href{cc8eb8b}{topology} of
$X$ then the collection
$$
  \mathcal B_Y:=\Set{B\cap Y}{B\in\mathcal B}
$$

is a basis for the \href{cddfbd8}{subspace topology} on $Y$.

\begin{proof}
  Given $U$ open in $X$ and given $y\in U\cap Y$, we can choose and element $B$
  of $\mathcal B$ such that $y\in B\subseteq U$. Then $y\in B\cap Y\subseteq
  U\cap Y$. It follows from \autoref{bc13024} that $\mathcal B_Y$ is a basis for the
  subspace topology on $Y$.
\end{proof}

\Lemma{Element-wise intersection of subbasis yields a subbasis for subspace topology}\label{da3a0b1}

If $\mathcal S$ is a \href{aba7b48}{subbasis} for the \href{cc8eb8b}{topology}
$\mathcal T$ of $X$, and $Y$ is a subset of $X$, then the collection
$$
  \mathcal S_Y:=\Set{S\cap Y}{S\in\mathcal S}
$$

is a subbasis for the \href{cddfbd8}{subspace topology} on $Y$.

\begin{proof}
  \def\T{\mathcal T}\def\TY{\mathcal T_Y}
  \def\S{\mathcal S}\def\SY{\mathcal S_Y}

  Recall that the subspace topology of $Y$ is given by
  $$
    \TY:=\Set{Y\cap U}{U\in\T}
  $$

  Let $\TY'$ denote the topology \href{d1d3329}{generated} by $\SY$. We will
  now show that $\TY=\TY'$.

  Let $U$ be an element of $\TY'$. Then $U$ is a union of finite intersections
  of elements of $\SY$. Taking an indexed family $\{S_\alpha\}_{\alpha\in J}$
  of subcollections of $\S$, we can express $U$ by
  \begin{align*}
    U &=\bigcup_{\alpha\in J}\Bigl(\bigcap_{V\in S_\alpha}\bigl(V\cap Y\bigr)\Bigr)\with{(\text{each }V\cap Y\in\SY)} \\
      &=\bigcup_{\alpha\in J}\Bigl(Y\cap\bigcap_{V\in S_\alpha}V\Bigr)                                                \\
      &\href{b34061c}{=}Y\cap\bigcup_{\alpha\in J}\Bigl(\bigcap_{V\in S_\alpha}V\Bigr)
  \end{align*}

  So then $U$ is also a union of $Y$ and a union of finite intersections of
  elements of $\S$ (observe that above, each $V\in S_\alpha\subseteq\S$). By
  definition of a \href{d1d3329}{subbasis}, the union of a finite intersections
  of elements of $\S$ is open in $\T$. So indeed, $U\in\TY$ and we have
  $\TY'\subseteq\TY$.

  By inspection, the argument above is fully reversible, and so
  $\TY'\subseteq\TY$, giving us inclusion in both directions, completing the
  proof.
\end{proof}

\Lemma{Open in an open subspace is open in the main space}\label{b9eeb21}
%+Munkres Lemma 16.2

Let $Y$ be a \href{cddfbd8}{subspace} of $X$. If $U$ is open in $Y$ and $Y$ is
open in $X$, then $U$ is open in $X$.

\begin{proof}
  Since $U$ is open in $Y$, by \href{cddfbd8}{definition} there is a set $V$
  open in $X$ such that $U=Y\cap V$. Since both $Y$ and $V$ are open in $X$,
  \href{cc8eb8b}{so is} $Y\cap V$.
\end{proof}

\Theorem{Product topology of subspaces is the subspace topology of the product}\label{f7ed5a3}
%+Munkres Theorem 16.3

If $A$ is a \href{cddfbd8}{subspace} of $X$ and $B$ is a subspace of $Y$, then
the \href{be6372e}{product topology} on $A\times B$ is the same as the topology
that $A\times B$ inherits as a subspace of $X\times Y$.

\begin{proof}
  By \href{be6372e}{definition}, the set $U\times V$ is a general
  \href{e896402}{basis} element for $X\times Y$, where $U$ is open in $X$ and
  $V$ is open in $Y$. \href{a4e8ce7}{Therefore}, $(U\times V)\cap(A\times B)$ is
  the general basis element for the subspace topology on $A\times B$.

  Now,
  $$
    (U\times V)\cap(A\times B)\href{e2948d3}{=}(U\cap A)\times(V\cap B)
  $$

  Since $U\cap A$ and $V\cap B$ are, by \href{cddfbd8}{definition}, the general
  open sets for the subspace topologies on $A$ and $B$ respectively,
  \href{be6372e}{by inspection}, the set $(U\cap A)\times(V\cap B)$ is the
  general basis element for the product topology on $A\times B$.

  So then the bases for the subspace topology on $A\times B$ and for the
  product topology on $A\times B$ are the same. Hence the topologies are the
  same.
\end{proof}

\Theorem{Convex subsets inherit order topology}\label{bcd8371}
%+Munkres Theorem 16.4

Let $X$ be an ordered set in the \href{aaff6da}{order topology}; let $Y$ be a
subset of $X$ that is \href{c19c232}{convex} in $X$. Then the order topology on
$Y$ is the same as the topology that $Y$ \href{cddfbd8}{inherits} as a subspace
of $X$.

\begin{proof}
  \def\T{\mathcal T}
  Consider the ray $(a,+\infty)$ in $X$, and think about its intersection with
  $Y$. If $a\in Y$, then
  $$
    (a,+\infty)\cap Y=\Set{x}{x\in Y\text{ and }x>a}
  $$

  which is an open ray of the ordered set $Y$. If $a\notin Y$, then $a$ is
  either a lower bound on $Y$ or an upper bound on $Y$, since $Y$ is convex. In
  the former case, $(a,+\infty)\cap Y=Y$, and in the latter case,
  $(a,+\infty)\cap Y=\emptyset$.

  Using a similar argument, we can show that all rays of the form
  $(-\infty,a)\cap Y$ are either open rays in $Y$, or $Y$ itself, or empty.

  \href{bc48b15}{Since} open rays in $X$ form a subbasis for its order topology,
  by \autoref{da3a0b1}, sets of the form $(-\infty,a)\cap Y$ or $(a,+\infty)\cap
  Y$ form a subbasis for the subspace topology on $Y$. But these sets are open
  in the order topology on $Y$ (them being either open rays, empty or $Y$
  itself), and hence the order topology contains the subspace topology on $Y$.

  On the other hand, any open ray of $Y$ equals the intersection of $Y$ and an
  open ray of $X$, so it is open in the \href{cddfbd8}{subspace topology} on
  $Y$. \href{bc48b15}{Since} the open rays of $Y$ are a subbasis for the order
  topology on $Y$, this topology is contained in the subspace topology.

  With inclusion both ways, this proof is complete.
\end{proof}
