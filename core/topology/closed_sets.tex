\subsection{Closed sets}\label{cacf885}

\Theorem{Closed sets on a topological space}\label{e48d738}
%+Munkres Theorem 17.1

Let $X$ be a \href{de3c1b1}{topological space}. Then the following conditions
hold:
\begin{itemize}
  \item[(\textbf{C1})] $\emptyset$ and $X$ are \href{dd23fec}{closed}.
  \item[(\textbf{C2})] Arbitrary intersections of closed sets are closed.
  \item[(\textbf{C3})] Finite unions of closed sets are closed.
\end{itemize}

Note the parallels to the definition of a \href{cc8eb8b}{topology}.

\begin{proof}
  \proofp{C1} $\emptyset$ and $X$ are closed because they are complements of $X$
  and $\emptyset$ respectively.

  \proofp{C2} Given a collection of closed sets $\{A_\alpha\}_{\alpha\in J}$, we
  apply \href{c28492b}{De Morgan's law}:
  $$
    X\setminus\bigcap_{\alpha\in J}A_\alpha=\bigcup_{\alpha\in J}(X\setminus A_\alpha)
  $$

  Since the sets $X\setminus A_\alpha$ are open by definition, the RHS is an
  arbitrary union of open sets, and \href{cc8eb8b}{hence} is open. Therefore,
  $\bigcap_{\alpha\in J}A_\alpha$ is closed.

  \proofp{C3} If $A_i$ is closed for $i=\iter1n$, consider the equation
  $$
    X\setminus\bigcup_{i=1}^nA_i=\bigcap_{i=1}^n(X\setminus A_i)
  $$

  The set on the RHS is a finite intersection of open sets and
  \href{cc8eb8b}{therefore} is open. Hence $\bigcup_{i=1}^nA_i$ is closed.
\end{proof}

\Theorem{Closed in subspace iff intersection of closed set with subset}\label{c1c7979}
%+Munkres Theorem 17.2

Let $Y$ be a \href{cddfbd8}{subspace} of $X$. Then a set $A$ is
\href{dd23fec}{closed} in $Y$ if and only if it equals the intersection of a
closed set of $X$ with $Y$.

\begin{proof}
  Assume that $A=C\cap Y$, where $C$ is closed in $X$. Then $X\setminus C$ is
  open in $X$, so that $(X\setminus C)\cap Y$ is open in $Y$, by definition of
  the \href{cddfbd8}{subspace topology}. Observe that $(X\setminus C)\cap
  Y=Y\setminus A$, and hence $Y\setminus A$ is open in $Y$, so that $A$ is
  closed in $Y$.

  Conversely, let $A$ be closed in $Y$. Then $Y\setminus A$ is open in $Y$, so
  that \href{cddfbd8}{by definition} it equals the intersection of an open set
  $U$ of $X$ with $Y$. Since $X\setminus U$ is closed in $X$, and
  $A=Y\cap(X\setminus U)$, so that $A$ equals the intersection of a closed set
  of $X$ with $Y$.
\end{proof}

\Theorem{Closed in space if closed in subspace and subspace is closed}\label{f493ac5}
%+Munkres Theorem 17.3

Let $Y$ be a \href{cddfbd8}{subspace} of $X$. If $A$ is \href{dd23fec}{closed}
in $Y$ and $Y$ is closed in $X$, then $A$ is closed in $X$.

\begin{proof}
  Since $A$ is closed in $Y$, we have that $U:=Y\setminus A$ is open in $Y$.
  Then by \href{cddfbd8}{definition}, there is a set $V$ open in $X$ such that
  $U=Y\cap V$.

  Since $Y$ is closed in $X$, we have that $X\setminus Y$ is open in $X$.
  \href{cc8eb8b}{Then} $(X\setminus Y)\cup V$ is open in $X$ too. Notice then
  that $A=X\setminus[(X\setminus Y)\cup V]$, and so $A$ is closed.
\end{proof}

\Theorem{Closure in subspace is intersection of subspace and closure}\label{cbef537}
%+Munkres Theorem 17.4

Let $Y$ be a \href{cddfbd8}{subspace} of $X$, and let $A$ be a subset of $Y$.
Let $\bar A$ denote the \href{abdd5f2}{closure} of $A$ in $X$. Then the closure
of $A$ in $Y$ equals $\bar A\cap Y$.

\begin{proof}
  Let $B$ denote the closure of $A$ in $Y$. The set $\bar A$ is closed in $X$,
  so $\bar A\cap Y$ is closed in $Y$ by \autoref{c1c7979}.

  Since $\bar A\cap Y$ contains $A$ (because $A$ is a subset of both $\bar A$
  and $Y$), and by \href{abdd5f2}{definition} $B$ equals the intersection of
  \textit{all} closed subsets of $Y$ containing $A$, we must have
  $B\subseteq(\bar A\cap Y)$.

  On the other hand, \href{abdd5f2}{we know} that $B$ is closed in $Y$. Hence
  by \autoref{c1c7979}, $B=C\cap Y$ for some set $C$ closed in $X$. Then $C$ is
  a closed set of $X$ containing $A$, and because $\bar A$ is the intersection
  of \textit{all} such closed sets, we conclude that $\bar A\subseteq C$. Then
  $(\bar A\cap Y)\subseteq(C\cap Y)=B$.

  With inclusion both ways, we conclude that $B=(\bar A\cap Y)$.
\end{proof}

\Theorem{Closure of a subset in terms of basis}\label{c68b271}
%+Munkres Theorem 17.5

Let $A$ be a subset of the \href{de3c1b1}{topological space} $X$.
\begin{enumerata}
  \item Then $x\in\bar A$ if and only if every open set $U$ containing $x$
        intersects $A$.
  \item Supposing the topology of $X$ is given by a \href{e896402}{basis}, then
        $x\in\href{abdd5f2}{\bar A}$ if and only if every basis element $B$
        containing $x$ intersects $A$.
\end{enumerata}

\begin{proof}
  \proofp{(a)} We will prove by contrapositive.

  If $x\notin\bar A$, then the set $U:=X\setminus\bar A$ is an open set
  (because $\bar A$ \href{abdd5f2}{is closed}) containing $x$ that does not
  intersect $A$, as desired.

  Conversely, if there exists an open set $U$ containing $x$ that does not
  intersect $A$, then $X\setminus U$ is a closed set containing $A$. By
  \href{abdd5f2}{definition}, the set $X\setminus U$ must contain $\bar A$.
  Therefore, $x$ cannot be in $\bar A$.

  \proofp{(b)} If every open set containing $x$ intersects $A$, then so does
  every basis element $B$ containing $x$, \href{e6b5306}{because} $B$ is open.
  Conversely, if every basis element containing $x$ intersects $A$, so does
  every open set $U$ containing $x$, \href{e6b5306}{because} $U$ contains a
  basis element that contains $x$.
\end{proof}

\Theorem{Closure of a subset in terms of basis*}\label{f24a0e1}
%+Munkres Theorem 17.5*

Restating the \href{c68b271}{original theorem} using the notion of a
\href{de512d5}{neighborhood}, it goes:

Let $A$ be a subset of the \href{de3c1b1}{topological space} $X$.
\begin{enumerata}
  \item Then $x\in\bar A$ if and only if every neighborhood of $x$ intersects
        $A$.
  \item Supposing the topology of $X$ is given by a \href{e896402}{basis}, then
        $x\in\href{abdd5f2}{\bar A}$ if and only if every basis element $B$
        containing $x$ intersects $A$.
\end{enumerata}

\Corollary{Cluster points of a set are in the closure of a set}\label{f79da12}

Let $A$ be the subset of a topological space $X$. Then $x$ is a
\href{b0219cd}{limit point} of $A$ if and only if $x$ is in the
\href{abdd5f2}{closure} of $A$ in $X$.

It follows immediately that $A'\subseteq\bar A$, where $A'$ is the set of all
limit points of $A$, and $\bar A$ is the closure of $A$.

\begin{proof}
  Let $x$ be a limit point of $A$. Then by \href{b0219cd}{definition}, every
  \href{de512d5}{neighborhood} of $x$ intersects $A$ at some point other than
  $x$. By \autoref{f24a0e1}, we have $x\in\bar A$.

  Conversely, if $x\in\bar A$, we use \autoref{f24a0e1} again to see that every
  neighborhood of $x$ intersects $A$ at some point other than itself, and by
  definition $x$ is a limit point of $A$.
\end{proof}

\Theorem{Closure is the union of the set and all its limit points}\label{ef10141}

Let $A$ be a subset a topological space $X$, and let $A'$ be the set of all
\href{b0219cd}{limit points} of $A$. Then
$$
  \href{abdd5f2}{\bar A}=A\cup A'
$$

\begin{proof}
  By \autoref{f79da12} we have $A'\subseteq\bar A$, and by
  \href{abdd5f2}{definition} we have $A\subseteq\bar A$. Hence $A\cup
  A'\subseteq\bar A$.

  Now let $x\in\bar A$. If $x$ happens to lie in $A$, it is trivial that $x\in
  A\cup A'$. So suppose that $x\notin A$. Since $x\in\bar A$, every
  neighborhood $U$ of $x$ intersects $A$, and since $x\notin A$, each
  intersection must be at a point other than $x$. \href{b0219cd}{Hence} $x\in
  A'\subseteq A\cup A'$.

  With inclusion both ways, the proof is complete.
\end{proof}

\Corollary{Closed iff contains all limit points}\label{aeb48aa}
%+Munkres Corollary 17.7

A subset of a \href{de3c1b1}{topological space} is \href{dd23fec}{closed} if
and only if it contains all its \href{b0219cd}{limit points}.

\begin{proof}
  By close inspection of \href{abdd5f2}{definition}, set $A$ is closed if and
  only if $A=\bar A$. By \autoref{ef10141}, we have $\bar A=A\cup A'$, and so
  the previous equality holds if and only if $A'\subseteq A$.
\end{proof}

\Theorem{Finite sets in Hausdorff spaces are closed}\label{e76d337}
%+Munkres Theorem 17.8

Every finite point set in a \href{e8a8d91}{Hausdorff space} $X$ is closed.

\begin{proof}
  \def\x{\{x_0\}}
  It suffices to show that every one-point set $\x$ is closed. If $x$ is a point
  of $X$ that differs from $x_0$, then $x$ and $x_0$ have disjoint neighborhoods
  $U$ and $V$ respectively. Since $U$ does not intersect $\x$, the point $x$
  cannot belong to the closure of the set $\x$. As a result, the closure of the
  set $\x$ is $\x$ itself, \href{abdd5f2}{so that} it is closed.
\end{proof}

\Theorem{In $T_1$ space, limit point of set ↔︎ all neighborhoods have infinitely many points in set}\label{f3c1e40}
%+\Munkres Theorem 17.9

Let $X$ be a space satisfying the \href{f294751}{$T_1$ axiom}. Let $A$ be a
subset of $X$. Then the point $x$ is a \href{b0219cd}{limit point} of $A$ if
and only if every \href{de512d5}{neighborhood} of $x$ contains infinitely many
points of $A$.

\begin{proof}
  If every neighborhood of $x$ intersects $A$ at infinitely many points, it
  certainly intersects $A$ at some point other than $x$ itself, so then $x$ is
  a limit point of $A$.

  Conversely, suppose that $x$ is a limit point of $A$, and suppose some
  neighborhood $U$ of $x$ intersects $A$ at only finitely many points. Then $U$
  also intersects $A\sans x$ at finitely many points. Let $F:=U\cap(A\sans x)$.
  Then the set $X\setminus F$ is open in $X$, since $F$, \href{f294751}{being
  finite}, is closed.

  So then $U\cap(X\setminus F)$ is a neighborhood of $x$ that does not
  intersect the set $A\sans x$ at all. This contradicts the assumption that $x$
  is a limit point of $A$.
\end{proof}

\Theorem{Uniqueness of convergence in Hausdorff space}\label{ca2aef2}
%+Munkres 17.10

If $X$ is a \href{e8a8d91}{Hausdorff space}, then a sequence of points of $X$
\href{c8f4bbb}{converges} to at most one point of $X$.

\begin{proof}
  Suppose that $\{x_n\}$ is a sequence of points of $X$ that converges to $x$.
  If $y\neq x$, let $U$ and $V$ be disjoint neighborhoods of $x$ and $y$
  respectively. Since $U$ contains $x_n$ for all but finitely many values of
  $n$, the set $V$ cannot.

  In other words, we can always find a set $V$ that is a neighborhood of $y$
  that contains a finite number of elements from the sequence $\{x_n\}$,
  thereby immediately failing the \href{c8f4bbb}{convergence} criteria.

  Therefore, $\{x_n\}$ cannot converge to $y$.
\end{proof}
