\subsection{The Order Topology}\label{e94f389}

\Remark{Definition for a basis for an Order Topology is a basis}\label{fe258d1}

Let $X$ be a set with a simple order relation $<$, and assume that $X$ has more
than one element. Let $\mathcal B$ be a collection as defined
\href{aaff6da}{here}. Then $\mathcal B$ is indeed a \href{e896402}{basis} for a
\href{cc8eb8b}{topology} on $X$.

\begin{itemize}
  \item Any $x\in X$ lies in at least one element of $\mathcal B$.
  \item The intersection of any two intervals in $\mathcal B$ is an element of
        $\mathcal B$ too. (Thus implying that if $x\in B_1\cap B_2$ with
        $B_1,B_2\in\mathcal B$, then $B_1\cap B_2\in\mathcal B$ too.)
\end{itemize}

Verifying these is left as an exercise for the reader.

\Lemma{Open rays are open in the order topology}\label{e52c18d}

If $X$ is an ordered set, then the \href{b1745d9}{open rays} are open in the
\href{aaff6da}{order topology} of $X$.

\begin{proof}
  Consider, for example, the ray $(a,+\infty)$. If $X$ has a largest element
  $b_0$, then $(a,+\infty)=(a,b_0]$, \href{aaff6da}{which is} an open in
  $X$. If $X$ has no largest element, then $(a,+\infty)$ equals the union of
  all \href{aaff6da}{basis elements} of the form $(a,x)$, for $x>a$,
  \href{cc8eb8b}{and so} it is open in $X$ too.

  A similar argument applies to the ray $(-\infty,a)$.
\end{proof}

\Lemma{Open rays form a subbasis for the order topology}\label{bc48b15}

If $X$ is an ordered set, then the \href{b1745d9}{open rays} form a
\href{aba7b48}{subbasis} for the \href{aaff6da}{order topology} on $X$.

That is to say, the order topology of $X$ is \href{d1d3329}{generated by the
subbasis} formed by the set of open rays.

\begin{proof}
  By \autoref{e52c18d}, open rays are open in the order topology, and
  \href{cc8eb8b}{thus} the topology they \href{d1d3329}{generate} is contained
  in the order topology.

  On the other hand, every basis element for the order topology equals a finite
  intersection of open rays; the interval $(a,b)$ equals
  $(-\infty,b)\cap(a,+\infty)$, while $(a,b_0]$ and $[a_0,b)$, if they exist,
  \href{aaff6da}{are themselves} open rays. Hence the topology generated by the
  open rays contains the order topology.

  With inclusion both ways, we have that the set generated by the subbasis of
  open rays is indeed the order topology of $X$.
\end{proof}
