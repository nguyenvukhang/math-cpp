\subsection{Vector spaces}\label{f7b8de8}

\Proposition{Vector spaces have a unique additive identity}\label{c444473}

Let $V$ be a vector space. Then its additive identity is unique.

\begin{proof}
  Let $e_1,e_2$ be the additive identities of $V$. Then,
  \begin{align*}
    e_1+e_2=e_1\desc{since $e_2$ is an additive identity}
  \end{align*}

  and
  \begin{align*}
    e_1+e_2=e_2\desc{since $e_1$ is an additive identity}
  \end{align*}

  But this means that we have $e_1=e_1+e_2=e_2$. Hence the additive identity is
  unique.
\end{proof}

\Proposition{Vectors have a unique additive inverse}\label{cb813a5}

Let $V$ be a \href{fc83050}{vector space}, and let $v\in V$. Then the additive
inverse of $v$ is unique.

\begin{proof}
  Let $a,b\in V$ be additive inverses of $v$. Then,
  \begin{align*}
    a &=a+0\desc{additive identity}    \\
      &=a+(v+b)\desc{additive inverse} \\
      &=(a+v)+b\desc{associativity}    \\
      &=0+b\desc{additive inverse}     \\
      &=b\desc{additive identity}
  \end{align*}

  Hence the additive inverse of every element $v\in V$ is unique.
\end{proof}

\Proposition{Zero times a vector is the zero vector}\label{ab62bb9}

Let $V$ be a \href{fc83050}{vector space}, and let $v\in V$. Then $0v=0$ ($0v$
is the zero vector).

\begin{proof}
  For $v\in V$, we have
  \begin{align*}
    0v &=(0+0)v\desc{scalar arithmetic} \\
       &=0v+0v\desc{distributivity}
  \end{align*}

  by definition, $0v$ is the additive identity and hence is the zero vector.
\end{proof}

\Proposition{Scalar times a zero vector is the zero vector}\label{e658c50}

Let $V$ be a \href{fc83050}{vector space}, and let $0\in V$. For all $a\in\F$,
we have $a0=0$.

\begin{proof}
  \begin{align*}
    a0 &=a(0+0)\desc{additive identity} \\
       &=a0+a0\desc{distributivity}
  \end{align*}

  by definition, $a0$ is the additive identity and hence is the zero vector.
\end{proof}

\Proposition{$-1$ times a vector gives the additive inverse}\label{f8e46a2}

Let $V$ be a \href{fc83050}{vector space}, and let $v\in V$. Then $-1(v)$ is
the additive inverse of $v$.

\begin{proof}
  \begin{align*}
    v+(-1)v &=1v+(-1)v                       \\
            &=(1+(-1))v\desc{distributivity} \\
            &=0v\desc{scalar arithmetic}     \\
            &=0\desc{by \autoref{ab62bb9}}
  \end{align*}
\end{proof}

\Proposition{Conditions for subspace}\label{dea139b}

If it is given that $U$ is a subset of a \href{fc83050}{vector space} $V$, to
show that $U$ is a \href{a0f0f06}{subspace} we only have to show
\begin{enumerati}
  \item $U$ is closed under vector addition. $u,v\in U\implies u+v\in U$.
  \item $U$ is closed under scalar multiplication. $a\in\F,\ u\in
  U\implies au\in U$
  \item $0\in U$.
\end{enumerati}

\begin{proof}
  (i) ensures that vector addition is well-defined. (ii) ensures that scalar
  multiplication is well-defined. (iii) ensures that the additive inverse exists
  in $U$. The rest of the axioms in the \href{fc83050}{definition of a vector
  space} are satisfies because they hold on the larger space $V$.
\end{proof}

\Proposition{Sum of subspaces is the smallest containing subspace}\label{ddb0ddb}

Suppose $\iter{V_1}{V_m}$ are subspaces of a vector space $V$. Then
\href{d7c30bb}{$V_1+\ldots+V_m$} is the smallest subspace of $V$ containing
$\iter{V_1}{V_m}$.

\begin{proof}
  Let $V_1+\ldots+V_m$. It is left as an exercise for the reader to verify
  that $V_1+\ldots+V_m$ satisfies \href{dea139b}{these conditions}.

  Clearly, every subspace $\iter{V_1}{V_m}$ is contained in $V_1+\ldots+V_m$,
  since for any element in any of $\iter{V_1}{V_m}$, it is clearly a member of
  the set
  $$
    \Set{v_1+\ldots+v_m}{\iter{v_1\in V_1}{v_m\in V_m}}
  $$

  which is $V_1+\ldots+V_m$ \href{d7c30bb}{by definition}.

  Conversely, every subspace $U$ of $V$ that contains all of $\iter{V_1}{V_m}$
  contains $V_1+\ldots+V_m$, because $U$ is closed under addition.

  Thus $V_1+\ldots+V_m$ is the smallest subspace of $V$ containing
  $\iter{V_1}{V_m}$.
\end{proof}

\Proposition{Condition for a direct sum}\label{ab66b9d}

Suppose $\iter{V_1}{V_m}$ are subspaces of a vector space $V$. Then
$V_1+\ldots+V_m$ is a \href{c67c961}{direct sum} if and only if the only way to
write $0$ as a sum $v_1+\ldots+v_m$, where each $v_k\in V_k$, is by setting
each $v_k=0$.

\begin{proof}
  First suppose $V_1+\ldots+V_m$ is a direct sum. Then the
  \href{c67c961}{definition} of a direct sum implies that the only way to write
  0 as a sum of $\iter{v_1}{v_m}$, where each $v_k\in V_k$, is by taking each
  $v_k=0$.

  Now suppose that the only way to write $0$ as a sum of $v_1+\ldots+v_m$,
  where each $v_k\in V_k$, is by setting each $v_k=0$. To show that
  $V_1+\ldots+V_m$ is a \href{c67c961}{direct sum}, let $v\in V_1+\ldots+V_m$
  be arbitrary. Then we can write
  \begin{equation*}
    v=v_1+\ldots+v_m\Tag{*}
  \end{equation*}

  for some $\iter{v_1\in V_1}{v_m\in V_m}$. To show that this representation is
  unique, suppose we also have
  $$
    v=u_1+\ldots+u_m
  $$

  where $\iter{u_1\in V_1}{u_m\in V_m}$. Subtracting these two equations, we
  have
  $$
    0=(v_1-u_1)+\ldots+(v_m-u_m)
  $$

  By closure under addition, $v_k-u_k\in V_k$ for every $k=\iter1m$. But
  because there is only one way to write $0$ as such a sum, we have $v_k-u_k=0$
  for each $k=\iter1m$. Thus $v_k=u_k$ for all $k$ and the representation of
  $v$ (as in $(*)$) is unique.
\end{proof}

\Proposition{Direct sum of two subspaces iff they intersect only at 0}\label{f41081f}

Suppose $U$ and $W$ are subspaces of vector space $V$. Then
$$
  \href{d7c30bb}{U+W}\text{ is a \href{c67c961}{direct sum}}\iff U\cap W=\{0\}
$$

\begin{proof}
  First, suppose that $U+W$ is a direct sum. If $v\in U\cap W$, then $0=v+(-v)$,
  where $v\in U$ and $-v\in W$. By \autoref{ab66b9d}, we have $v=0$. Thus $U\cap
  W=\{0\}$, completing the proof in one direction.

  To prove the other direction, now suppose that $U\cap W=\{0\}$. Let $u\in U$
  and $w\in W$ such that
  $$
    0=u+w
  $$

  But this implies that $u=-w$, and by closure under scalar multiplication (by
  $-1$), $u\in W$. Thus $u\in U\cap W$. Hence $u=0$, which then implies that
  $w=0$. We've thus shown that the only way to write zero as $u+w$ is to have
  both $u=0$ and $w=0$. By \autoref{ab66b9d}, this implies that $U+W$ is a
  direct sum.
\end{proof}
