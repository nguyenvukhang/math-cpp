\subsection{Span and linear independence}\label{aa87399}

\Lemma{Span is the smallest containing subspace}\label{fc857d6}

The span of a list of vectors in vector space $V$ is the smallest subspace of
$V$ containing all vectors in the list.

\begin{proof}
  Suppose $\iter{v_1}{v_n}$ is a list of vectors in $V$.

  First, we will show that $S:=\Span\{\iter{v_1}{v_n}\}$ \href{dea139b}{is a
  subspace} of $V$. For all $u,w\in S$, we can write $u=\sum_{i=1}^na_iv_i$ and
  $w=\sum_{i=1}^nb_iv_i$ for some collections of numbers $\iter{a_1}{a_n}$ and
  $\iter{b_1}{b_n}$.
  \begin{enumerati}
    \item Closed under vector addition:
    \begin{align*}
      u+w=\sum_{i=1}^n(a_i+b_i)v_i\in S
    \end{align*}
    \item Closed under scalar multiplication:
    \begin{align*}
      \lambda u=\sum_{i=1}^n(\lambda a_i)v_i\in S\with{(\lambda\in\F)}
    \end{align*}
    \item Contains the additive identity:
    $$
      0=\sum_{i=1}^n0v_i\in\Span\{\iter{v_1}{v_n}\}
    $$
  \end{enumerati}

  Thus, $S$ is a subspace of $V$. Clearly, $v_k\in\Span\{\iter{v_1}{v_n}\}$ for
  each $k=\iter1n$; so $S$ contains every vector in that list.

  Conversely, because subspaces are closed under vector addition and scalar
  multiplication, every subspace of $V$ that contains all of $\iter{v_1}{v_n}$
  will also contain $\Span\{\iter{v_1}{v_n}\}$.

  Thus, $S=\Span\{\iter{v_1}{v_n}\}$ is the smallest subspace of $V$ containing
  all the vectors $\iter{v_1}{v_n}$.
\end{proof}

\Lemma{Linear dependence lemma}\label{ba96a6f}

Let $V$ be a vector space, and let $\iter{v_1}{v_m}\in V$ be linearly
dependent. Then there exists $k\in[1,m]$ such that
$$
  v_k\in\Span\{\iter{v_1}{v_{k-1}}\}
$$

Furthermore, if $k$ satisfies the condition above and the $k$-th term is
removed from $\iter{v_1}{v_m}$, then the span of the remaining list equals
$\Span\{\iter{v_1}{v_m}\}$.

\begin{proof}
  Because $\iter{v_1}{v_m}$ is linearly dependent, there exists numbers
  $\iter{a_1}{a_m}\in\F$, not all $0$, such that
  $$
    a_1v_1+\ldots+a_mv_m=0
  $$

  Let $k$ be the largest element of $\{\iter1m\}$ such that $a_k\neq0$. Then
  $$
    v_k=-\frac{a_1}{a_k}v_1-\ldots-\frac{a_{k-1}}{a_k}v_{k-1}
  $$

  which shows that $v_k\in\Span\{\iter{v_1}{v_m}\}$, as desired.

  Now, for the second part. Suppose that $k$ is any element in $\{\iter1m\}$
  such that $v_k\in\Span\{\iter{v_1}{v_m}\}$. Let $\iter{b_1}{b_{k-1}}\in\F$ be
  such that
  \begin{equation*}
    v_k=b_1v_1+\ldots+b_{k-1}v_{k-1}\Tag{*}
  \end{equation*}

  Let $u\in\Span\{\iter{v_1}{v_m}\}$ be arbitrary. Then there exists
  $\iter{c_1}{c_m}\in\F$ such that
  $$
    u=c_1v_1+\ldots+c_mv_m
  $$

  But in this equation, we can replace $v_k$ on the RHS with $(*)$, which shows
  that $u$ is in the span of the list obtained by removing the $k$-term from
  $\iter{v_1}{v_m}$. Thus, removing the $k$-th term from the list
  $\iter{v_1}{v_m}$ does not change the span of the list.
\end{proof}

\Lemma{Length of linearly independent list $≤$ length of spanning list}\label{d8487b6}

In a finite-dimensional vector space, the length of every linearly independent
list of vectors is equal to or less than the length of every spanning list of
vectors.

\begin{proof}
  Suppose that $U:=\iter{u_1}{u_m}$ is linearly independent in $V$, a
  finite-dimensional vector space. Suppose also that $B:=\iter{w_1}{w_n}$ spans
  $V$. We need to prove that $m\leq n$. We do so through the process described
  below with $m$ steps.

  \paragraph{Step 1}

  Prepending $u_1$ to the list $B$ produces a linearly dependent list, since
  $u_1\in V$ and $B$ spans $V$. So
  $$
    S_1^+:=u_1,w_1,\ldots,w_n
  $$

  is linearly dependent. Thus, by the \href{ba96a6f}{linear dependence lemma},
  one of the vectors in $S_1^+$ is a linear combination of the previous vectors
  in $S_1^+$. We know that $u_1\neq0$ (otherwise $U$ is linearly dependent),
  and thus $u_1$ is not in the span of the previous vectors (The span of the
  empty list is $\{0\}$). Hence, we can remove one of the $w$'s so that the new
  list $S_1$ (of length $n$) consisting of $u_1$ and the remaining $w$'s spans
  $V$.

  \paragraph{Step $k$ (for $k=\iter2m$)}

  The list $S_{k-1}$ (of length $n$) spans $V$. In particular, $u_k\in\Span
  S_{k-1}$. Thus,
  $$
    S_k^+:=u_k\cup S_{k-1}
  $$

  is linearly dependent. We arrange $S_k^+$ such that the first $k$ elements
  are $\iter{u_1}{u_k}$. Again by the \href{ba96a6f}{linear dependece lemma},
  one of the vectors in $S_k^+$ is in the span of the previous ones, and
  because $\iter{u_1}{u_k}$ is linearly independent, this vector cannot be one
  of the $u$'s.

  Hence there must still be at least one remaining $w$ at this step. We can
  remove from $S_k^+$ a $w$ that is a linear combination of the previous
  vectors in $S_k^+$, to obtain $S_k$ (of length $n$) that spans $V$.

  Now, after step $m$, we have $S_m$ which has length $m$ and contains all the
  $u$'s. At each step as we add some $u$, the linear dependence lemma
  guarantees that there is some $w$ to remove. Hence there are at least as many
  $w$'s as $u$'s.
\end{proof}

\Proposition{Uniqueness of expression under linear independence}\label{dcafe9d}

Let $V$ be a vector space, and let $U:=\{\iter{u_1}{u_m}\}\subset V$ be
linearly independent. Then if $w\in\Span U$, there is a unique expression of
$w$ as a \href{ceb6342}{linear combination} of the vectors in $U$.

\begin{proof}
  Let $w$ have two representations of vectors in $U$:
  \begin{align*}
    w &=a_1u_1+\ldots+a_mu_m \\
      &=b_1u_1+\ldots+b_mu_m
  \end{align*}

  where $\iter{a_1}{a_m},\iter{b_1}{b_m}\in\F$. Then we have
  $$
    0=w-w=(a_1-b_1)u_1+\ldots+(a_m+b_m)u_m.
  $$

  Since $U$ is \href{c133a44}{linearly independent}, we must have $a_i-b_i=0$
  for $i=\iter1m$. So then $a_i=b_i$ for all $i=\iter1m$, and we have that the
  representation of $w$ in terms of $U$ is unique.
\end{proof}
