\subsection{Matrix multiplication}\label{f41e26f}

\Proposition{Entry of a matrix product equals row times column}\label{bfd6f3c}

\texttt{\href{d76dfe6}{use notation};} Suppose $A\in\F^{m\times n}$ and
$B\in\F^{n\times p}$. Then
$$
  (AB)_{i,j}=A_{i,\cdot}B_{\cdot,j}\with{(i=\iter1m;\ j=\iter1p)}
$$

\begin{proof}
  Let $1\leq i\leq m$ and $1\leq j\leq p$. The \href{d786633}{definition of
  matrix multiplication} states that
  \begin{equation*}
    (AB)_{i,j}=A_{i,1}B_{1,j}+\ldots+A_{i,n}B_{n,j}\Tag{*}
  \end{equation*}

  But the same definition also implies that the product of the $1\times n$
  matrix $A_{i,\cdot}$ and the $n\times 1$ matrix $B_{\cdot,j}$ is the $1\times
  1$ matrix whose entry is the number on the RHS of $(*)$.
\end{proof}

\Proposition{Column of matrix product equals matrix times column}\label{a27f96a}

\texttt{\href{d76dfe6}{use notation};} Suppose $A\in\F^{m\times n}$ and
$B\in\F^{n\times p}$. Then
$$
  (AB)_{\cdot,j}=AB_{\cdot,j}\with{(j=\iter1p)}
$$

\begin{proof}
  Let $1\leq i\leq m$ and $1\leq j\leq p$. The \href{d786633}{definition of
  matrix multiplication} states that
  \begin{equation*}
    (AB)_{i,j}=A_{i,1}B_{1,j}+\ldots+A_{i,n}B_{n,j}\Tag{*}
  \end{equation*}

  Note that both $(AB)_{\cdot,j}$ and $AB_{\cdot,j}$ are $m\times 1$ matrices.
  For $i=\iter1m$, the entry on row $i$ of $(AB)_{\cdot,j}$ is the LHS of
  $(*)$, and the entry on row $i$ of $AB_{\cdot,j}$ is the RHS of $(*)$. Hence
  the matrices are equal and we have $(AB)_{\cdot,j}=AB_{\cdot,j}$.
\end{proof}

\Proposition{Linear combination of columns}\label{be0cc53}

\texttt{\href{d76dfe6}{use notation};} Let $A\in\F^{m\times n}$ and $x\in\F^n$.
Then
$$
  Ax=x_1A_{\cdot,1}+\ldots+x_nA_{\cdot,n}
$$

In other words, $Ax$ is a linear combination of the columns of $A$, with the
coefficients coming from $x$.

\begin{proof}
  If $k\in\{\iter1m\}$, then the definition of \href{d786633}{matrix
  multiplication} implies that the entry in row $k$ of $Ax\in\F^m$ is
  $$
    A_{k,1}x_1+\ldots+A_{k,n}x_n
  $$

  But this is equal to the entry in row $k$ of
  $x_1A_{\cdot,1}+\ldots+x_nA_{\cdot,n}$. Since these lists of $\F^m$ agree on
  every entry, they are the same.
\end{proof}

\Proposition{Matrix multiplication as linear combinations of columns}\label{a9e7369}

\texttt{\href{d76dfe6}{use notation};} Suppose $C\in\F^{m\times p}$ and
$R\in\F^{p\times n}$. Then
\begin{enumerata}
  \item if $i\in\{\iter1n\}$, then column $i$ of $CR$ is a linear combination
        of the columns of $C$, with coefficients coming from column $i$ of $R$.
  \item if $j\in\{\iter1m\}$, then row $j$ of $CR$ is a linear combination of
        the rows of $R$, with coefficients coming from row $j$ of $C$.
\end{enumerata}

\begin{proof}
  Suppose $i\in\{\iter1n\}$. Then by \autoref{a27f96a}, column $i$ of $CR$ is
  $CR_{\cdot,i}$. By \autoref{be0cc53}, $CR_{\cdot,i}$ is the linear combination
  of columns of $C$ with coefficients coming from $R_{\cdot,i}$ (the $i$-th
  column of R). Thus (a) holds.

  For (b), consider \href{e8b98fd}{$(CR)^T=R^TC^T$}. (a) now states that if
  $j\in\{\iter1m\}$, then the column $j$ of $R^TC^T$ is a linear combination of
  the columns of $R^T$, with coefficients coming from column $i$ of $C^T$. But
  by definition of tranpose, the $i$-th column of of $C^T$ is the $i$-th row of
  $C$, and this statement can be rewritten as:

  ``The row $j$ of $CR$ is a linear combination of the rows of $R$,
  with coefficients coming from row $i$ of $C$", exactly as (b) claims.
\end{proof}

\Proposition{Column-row factorization}\label{e75a67d}

\texttt{\href{d76dfe6}{use notation};} Let $A\in\F^{m\times n}$ be a matrix with
\href{bc27a95}{column rank} $c\geq1$. Then there exists $C\in\F^{m\times c}$ and
$R\in\F^{c\times n}$ such that $A=CR$.

\begin{proof}
  Each column of $A$ is in $\F^m$. The list
  $\{\iter{A_{\cdot,1}}{A_{\cdot,n}}\}$ can be reduced to a basis (of length
  $c$, by definition of the column rank) of $\F^{m\times 1}$. The $c$ columns in
  this basis can be put together to form $C\in\F^{m\times c}$.

  If $k\in\{\iter1n\}$, then the column $k$ of $A$ is a linear combination of
  the columns of $C$. Make the coefficients of this linear combination into the
  $k$-th column of $R\in\F^{c\times n}$. Then from \autoref{a9e7369}, we have
  $A=CR$.
\end{proof}

\Proposition{Column rank equals row rank}\label{a8e348c}

Let $A\in\F^{m\times n}$. Then the \href{bc27a95}{column rank} of $A$ equals
the \href{bae7bca}{row rank} of $A$.

\begin{proof}
  Let $c$ denote the column rank of $A$. Let $A=CR$ be the
  \href{e75a67d}{column-row factorization} of $A$, where $C\in\F^{m\times c}$
  and $R\in\F^{c\times n}$. By \autoref{a9e7369}, every row of $A$ is a linear
  combination of the rows of $R$. Since $R$ has $c$ rows, this implies that the
  row rank of $A$ is less than or equal to the column rank $c$ of $A$.

  Now, we apply a similar argument to $A^T$:
  \begin{align*}
    \text{column rank of }A &=\text{row rank of }A^T\desc{by transposing}               \\
                            &\leq\text{column rank of }A^T\desc{by the paragraph above} \\
                            &=\text{row rank of }A\desc{by transposing}
  \end{align*}

  Thus, the column rank of $A$ equals the row rank of $A$.
\end{proof}
