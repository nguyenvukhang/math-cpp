\subsection{Linear maps}\label{e9864a1}

\Lemma{Linear map lemma}\label{f1c27fd}

Let $V,W$ be vector spaces. Suppose $\iter{v_1}{v_n}$ is a basis of $V$ and
$\iter{w_1}{w_n}\in W$. Then there exists a unique \href{d7d1925}{linear map}
$T:V\to W$ such that
$$
  T(v_k)=w_k\with{(k=\iter1n)}
$$

Note that $\iter{w_1}{w_n}$ need \textbf{not} be a basis of $W$.

\begin{proof}
  First, we show existence. Define $T:V\to W$ by
  $$
    T\biggl(\sum_{i=1}^na_iv_i\biggr)=\sum_{i=1}^na_iw_i
  $$

  where each $a_i\in\F$. Since $v_i$ is a basis of $V$, $T$ is indeed a
  function from $V$ to $W$, since all elements of $V$ are acted upon.

  Clearly, $T$ satisfies $T(v_k)=w_k$ for all $k=\iter1n$.

  To show that it's a linear map, let $u,v\in V$. Then there exists collections
  $\{a_i\}\subset\F$ and $\{b_i\}\subset\F$ such that $u=\sum_{i=1}^na_iv_i$
  and $v=\sum_{i=1}^nb_iv_i$. Then
  \begin{align*}
    T(u+v) &=T\biggl(\sum_{i=1}^na_iv_i+\sum_{i=1}^nb_iv_i\biggr)                \\
           &=T\biggl(\sum_{i=1}^n(a_i+b_i)v_i\biggr)                             \\
           &=\sum_{i=1}^n(a_i+b_i)w_i                                            \\
           &=\sum_{i=1}^na_iw_i+\sum_{i=1}^nb_iw_i                               \\
           &=T\biggl(\sum_{i=1}^na_iv_i\biggr)+T\biggl(\sum_{i=1}^nb_iv_i\biggr) \\
           &=T(u)+T(v)
  \end{align*}

  Showing that $T(\lambda u)=\lambda T(u)$ for all $\lambda\in\F$ is left as an
  exercise. Thus $T$ is a linear map from $V$ to $W$.

  Next, to prove uniqueness, suppose that $T\in\href{ab1f2fb}{\L(V,W)}$ and
  that $T(v_k)=w_k$ for each $k=\iter1n$. The \href{d7d1925}{homogeneity} of
  $T$ implies that $T(a_kv_k)=a_kw_k$ for each $k=\iter1n$, and the
  \href{d7d1925}{additivity} of $T$ implies that
  $$
    T\biggl(\sum_{k=1}^na_kv_k\biggr)=\sum_{k=1}^na_kw_k.
  $$

  Thus $T$ is uniquely determined on $\Span\{\iter{v_1}{v_n}\}$, but since
  $\iter{v_1}{v_n}$ is a basis of $V$, then $T$ is uniquely determined on $V$.
\end{proof}

\Proposition{$\mathcal L(V,W)$ is a vector space}\label{dc79809}

Let $V,W$ be vector spaces. Using vector addition and scalar multiplication as
defined \href{e257b42}{here}, $\L(V,W)$ is a vector space.

\begin{proof}
  Let $S,T\in\L(V,W)$. Then for all $u\in V$ and $a,b\in\F$,
  \begin{enumerati}
    \item vector addition is commutative:
    $$
      (S+T)(u)=S(u)+T(u)=T(u)+S(u)=(T+S)(u)
    $$
    \item vector addition is associative:
    $$
      (R+(S+T))(u)=R(u)+(S+T)(u)=R(u)+S(u)+T(u)=\ldots=((R+S)+T)(u)
    $$
    \item there is an additive identity, $E$, given by $E(u):=0\in W$ for all
          $u\in V$.
    $$
      (S+E)(u)=S(u)+E(u)=S(u)
    $$
    \item every $S\in\L(V,W)$ has an additive inverse $S^{-1}$ given by
          $S^{-1}(u):=-S(u)$, so that we have $S+S^{-1}=E$.
    \item there is the scalar multiplicative identity of $0\in\F$.
    \item addition and scalar multiplication are distributive both ways.
          Firstly, that $a(S+T)\equiv aT+aS$:
    \begin{align*}
      (a(S+T))(u) &=a(S+T)(u)\desc{by defn. of scalar mult.}   \\
                  &=a[S(u)+T(u)]\desc{by defn. of vector add.} \\
                  &=aS(u)+aT(u)                                \\
                  &=(aS+aT)(u)\desc{by defn. of vector add.}
    \end{align*}

    and secondly, that $(a+b)S\equiv aS+bS$:
    \begin{align*}
      ((a+b)S)(u) &=(a+b)S(u)   \\
                  &=aS(u)+bS(u) \\
                  &=(aS+bS)(u)
    \end{align*}
  \end{enumerati}

  Thus $\L(V,W)$ forms a vector space.
\end{proof}

\Proposition{Algebraic properties of linear maps}\label{b42b8cd}

Let $V,W$ be vector spaces.

\begin{enumerati}
  \item\textit{(Associativity)} $(T_1T_2)T_3=T_1(T_2T_3)$ whenever $T_1$, $T_2$,
  and $T_3$ are linear maps such that the products make sense (where $T_3$ maps
  into the domain of $T_2$, and $T_2$ maps into the domain of $T_1$).
  \item\textit{(Identity)} $TI=IT=T$ whenever $T\in\L(V,W)$. Here, the first $I$
  is the identity operator on $V$, and the second $I$ is the identity operator
  on $W$.
  \item\textit{(Distributivity)} If $T,T_1,T_2\in\L(U,V)$ and
  $S,S_1,S_2\in\L(V,W)$, then we have
  \begin{gather*}
    (S_1+S_2)T\equiv S_1T+S_2T \\
    S(T_1+T_2)\equiv ST_1+ST_2
  \end{gather*}
\end{enumerati}

\begin{proof}
  The routine proofs are left to the reader.
\end{proof}

\Result{Linear maps take 0 to 0}\label{c5eb127}

Let $V,W$ be vector spaces and let $T:V\to W$ be a \href{d7d1925}{linear map}.
Then $T(0)=0$.

\begin{proof}
  By linearity, we have $T(0+0)=T(0)+T(0)$, but by the zero property in
  $V$, we have $T(0+0)=T(0)$. Hence, we've just shown that
  $$
    T(0)+T(0)=T(0)
  $$

  thus, $T(0)$ is the zero vector in $W$.
\end{proof}

\Proposition{Extending linear maps on a subspace}\label{ebbd7cd}

Let $V$ be a finite-dimensional vector space. Then every linear map on a
subspace of $V$ can be extended to a linear map on $V$.

In other words, if $U$ is a subspace of $V$ and $S\in\L(U,W)$, then there
exists $T\in\L(V,W)$ such that
$$
  T(u)=S(u)\with{(\forall u\in U)}
$$

\begin{proof}
  Let $\iter{u_1}{u_m}$ be a basis of $U$. Thinking of it as a linearly
  independent list of vectors, by \autoref{f0fa1cd}, it can be extended to
  a basis of $V$. Let this basis be
  $$
    \iter{u_1}{u_m},\iter{v_1}{v_n}.
  $$

  Given $S\in\L(U,W)$, we define $T\in\L(V,W)$ such that
  \begin{align*}
    T(u_i) &=S(u_i)\with{i=\iter1m} \\
    T(v_j) &=0\with{j=\iter1n}
  \end{align*}

  By construction, we have $T(u)=S(u)$ for all $u\in U$ because
  $\iter{u_1}{u_m}$ is a basis of $U$ and hence by linearity $T$ and $S$ agree
  when acting on all elements of $U$.

  Also, $T$ is well-defined on $V$, since $\iter{u_1}{u_m},\iter{v_1}{v_n}$ is
  a basis of $V$.
\end{proof}
