\subsection{Inner Product Spaces}\label{cfb69c6}

\Result{Dot product with one slot fixed is a linear map}\label{dcae040}

Fix an element $v\in\mathcal F^n$. Then the function $\phi_v:\mathcal
F^n\to\mathcal F$ defined by
$$
  \phi_v(u):=v\cdot u
$$

is a \href{d7d1925}{linear map}.

\begin{proof}
  For all $u,w\in\mathcal F^n$ and all $a\in\mathcal F$,
  $$
    \phi_v(u+w)
    =v\cdot(u+w)
    =v\cdot u +v\cdot w
    =\phi_v(u)+\phi_v(w)
  $$

  and
  $$
    \phi_v(au)=v\cdot(au)=a(v\cdot u)=a\phi_v(u)
  $$
\end{proof}

\Result{Inner product with second slot fixed is a linear map}\label{b636029}

Fix an element $v\in V$, where $V$ is an \href{b9935c8}{inner product space}.
Then the function
$$
  u\mapsto\inner uv
$$

that takes $u\in V$ to $\inner uv$ is \href{d7d1925}{linear}.

\begin{proof}
  This follows almost immediately from \href{cebd07a}{definition} by additivity
  in the first slot and homogeneity in the first slot. It is left as an exercise
  for the reader to find
  $$
    \inner{u+w}v=\inner uv+\inner wv\quad\text{ and }\quad
    \inner{\lambda u}v=\lambda\inner uv
  $$

  in the definitions.
\end{proof}

\Proposition{Basic properties of an inner product}\label{fb218c8}

\begin{enumerata}
  \item $\inner0v=0$ for every $v\in V$.
  \item $\inner v0=0$ for every $v\in V$.
  \item $\inner u{v+w}=\inner uv+\inner uw$ for every $u,v,w\in V$.
  \item $\inner u{\lambda v}=\overline\lambda\inner uv$ for all $\lambda\in\F$
  and all $u,v\in V$.
  \item $\inner vv=0$ if and only if $v=0$.
\end{enumerata}

\begin{proof}
  (a) comes from \href{c5eb127}{linear maps sending 0 to 0}, and $u\mapsto
  \inner uv$ is linear by \autoref{b636029}. (b) follows from the property of
  \href{cebd07a}{conjugate symmetry} applied on (a).

  Now, for (c) and (d):
  \begin{align*}
    \inner u{av+bw} &=\overline{\inner{av+bw}u}\desc{conjugate symmetry}          \\
                    &=\overline{a\inner vu+b\inner wu}\desc{linearity in 1st arg} \\
                    &=\overline{a\inner vu}+\overline{b\inner wu}                 \\
                    &=\bar a\overline{\inner vu}+\bar b\overline{\inner wu}       \\
                    &=\bar a\inner uv+\bar b\inner uw\desc{conjugate symmetry}
  \end{align*}

  (c) follows from using $a=b=1$, and (d) follows from $b=0$.

  For (e): If $\inner vv=0$, then by the contrapositive of
  \href{cebd07a}{positive definiteness}, we have $v=0$. And if $v=0$, by (a),
  we have $\inner vv=0$.
\end{proof}

\Theorem{Pythagorean theorem on inner products}\label{c5e5d7d}

Let $u,v\in V$ be \href{d9735e5}{orthogonal} vectors in an \href{b9935c8}{inner
product space} $V$. Then
\begin{align*}
  \inner{u+v}{u+v}       &=\inner uu+\inner vv                                                          \\
  \pre{\iff}\norm{u+v}^2 &=\norm u^2+\norm v^2\desc{see: \href{d828dac}{norms} of inner product spaces}
\end{align*}

This is called the Pythagorean theorem because of how it pans out when the
inner product is the standard dot product.

\begin{proof}
  Since $u,v$ are orthogonal, then $\inner uv=0$. Note that by conjugate
  symmetry, we also have $\inner vu=0$. Then
  \begin{align*}
    \inner{u+v}{u+v} &=\inner u{u+v}+\inner v{u+v}\desc{\href{cebd07a}{linearity in 1st arg}}  \\
                     &=\inner uu+\inner uv+\inner vu+\inner vv\desc{from \href{fb218c8}{this}} \\
                     &=\inner uu+\inner vv
  \end{align*}
\end{proof}

\Remark{Orthogonal decomposition}\label{a7dfcb8}

Let $V$ be an \href{b9935c8}{inner product space} and suppose $u,v\in V$ with
$v\neq0$. Then we can decompose $u$ into the sum of a scaled $v$ and another
vector $w\in V$ orthogonal to $v$ by setting
$$
  c:=\frac{\inner uv}{\inner vv}\quad\text{ and }\quad w:=u-\frac{\inner uv}{\inner vv}v
$$

Observe that we have $u=cv+w$ (clearly), and $\inner wv=0$:
\begin{align*}
  \inner wv &=\biggl\langle u-\frac{\inner uv}{\inner vv}v,v\biggr\rangle                                                    \\
            &=\inner uv-\biggl\langle\frac{\inner uv}{\inner vv}v,v\biggr\rangle\desc{Recall: \href{cebd07a}{inner product}} \\
            &=\inner uv-\frac{\inner uv}{\inner vv}\inner vv                                                                 \\
            &=0
\end{align*}

Here, we call $cv$ the \href{fc332ef}{projection} of $u$ onto $v$.

\Proposition{Triangle inequality for vectors}\label{f5f1056}

Let $V$ be an \href{b9935c8}{inner product space}, and let $u,v\in V$. Then
$$
  \norm{u+v}\leq\norm u+\norm v
$$

where $\norm{\,\cdot\,}$ is the \href{d828dac}{induced norm} of $V$.

This inequality holds with equality if and only if one of $u,v$ is a
non-negative real multiple of the other.

\begin{proof}
  \begin{align*}
    \norm{u+v}^2 &=\inner{u+v}{u+v}                                                                            \\
                 &=\inner uu+\inner vv+\inner uv+\inner vu                                                     \\
                 &=\inner uu+\inner vv+\inner uv+\overline{\inner uv}\desc{\href{cebd07a}{conjugate symmetry}} \\
                 &=\norm u^2+\norm v^2+2\Re\inner uv                                                           \\
                 &\leq\norm u^2+\norm v^2+2|\inner uv|                                                         \\
                 &\leq\norm u^2+\norm v^2+2\norm u\norm v\desc{\href{f85ac46}{Cauchy-Schwarz}}                 \\
                 &=(\norm u+\norm v)^2
  \end{align*}

  Taking square roots on both sides yield the first statement of the
  proposition.

  The proof above shows that the inequality holds with equality if and only if
  the two lines with $\leq$ hold with equality. This happens if and only if
  \begin{equation*}
    \inner uv=\norm u\norm v\Tag{*}
  \end{equation*}

  Note that when this holds, then $\inner uv$ is real. Now, if one of $u,v$ is
  a non-negative real multiple of the other, then $(*)$ holds, because
  $$
    \inner{cu}u=c\inner uu=c\norm u^2=\norm u\norm{cu}
  $$

  and conversely, if $(*)$ holds, then by \href{f012e25}{this remark}, one of
  $u,v$ is a scalar multiple of the other. By inspection on $(*)$, this scalar
  must be a non-negative real number. This proves the second claim.
\end{proof}

\Proposition{Parallelogram equality}\label{bfa1e24}

Let $V$ be a vector space, and let $u,v\in V$. Then
$$
  \norm{u+v}^2+\norm{u-v}^2=2(\norm u^2+\norm v^2)
$$

\begin{proof}
  Recall that by \href{d828dac}{definition}, $\norm x=\inner xx$ for all $x\in
  V$.
  \begin{align*}
    \norm{u+v}^2+\norm{u-v}^2 &=\inner{u+v}{u+v}+\inner{u-v}{u-v}                                               \\
                              &=\norm u^2+\norm v^2+\inner uv+\inner vu+\norm u^2+\norm v^2-\inner uv-\inner vu \\
                              &=2(\norm u^2+\norm v^2)
  \end{align*}
\end{proof}
