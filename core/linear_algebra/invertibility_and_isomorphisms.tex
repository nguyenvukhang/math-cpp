\subsection{Invertibility and Isomorphisms}\label{f475f43}

\Proposition{Inverse of a linear map is unique}\label{c2b81d6}

Let $V,W$ be vector spaces and let $T\in\href{ab1f2fb}{\L(V,W)}$. Then the
inverse of $T$ is unique.

\begin{proof}
  Let $S_1,S_2$ be two \href{e1ba7ee}{inverses} of $T$. Then
  $$
    S_1=S_1I=S_1(TS_2)=(S_1T)S_2=IS_2=S_2
  $$

  This holds because \href{b42b8cd}{linear map products are associative}, and
  thus the inverse of $T$ is unique.
\end{proof}

\Proposition{Linear map is invertible ↔︎ it is bijective}\label{c1de7b1}

A linear map is invertible if and only if it is injective and surjective.

\begin{proof}
  Let $V,W$ be vector spaces, and let $T\in\href{ab1f2fb}{\L(V,W)}$.

  ($\implies$) Now, suppose $T$ is \href{e1ba7ee}{invertible}. To show that $T$
  is injective, suppose $u,v\in V$ and $T(u)=T(v)$. Then
  $$
    u=T^{-1}(T(u))=T^{-1}(T(v))=v
  $$

  so $u=v$, hence $T$ is injective.

  To show that $T$ is surjective, let $w\in W$. Then we have the element
  $v:=T^{-1}(w)\in V$ such that $T(v)=w$:
  $$
    T(v)=T(T^{-1}(w))=w
  $$

  hence $T$ is surjective.

  ($\impliedby$) Now, suppose $T$ is injective and surjective. We want to prove
  that $T$ is invertible.

  Let $S:W\to V$. For each $w\in W$, define $S(w)$ to be the unique element of
  $V$ such that $T(S(w))=w$. The existence of $S(w)$ comes from the
  surjectivity of $T$, and the uniqueness of $S(w)$ comes from the injectivity
  of $T$. Then by construction, $TS$ is the identity operator on $W$.

  To prove that $ST$ equals the identity operator on $V$, let $v\in V$. Then
  \begin{align*}
    T(ST(v)) &=(TS)(T(v))\desc{\href{b42b8cd}{associativity}} \\
             &=T(v)\desc{since $TS=I$ from earlier}
  \end{align*}

  But since $T$ is injective, this implies that $ST(v)=v$ for all $v$, and so
  $ST=I$.

  Now we have to prove that $S$ is linear. Suppose $w_1,w_2\in W$. Then
  \begin{align*}
    T(S(w_1)+S(w_2)) &=T(S(w_1))+T(S(w_2))\desc{by \href{d7d1925}{linearity} of $T$} \\
                     &=w_1+w_2\desc{since $TS=I$}                                    \\
                     &=T(S(w_1+w_2))\desc{by defn. of $S$}
  \end{align*}

  Since $T$ is injective, we have $S(w_1)+S(w_2)=S(w_1+w_2)$.

  Now let $w\in W$ and $\lambda\in\F$. Then
  \begin{align*}
    T(\lambda S(w)) &=\lambda T(S(w))\desc{by \href{d7d1925}{linearity} of $T$} \\
                    &=\lambda w\desc{since $TS=I$}                              \\
                    &=T(S(\lambda w))\desc{by defn. of $S$}
  \end{align*}

  Again by $T$'s injectivity, we have $\lambda S(w)=S(\lambda w)$.

  Hence $S$ is a linear map from $W$ to $V$ such that $ST=I$ and $TS=I$, so $S$
  is the \href{e1ba7ee}{inverse} of $T$, hence $T$ is an invertible linear map.
\end{proof}

\Theorem{Injective ↔︎ surjective ↔︎ invertible for dimension-preserving linear maps}\label{bd0d827}

Let $V,W$ be vector spaces such that $\dim V=\dim W$ and let
$T\in\href{ab1f2fb}{\L(V,W)}$. Then
$$
  T\text{ is invertible}\iff T\text{ is injective}\iff T\text{ is surjective}
$$

\begin{proof}
  By the \href{e83dffc}{Rank-Nullity Theorem},
  $$
    \dim V=\Null T+\Rank T
  $$

  If $T$ is injective, \href{f68db52}{then} $\ker T=\{0\}$ and hence $\Null
  T=0$. So we must have $\Rank T=\dim V$. By \autoref{a41ddec}, $T$ is
  surjective.

  This argument is easily reversible to show that surjectivity implies
  injectivity. Since \href{b2530a8}{all bijective functions are invertible},
  this completes the proof.
\end{proof}

\Proposition{$ST=I$ ↔︎ $TS=I$ on vector spaces of the same dimension}\label{ddb6618}

Let $V,W$ be vector spaces such that $\dim V=\dim W$ and let
$T\in\href{ab1f2fb}{\L(V,W)}$ and $S\in\L(W,V)$. Then $ST=I$ if and only if
$TS=I$.

\begin{proof}
  First suppose $ST=I$. If $v\in V$ and $T(v)=0$, then
  \begin{align*}
    v &=I(v)                                                 \\
      &=(ST)(v)                                              \\
      &=S(T(v))\desc{\href{a6afdc2}{product of linear maps}} \\
      &=S(0)                                                 \\
      &=0\desc{\href{c5eb127}{linear maps take 0 to 0}}
  \end{align*}

  So then $T$ has a trivial kernel, which \href{f68db52}{implies} that it is
  injective. By \autoref{bd0d827}, $T$ is invertible.

  Now multiply both sides of the equation $ST=I$ by $T^{-1}$ on the right,
  obtaining $S=T^{-1}$. Thus $TS=TT^{-1}=I$ as desired.

  By symmetry, the converse holds ($TS=I\implies ST=I$), and the proof is
  complete.
\end{proof}
