\subsection{Eigenvalues and Eigenvectors}\label{bd04270}

\Proposition{Equivalent conditions to be an eigenvalue}\label{a96a1a4}

Suppose $V$ is finite-dimensional, $T\in\href{ab1f2fb}{\L(V)}$, and
$\lambda\in\F$. Then the following are equivalent:
\begin{enumerata}
  \item $\lambda$ is an \href{e174ec3}{eigenvalue} of $T$.
  \item $T-\lambda I$ is not injective.
  \item $T-\lambda I$ is not surjective.
  \item $T-\lambda I$ is not invertible.
\end{enumerata}

where $I\in\L(V)$ is the identity operator, thus $I(v)=v$ for all $v\in V$.

\begin{proof}
  Since
  $$
    T(v)=\lambda v\iff(T-\lambda I)v=0,
  $$

  by \autoref{f68db52}, conditions (a) and (b) are equivalent.

  Conditions (b), (c), and (d) are equivalent by \autoref{bd0d827}.
\end{proof}

\Proposition{Linearly independent eigenvectors}\label{a272503}

Let $V$ be a vector space, and let $T\in\href{ab1f2fb}{\L(V)}$. Then every list
of \href{ac14802}{eigenvectors} of $T$ corresponding to distinct
\href{e174ec3}{eigenvalues} of $T$ is \href{c133a44}{linearly independent}.

\begin{proof}
  When $\dim V=0$ or $\dim V=1$, the result is trivial. So we focus on the case
  when $\dim V\geq2$.

  Suppose the desired result is false. Then there exists a smallest positive
  integer $m$ such that there exists a linearly dependent list
  $\iter{v_1}{v_m}$ of eigenvectors of $T$ corresponding to distinct
  eigenvalues $\iter{\lambda_1}{\lambda_m}$ of $T$ (note that $m\geq2$ because
  \href{ac14802}{eigenvectors are non-zero}).

  So there exist $\iter{a_1}{a_m}\in\F$, all non-zero because of the minimality
  of $m$, such that
  $$
    a_1v_1+\ldots+a_mv_m=0
  $$

  Applying $T-\lambda_mI$ to both sides yields
  $$
    a_1(\lambda_1-\lambda_m)v_1+\ldots+a_{m-1}(\lambda_{m-1}-\lambda_m)v_{m-1}=0
  $$

  Because eigenvalues $\iter{\lambda_1}{\lambda_m}$ are distinct, none of the
  coefficients above equal 0. Thus $\iter{v_1}{v_{m-1}}$ is a linearly
  independent list of eigenvectors of $T$ corresponding to distinct
  eigenvalues, contradicting the minimality of $m$. This contradiction
  completes the proof.
\end{proof}

\Proposition{Operators can't have more eigenvalues than dimension of vector space}\label{a36bdd2}

Suppose $V$ is a finite-dimensional vector space. Then each
\href{bd31d9c}{operator} on $V$ has at most $\dim V$ distinct
\href{e174ec3}{eigenvalues}.

\begin{proof}
  Let $T\in\href{ab1f2fb}{\L(V)}$. Suppose $\iter{\lambda_1}{\lambda_m}$ are
  distinct eigenvalues of $T$. Let $\iter{v_1}{v_m}$ be corresponding
  eigenvectors. Then \autoref{a272503} implies that the list $\iter{v_1}{v_m}$
  is linearly independent. Thus by \autoref{d8487b6}, $m\leq\dim V$, as desired.
\end{proof}

\Proposition{Polynomial of a linear map is linear}\label{a3468aa}

Let $p\in\mathcal P(\F)$ be the \href{df84c07}{polynomial function}. Let $V,W$
be vector spaces, and let $T\in\href{ab1f2fb}{\L(V,W)}$. Then $p(T)\in\L(V)$.

\begin{proof}
  Let $u,v\in V$. Then
  \begin{align*}
    p(T)(u+v) &=a_0(u+v)+a_1T(u+v)+a_2T^2(u+v)+\ldots                       \\
              &=[a_0u+a_1T(u)+a_2T(u)+\ldots]+[a_0v+a_1T(v)+a_2T(v)+\ldots] \\
              &=p(T)(u)+p(T)(v)
  \end{align*}

  and for all $\lambda\in\F$, we have
  \begin{align*}
    p(T)(\lambda v) &=a_0(\lambda v)+a_1T(\lambda v)+a_2T^2(\lambda v)+\ldots \\
                    &=\lambda[a_0v+a_1T(v)+a_2T(v)+\ldots]                    \\
                    &=\lambda p(T)(v)
  \end{align*}
\end{proof}

\Proposition{The polynomial function $p$ is linear}\label{e660331}

Let $p\in\mathcal P(\F)$ be the \href{df84c07}{polynomial function}. Let $V,W$
be vector spaces, and let $T\in\href{ab1f2fb}{\L(V,W)}$. Then the map $p\mapsto
p(T)$ is linear.

\begin{proof}
  Here, we fix $T\in\L(V)$. Let $f:\mathcal P(\F)\to\L(V)$ be defined by
  $f(p)=p(T)$. Our goal is to show that $f$ is \href{d7d1925}{linear}.

  Let $p,q\in\mathcal P(\F)$, where $\iter{a_0}{a_m}$ are the coefficients of
  $p$, and $\iter{b_1}{b_m}$ are the coefficients of $q$. Then
  $$
    f(p+q)=(p+q)(T)\href{df84c07}{=}p(T)+q(T)=f(p)+f(q)
  $$

  And then for all $\lambda\in\F$, we have
  $$
    f(\lambda p)=(\lambda p)(T)\href{df84c07}{=}\lambda(p(T))=\lambda f(p)
  $$

  Hence $f$ is linear, as desired.
\end{proof}

\Proposition{Kernel and range of $p(T)$ are invariant under $T$}\label{a22eafd}

Let $p\in\mathcal P(\F)$ be a \href{df84c07}{polynomial}. Let $V$ be a vector
space and let $T\in\href{ab1f2fb}{\L(V)}$. Then $\ker p(T)$ and $\range p(T)$
are invariant under $T$.

\begin{proof}
  Suppose $u\in\ker p(T)$. Then $p(T)(u)=0$. Thus,
  \begin{align*}
    [p(T)][T(u)] &=T(p(T)(u))\desc{\href{b42b8cd}{associativity} of linear maps} \\
                 &=T(0)                                                          \\
                 &\href{c5eb127}{=}0
  \end{align*}

  Hence $T(u)\in\ker p(T)$, and thus $\ker p(T)$ is invariant under $T$.

  Now suppose $u\in\range p(T)$. Then there exists $v\in V$ such that
  $u=p(T)(v)$. Thus
  \begin{align*}
    T(u) &=T(p(T)(v))                                                    \\
         &=p(T)(T(v))\desc{\href{b42b8cd}{associativity} of linear maps}
  \end{align*}

  Hence $T(u)\in\range T$, and so $\range p(T)$ is invariant under $T$.
\end{proof}
