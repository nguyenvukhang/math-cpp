\chapter{\texttt{core::counting}}\label{ec1e13a}

\subsection{Basics}\label{e84f7ed}

\Lemma{Breaking apart the binomial coefficient}\label{e676e33}

$$
  k\cdot\binom nk = n\cdot\binom{n-1}{k-1}
$$

\begin{proof}
  By the \href{aff0ae6}{binomial coefficient formula},
  \begin{align*}
    \binom{n-1}{k-1} &=\frac{(n-1)!}{(k-1)![(n-1)-(k-1)]!} \\
                     &=\frac{(n-1)!}{(k-1)!(n-k)!}\Tag{*}
  \end{align*}

  So then
  \begin{align*}
    \binom nk &=\frac{n!}{k!(n-k)!}\desc{by \href{aff0ae6}{definition}} \\
              &=\frac{n(n-1)!}{k(k-1)!(n-k)!}                           \\
              &=\frac nk\cdot\frac{(n-1)!}{(k-1)!(n-k)!}                \\
              &=\frac nk\cdot\binom{n-1}{k-1}\desc{from $(*)$}
  \end{align*}

  This completes the proof.
\end{proof}

\Lemma{}\label{e39bc84}

$$
  x\binom{x+r-1}{r-1}=r\binom{x+r-1}r
$$

This may be useful when trying to make the coefficient outside the brackets a
constant in a sum.

\begin{proof}
  \begin{equation*}
    x\binom{x+r-1}{r-1}
    =x\frac{(x+r-1)!}{(r-1)!x!}
    =r\frac{(x+r-1)!}{r!(x-1)!}
    =r\binom{x+r-1}r
  \end{equation*}
\end{proof}

\Theorem{Newton's binomial theorem}\label{d972e65}

Let $n\in\Z_+$ and let $x,y\in\R$. Then
$$
  (x+y)^n=\sum_{k=0}^n\binom nkx^ny^{n-k}
$$
