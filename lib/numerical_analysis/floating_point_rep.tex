\subsection{Floating point representations}\label{e5cedb7}

\Remark{Computer representation of real numbers}\label{ede07b3}

Any $x\in\R$ is accurately representable by an infinite sequence of digits.
This comes from \href{bb3cf6b}{$\Q$ being dense in $\R$}, so any $x\in\R$ can
be approached to arbitrary accuracy by a sequence of rational numbers.

Thus, we can write
$$
  x=\pm(1.d_1d_2\ldots d_t\ldots)\times2^e
$$

where $e$ is the integer \textit{exponent}. Each $d_i$ can be either 0 or 1.

\Definition{Floating point systems}\label{de15ad9}

A floating point system can be characterized by four values $(\beta,t,L,U)$,
where
\begin{itemize}
  \item $\beta$ is the base of the number system,
  \item $t$ is the precision (number of digits),
  \item $L$ is the lower bound on exponent $e$,
  \item $U$ is the upper bound on exponent $e$.
\end{itemize}

We can then generalize the binary representation in \autoref{ede07b3} by
$$
  \fl(x)=\pm\left(
  \frac{\tilde d_0}{\beta^0}+\frac{\tilde d_1}{\beta^1}+\ldots
  \frac{\tilde d_{t-1}}{\beta^{t-1}}\right)\times\beta^e
$$

with the constraint that $L\leq e\leq U$.

\Remark{Rounding unit}\label{e4a24af}

For a general floating point system $(\beta,t,L,U)$ the rounding unit is
$$
  \eta=\frac12\beta^{1-t}.
$$

We use this to bound the relative error:
$$
  \text{relative error}=\frac{|x-\fl(x)|}{|x|}\leq\eta
$$
