\subsection{Polynomials}\label{d68e870}

\Definition{Legendre polynomials}\label{e9ef1ac}

The $k$-th Legendre polynomial is given by
$$
  q_k(x):=\frac1{2^kk!}\frac{d^k}{dx^k}(x^2-1)^k
$$

These are obtained when applying \href{ee6ac50}{Gram-Schmidt} to a basis of the
\href{b9935c8}{inner product space} formed by the set of all polynomials with
the inner product defined by
$$
  \inner pq:=\int_{-1}^1p(x)q(x)\,dx
$$

These are useful for solving (numerical) integration problems.

\Definition{Chebyshev polynomials*}\label{e7283ca}

The $k$-th Chebyshev polynomial is given by
\begin{align*}
  T_k(x) &:=\cos(n\arccos x)                       \\
         &\href{a0c2aca}{=}2xT_{k-1}(x)-T_{n-2}(x)
\end{align*}

These are obtained when applying \href{ee6ac50}{Gram-Schmidt} to a basis of the
\href{b9935c8}{inner product space} formed by the set of all polynomials with
the inner product defined by
$$
  \inner pq:=\int_{-1}^1p(x)q(x)\frac1{\sqrt{1-x^2}}\,dx
$$

\Definition{Taylor series}\label{ab7e885}

Given an infinitely differentiable function $f:X\to Y$, its Taylor series at
$a\in X$ is
$$
  \sum_{k=0}^\infty\frac{f^{(k)}(a)}{k!}(x-a)^k=f(a)+f'(a)(x-a)+\frac{f''(a)}2(x-a)^2+\ldots
$$

The $n$-th Taylor polynomial is given by
$$
  p_n(x):=\sum_{k=0}^n\frac{f^{(k)}(a)}{k!}(x-a)^k
$$

and this has the property that $p_n(a)=f(a)$, $p_n'(a)=f'(a)$,
$p_n''(a)=f''(a)$, and so on.

Note that we need a degree $n$ Taylor polynomial to match the function value at
$a$ and the values of the first $n$ derivatives at $a$.

\Definition{Piecewise constant interpolation}\label{c6b82a5}

This is the bare minimum interpolation. Given \href{a33bce0}{interpolating
nodes} $\iter{(x_0,y_0)}{(x_n,y_n)}$, sorted in ascending order by $x$'s, we
define
\begin{equation*}
  a_k:=\frac{x_k-x_{k-1}}2,
  b_k:=\frac{x_{k+1}-x_k}2\with{(k=\iter1{n-1})}
\end{equation*}

and then estimate $f(x)$ using $y_k$, where $a_k\leq x<b_k$. That is, we find
the interpolating node closest to $x$, and use that value to estimate $f(x)$.
(for points outside the interpolating nodes, we still use the value at the
closest interpolating node as estimation)

\Definition{Piecewise linear interpolation}\label{b2a543c}

Given interpolating nodes $\iter{(x_0,y_0)}{(x_n,y_n)}$ in ascending order by
$x$'s, we estimate the value of $f(x)$ by finding $x_k$ such that
$x\in[x_k,x_{k+1})$, and then taking
$$
  f(x)=y_k+\frac{y_{k+1}-y_k}{x_{k+1}-x_k}x
$$

This can be rewritten as
$$
  f(x)=y_{k+1}\frac{x-x_k}{x_{k+1}-x_k}+y_k\frac{x_{k+1}-x}{x_{k+1}-x_k}
$$

\Definition{Piecewise quadratic interpolation}\label{aeef34f}

Given $2n+1$ interpolating nodes $\iter{(x_0,y_0)}{(x_{2n},y_{2n})}$ in
ascending order by $x$'s, to estimate the value of $f(x)$, we first obtain
$x_k$ such that $k$ is even and $x\in[x_k,x_{k+2})$.

Then, find the unique parabola $f_k(x):=ax^2+bx+c$ that passes through the
points $(x_{k},y_{k})$, $(x_{k+1},y_{k+1})$, and $(x_{k+2},y_{k+2})$. Formally,
that
$$
  f_k(x_j)=y_j\with{(j=k,k+1,k+2)}
$$

Finally, we estimate $f(x)$ by using $ax^2+bx+c$.

Note that this interpolation function is continuous but not differentiable at
each $x_k$ if $k$ is even.

\Definition{Differentiable piecewise quadratic interpolation}\label{b29ed82}

Given interpolating nodes $\iter{(x_0,y_0)}{(x_n,y_n)}$ in ascending order by
$x$'s, instead of using three points per parabola \href{aeef34f}{like before},
we use two points but now add the constraint that at each point, two
neighboring parabolas need to meet at with the same first \textit{and} second
derivative.

Consider two adjacent interpolating nodes $x_k$ and $x_{k+1}$. We then set
these constraints on the parabola $p_k:[x_k,x_{k+1})\to\R$:
\begin{enumerati}
  \item It must pass through these two points:
  $$
    p_k(x_k)=y_k,\quad p_k(x_{k+1})=y_{k+1}
  $$
  \item It must match the gradient of the parabola to its right at $x_{k+1}$:
  $$
    p_k'(x_{k+1})=p_{k+1}'(x_{k+1})
  $$
  \item It must match the second derivative of the parabola to its right at
        $x_{k+1}$:
  $$
    p_k''(x_{k+1})=p_{k+1}''(x_{k+1})
  $$
\end{enumerati}

This results in an interpolating function that is continuous and twice
differentiable.
