\subsection{QR via Householder reflections}\label{e4faa2d}

\Definition{Householder reflection}\label{d7e9a69}

The reflection hyperplane can be defined by its normal vector, a unit vector
$v$ that is \href{d9735e5}{orthogonal} to the hyperplane. The reflection $x$
about this hyperplane is the linear transformation:
$$
  x\mapsto x-2vv^*x
$$

Observe that since $\norm v=1$, then $vv^*x$ is the \href{fc332ef}{projection}
of $x$ onto $v$. Therefore, $-vv^*x$ takes $x$ \textbf{to} the hyperplane.
Which then means that $-2vv^*x$ takes $x$ to its \textbf{reflection} about the
hyperplane.

\Definition{Householder matrix}\label{ae0f3b4}

The matrix constructed from the \href{d7e9a69}{Householder reflection} can be
expressed as
$$
  P=I-2vv^*
$$

where $v$ is the unit vector that decides the hyperplane, $v^*$ is its
conjugate transpose, and $I\in\R^{n\times n}$ is the identity matrix.

The Householder matrix has the following properties:
\begin{enumerati}
  \item it is \href{a633178}{Hermitian}: $P=P^*$. (Because
        \href{dd38b0c}{$vv^*$ is Hermitian}, and the \href{a2ea95a}{sum of
        Hermitian matrices is Hermitian})
  \item it is \href{bb02509}{involutory}: $P=P^{-1}$. (Shown
        \href{d6c51f2}{here})
  \item it is \href{a32560c}{unitary}: $P^{-1}=P^*$. (Follows from it being
        Hermitian and involutory)
  \item $P$ has eigenvalues $\{-1,1\}$. To see this, notice that if $u$ is
  \href{d9735e5}{orthogonal} to $v$ which was used to create the reflector,
  then $Pu=u$. (i.e. $1$ is an eigenvalue of multiplicity $n-1$, since there
  are \href{a0fa485}{$n-1$ linearly independent vectors orthogonal to $v$}).
  Also, $Pv=-v$, and so $-1$ is an eigenvalue with multiplicity $1$.
\end{enumerati}

\Proposition{Householder matrix is involutory}\label{d6c51f2}

Let $P$ be the \href{ae0f3b4}{Householder matrix}. Then $P$ is
\href{bb02509}{involutory}, that is
$$
  P=P^{-1}
$$

\begin{proof}
  \begin{align*}
    PP &=(I-2vv^*)(I-2vv^*)                       \\
       &=I^2-I(2vv^*)-2vv^*I+4(vv^*)(vv^*)        \\
       &=I-4vv^*+4(vv^*)(vv^*)                    \\
       &=I-4vv^*+4v(v^*v)v^*                      \\
       &=I-4vv^*+4vv^* \desc{$v^*v=\norm{v}^2=1$} \\
       &=I
  \end{align*}
\end{proof}

\Remark{Using Householder reflections for QR factorization}\label{a77c7b1}

Let $A\in\R^{m\times n}$, with $m\geq n$ as in \href{c465f7c}{QR
factorization}. Our goal is to decompose it into
$$
  A=QR
$$

We want to reflect $a_1$, the first column vector of $A$, to
$\norm{a_1}\href{c01037d}{e_1}$, since this preserves the length of $a_1$ (and
hence is a reflection) and zeros-out the rest of the rows.

So we pick $v_1:=\norm{a_1}e_1-a_1$. Observe that the hyperplane with $v_1$ as
its normal vector bisects $a_1$ and $\norm{a_1}e_1$, so reflecting $a_1$ about
this hyperplane sends it to $\norm{a_1}e_1$.

Now we compute the first \href{ae0f3b4}{Householder matrix}, $P_1$:
$$
  P_1:=I-2\frac{v_1v_1^*}{v_1^*v_1}
$$

So far, our factorization looks like this:
$$
  P_1A=\begin{bmat}
    \norm{a_1} & *      & \cdots & *      \\
    0          & \vdots & \vdots & \vdots \\
    \vdots     & \vdots & \vdots & \vdots \\
  \end{bmat}
$$

On the next iteration to calculate $P_2$, we set $v_2=\norm{a'_2}e_2-a'_2$,
where $a'_2$ is just $a_2$ but with the first element zero. So then $v_2$ will
have the first element zero, and $P_2$ given by
$$
  P_2:=I-2\frac{v_2v_2^*}{v_2^*v_2}
$$

will have its first row and first column looking like the identity:
$$
  P_2=\begin{bmat}
    1      & 0      & \cdots & 0      \\
    0      & *      & \cdots & *      \\
    \vdots & \vdots & \ddots & \vdots \\
    0      & *      & \cdots & *      \\
  \end{bmat}
$$

Multiplying this to $P_1A$ will zero-out the second column:
$$
  P_2P_1A=\begin{bmat}
    \norm{a_1} & 0          & *      & \cdots & *      \\
    0          & \norm{a_2} & \vdots & \vdots & \vdots \\
    0          & 0          & \vdots & \vdots & \vdots \\
    \vdots     & \vdots     & \vdots & \vdots & \vdots \\
  \end{bmat}
$$

We continue until $P_n\ldots P_1A$ is \href{c39b6bf}{upper triangular}, and
then we call it $R$. (Note that at each $P_k$, we ignore the first $k-1$ rows
when calculating $v_k:=\norm{a'_k}e_k-a'_k$)
$$
  P_n\ldots P_1A=R
$$

So working backwards, since we want $A=QR$,
$$
  Q=(P_n\ldots P_1)^{-1}=P_1^{-1}\ldots P_n^{-1}=P_1\ldots P_n
$$

Note that the last equality holds because \href{ae0f3b4}{Householder matrices
are involutions}.
