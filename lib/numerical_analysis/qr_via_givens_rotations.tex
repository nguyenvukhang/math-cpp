\subsection{QR via Givens rotations}\label{dfa3a95}

\Remark{Motivation for QR with Givens rotations}\label{dfc4864}

Apart from reflections used in Householder matrices, the other operation that
preserve norms is \textbf{rotations}. Here, instead of reflecting each column
vector to their respective \href{c01037d}{canonical basis vector}, the idea now
is to rotate them.

\Remark{Helloworlding Givens rotations}\label{e0785d9}

Let's say we have a column vector (which could be one in many in a matrix) that
we want to operate on using a Givens rotation:
$$
  \begin{pmat}x_1\\x_2\end{pmat},
$$

and we want to end up with
$$
  \begin{pmat}y\\0\end{pmat}
$$

such that $y^2=x_1^2+x_2^2$ (to preserve $\ell_2$ norm).

We can apply a rotation matrix to it:
\begin{equation*}
  \begin{pmat}
    \cos\theta  & \sin\theta \\
    -\sin\theta & \cos\theta \\
  \end{pmat}
  \begin{pmat}x_1\\x_2\end{pmat}
  =\begin{pmat}y\\0\end{pmat},\Tag{*}
\end{equation*}

where $\theta$ is the angle of rotation. We can use $x_1$ and $x_2$ to find
$y$, then use them to find $\theta$ with
$$
  \begin{pmat}
    x_1 & x_2  \\
    x_2 & -x_1 \\
  \end{pmat}
  \begin{pmat}
    \cos\theta \\
    \sin\theta \\
  \end{pmat}
  =\begin{pmat}y\\0\end{pmat}.\desc{equivalent to $(*)$}
$$

\href{b1c3071}{Elimination} and \href{a0fa0a9}{back substitution} yield
$$
  \cos\theta=\frac{x_1y}{x_1^2+x_2^2},\quad
  \sin\theta=\frac{x_2y}{x_1^2+x_2^2}
$$

and through some manipulation we have
$$
  \cos\theta=\frac{x_1}{\sqrt{x_1^2+x_2^2}},\quad
  \sin\theta=\frac{x_2}{\sqrt{x_1^2+x_2^2}}
$$

\Remark{Extending Givens rotations to higher dimensions}\label{b80ae20}

Instead of $\R^2$ in \autoref{e0785d9}, let's deal with $\R^5$. This example
easily extends to $\R^n$ once understood generally.

To make the move
$$
  \begin{pmat}x_1\\x_2\\x_3\\x_4\\x_5\end{pmat}\to
  \begin{pmat}x_1\\y\\x_3\\0\\x_5\end{pmat},
$$

we apply the Givens rotation matrix of the following shape:
\begin{equation*}
  \def\z{\textcolor{slate300}{0}}
  \begin{pmat}
    1  & \z          & \z & \z         & \z \\
    \z & \cos\theta  & \z & \sin\theta & \z \\
    \z & \z          & 1  & \z         & \z \\
    \z & -\sin\theta & \z & \cos\theta & \z \\
    \z & \z          & \z & \z         & 1  \\
  \end{pmat}
  \begin{pmat}
    x_1 \\x_2\\x_3\\x_4\\x_5\\
  \end{pmat}=
  \begin{pmat}
    x_1 \\y\\x_3\\0\\x_5\\
  \end{pmat}
\end{equation*}

\Remark{Applying Givens rotations to QR factorization}\label{e51b996}

Again, we will be working on a fix-sized example of $A\in\R^{3\times2}$, and
this easily extends to $\R^{m\times n}$ once understood generally.

Let $A\in\R^{3\times2}$ be given by
$$
  \begin{pmat}
    a_{11} & a_{12} \\
    a_{21} & a_{22} \\
    a_{31} & a_{32} \\
  \end{pmat}
$$

First, we focus on the first column of $A$ and generate a \href{b80ae20}{Givens
rotation matrix} $G_1$ such that
$$
  G_1\begin{pmat}
    a_{11} \\
    a_{21} \\
    a_{31} \\
  \end{pmat}=\begin{pmat}
    a'_{11} \\
    0       \\
    a_{31}  \\
  \end{pmat}
$$

Applying $G_1$ to $A$ we obtain
$$
  G_1A=\begin{pmat}
    a'_{11} & a'_{12} \\
    0       & a'_{22} \\
    a_{31}  & a'_{32} \\
  \end{pmat}
$$

Note the effect on the rest of the matrix. The only real change is that we've
used $a_{11}$ to zero-out $a_{21}$.

By this same pattern, we generate $G_2$ to use $a'_{11}$ to zero-out $a_{31}$:
$$
  G_2G_1A=\begin{pmat}
    a''_{11} & a''_{12} \\
    0        & a''_{22} \\
    0        & a''_{32} \\
  \end{pmat}
$$

and the last step here will be to generate a final $G_3$ to use $a''_{22}$ to
zero-out $a''_{32}$:
$$
  G_3G_2G_1A=\begin{pmat}
    a'''_{11} & a'''_{12} \\
    0         & a'''_{22} \\
    0         & 0         \\
  \end{pmat}
$$

\Remark{Extracting $QR$ from Givens rotations}\label{cd43082}

Consider \autoref{e51b996} on a general matrix $A\in\R^{m\times n}$. At the end
of the factorization of $k$ steps, for some $k\in\N$, we have
$$
  G_k\ldots G_1A=R
$$

where $R$ has the same dimensions as $A$ and is \href{c39b6bf}{upper
triangular}. But we want to find the $A=QR$ \href{c465f7c}{factorization}, and
$Q$ is still missing.

To obtain $Q$, we need to find
$$
  Q=(G_k\ldots G_1)^{-1}=G_1^{-1}\ldots G_k^{-1}
$$

Each $G_i$ can be obtained by negating the value of $\theta$ used to generate
it, since the rotation by $-\theta$ is the inverse of the rotation by $\theta$.
