\subsection{Special systems}\label{b9466fa}

\Definition{Banded matrices}\label{e385669}

A matrix $A$ is \textit{banded} if all its elements are zero outside a band of
diagonals around the main diagonal. Formally, we have $p,q\in\N$ with
$p,q\in[1,n]$ such that
$$
  a_{ij}=0 \text{ if } i\geq j+p \text{ or } j\geq i+q
$$

The \textbf{bandwidth} is defined as the number of consecutive diagonals that
may contain non-zero elements (i.e. $p+q-1$). We call $p$ the \textit{upper
bandwidth}, and $q$ the \textit{lower bandwidth}.

\Algorithm{LU decomposition for banded matrices}\label{a8ff951}

Given a banded matrix $A\in\R^{n\times n}$ which requires no pivoting, and with
upper bandwidth $p$ and lower bandwidth $q$, the following overwrites the upper
triangular part of $A$ by $U$:

\begin{pseudocode}
  \For $k=1:n-1$ \\
  \tab\For $i=k+1:k+p-1$ \\[0.5em]
  \tab\tab $\ell_{ik}=\dfrac{a_{ik}}{a_{kk}}$ \\[0.5em]
  \tab\tab\For $j=k+1:k+q-1$ \\
  \tab\tab\tab $a_{ij}\gets a_{ij}-\ell_{ik}a_{kj}$ \\
  \tab\tab\End \\
  \tab\End \\
  \End
\end{pseudocode}
