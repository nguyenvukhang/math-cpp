\subsection{LU factorization}\label{d265b53}

\Definition{Gaussian elimination}\label{b1c3071}

This is a general method to solve the system of equations $Ax=b$.

First, use elementary row operations to reduce the given linear system to
\href{c39b6bf}{upper triangular} form. Then solve the resulting system using
\href{a0fa0a9}{backward substitution}.

This works because the solution to $Ax=b$ is \href{fd54a50}{invariant} to
elementary row operations.

\Algorithm{Gaussian elimination}\label{bcdabd2}

Gaussian elimination aims to send all elements below the diagonal to zero
(thereby achieving upper triangularity). On the first iteration,
\begin{enumerate}
  \item subtract $\frac{a_{21}}{a_{11}}\ \times$ the first row from the second
        row.
  \item subtract $\frac{a_{31}}{a_{11}}\ \times$ the first row from the third
        row.
  \item repeat this to the $n$-th row.
\end{enumerate}

This produces
$$
  (A^{(1)}|b^{(1)})=\begin{pmat}[cccc|c]
    a_{11} & a_{12}       & \ldots & a_{1n}       & b_1       \\[0.5em]
    0      & a_{22}^{(1)} & \ldots & a_{2n}^{(1)} & b_2^{(1)} \\[0.5em]
    \vdots & \vdots       & \ddots & \vdots       & \vdots    \\[0.5em]
    0      & a_{n2}^{(1)} & \ldots & a_{nn}^{(1)} & b_n^{(1)} \\
  \end{pmat}
$$

Next, we ignore the first row and first column and apply the same process on
the remaining submatrix. Note that this doesn't modify the first row (because
it is actually untouched) and doesn't modify the first column (because it's all
zeros)

We then obtain the second iterate:
$$
  (A^{(2)}|b^{(2)})=\begin{pmat}[ccccc|c]
    a_{11} & a_{12}       & \ldots       & \ldots & a_{1n}       & b_1       \\[0.5em]
    0      & a_{22}^{(1)} & a_{23}^{(1)} & \ldots & a_{2n}^{(1)} & b_2^{(1)} \\[0.5em]
    \vdots & 0            & a_{33}^{(2)} & \ldots & a_{3n}^{(2)} & b_3^{(2)} \\[0.5em]
    \vdots & \vdots       & \vdots       & \ddots & \vdots       & \vdots    \\[0.5em]
    0      & 0            & a_{n3}^{(2)} & \ldots & a_{nn}^{(2)} & b_n^{(2)} \\
  \end{pmat}
$$

Doing this $(n-1)$ times, we get
$$
  (A^{(n-1)}|b^{(n-1)})=\begin{pmat}[ccccc|c]
    a_{11} & a_{12}       & \ldots       & \ldots & a_{1n}         & b_1         \\[0.5em]
    0      & a_{22}^{(1)} & a_{23}^{(1)} & \ldots & a_{2n}^{(1)}   & b_2^{(1)}   \\[0.5em]
    \vdots & 0            & a_{33}^{(2)} & \ldots & a_{3n}^{(2)}   & b_3^{(2)}   \\[0.5em]
    \vdots & \vdots       & \ddots       & \ddots & \vdots         & \vdots      \\[0.5em]
    0      & 0            & 0            & \ldots & a_{nn}^{(n-1)} & b_n^{(n-1)} \\
  \end{pmat}
$$

where $A^{(n-1)}$ is upper triangular, and hence can be solved by
\href{a0fa0a9}{backward substitution}.

The primitive implementation of Gaussian elimination does not account for
possible zero-division nor errors magnified by division by small numbers.

In pseudocode, this is written as

\begin{pseudocode}
  \For $k=1:n-1$ \\
  \tab\For $i=k+1:n$ \\[0.5em]
  \tab\tab $\ell:=\dfrac{a_{ik}}{a_{kk}}$ \\[0.5em]
  \tab\tab\For $j=k+1:n$ \\
  \tab\tab\tab $a_{ij}\gets a_{ij}-\ell\cdot a_{kj}$ \\
  \tab\tab\End \\
  \tab\tab $b_i\gets b_i-\ell\cdot b_k$ \\
  \tab\End \\
  \End
\end{pseudocode}

\Remark{Cost of Gaussian elimination}\label{d91901b}

The cost of Gaussian elimination implemented \href{bcdabd2}{here} is given by
$$
  \frac23n^3-\frac23n
$$

\begin{compute}
  \begin{align*}
    \sum_{k=1}^{n-1}\sum_{i=k+1}^n(3+([n-(k+1)]+[n-k]))
     &=\sum_{k=1}^{n-1}\sum_{i=k+1}^n2n-2k+2                                \\
     &=\sum_{k=1}^{n-1}(n-k)(2n-2k+2)                                       \\
     &=2\sum_{k=1}^{n-1}(n-k)^2 +2\sum_{k=1}^{n-1}(n-k)                     \\
     &=2\sum_{k=1}^{n-1}k^2 +2\sum_{k=1}^{n-1}k                             \\
     &=-2n^2+2\left(\frac{n^3}3+\frac{n^2}2+\frac n6\right)+2\frac{n(n-1)}2
  \end{align*}

  % [✓ verified]
  %
  % ```python
  % n = 100
  % c = 0
  % r = lambda x, y: range(x, y + 1)
  % for k in r(1, n - 1):
  %     for i in r(k + 1, n):
  %         c += 3
  %         for j in r(k + 1, n):
  %             c += 2
  %         c -= 1  # account for the extra + caused by the above loop
  % print(c, (2 / 3) * (n**3) - (2 / 3) * n)
  % ```
\end{compute}

\Definition{LU decomposition}\label{f8b347e}

The process of \href{b1c3071}{Gaussian elimination} in fact decomposes matrix
$A$ into a product $L\times U$ of a \href{ce94591}{lower triangular matrix} $L$
and an \href{c39b6bf}{upper triangular matrix} $U$.

Decomposing $A$ into $LU$ is known as LU decomposition.

\Remark{LU decomposition}\label{dbfca19}

Consider the first stage of \href{bcdabd2}{Gaussian elimination}, where we
obtain $A^{(1)}$ from $A$. We can capture the effect of this step by a lower
triangular matrix $M^{(1)}$ given by
$$
  M^{(1)}=\begin{pmat}
    1          &   &   &        &   \\
    -\ell_{21} & 1 &   &        &   \\
    -\ell_{31} &   & 1 &        &   \\
    \vdots     &   &   & \ddots &   \\
    -\ell_{n1} &   &   &        & 1 \\
  \end{pmat}
$$

so that we have
$$
  A^{(1)}=M^{(1)}A\Quad\text{and}\Quad
  b^{(1)}=M^{(1)}b
$$

likewise, the effect of the second step of Gaussian elimination can be
represented by $M^{(2)}$, given by
$$
  M^{(2)}=\begin{pmat}
    1 &            &   &        &   \\
      & 1          &   &        &   \\
      & -\ell_{32} & 1 &        &   \\
      & \vdots     &   & \ddots &   \\
      & -\ell_{n2} &   &        & 1 \\
  \end{pmat}
$$

so that
$$
  A^{(2)}=M^{(2)}A^{(1)}=M^{(2)}M^{(1)}A\Quad\text{and}\Quad
  b^{(2)}=M^{(2)}b^{(1)}=M^{(2)}M^{(1)}b
$$

So then we can see that $LU$ is obtained by
$$
  U=A^{(n-1)}=M^{(n-1)}\cdots M^{(1)}A
$$

and
$$
  L=[M^{(1)}]^{-1}
  [M^{(2)}]^{-1}\cdots
  [M^{(n-1)}]^{-1}
$$

It can be directly verified that $[M^{(2)}]^{-1}$ is simply $M^{(2)}$ but with
its $\ell$ terms negated:
$$
  [M^{(2)}]^{-1}=\begin{pmat}
    1 &           &   &        &   \\
      & 1         &   &        &   \\
      & \ell_{32} & 1 &        &   \\
      & \vdots    &   & \ddots &   \\
      & \ell_{n2} &   &        & 1 \\
  \end{pmat}
$$

In turn, it can be directly verified that $L$ is given by
\begin{align*}
  L &=[M^{(1)}]^{-1}[M^{(2)}]^{-1}\cdots[M^{(n-1)}]^{-1} \\
    &=\begin{pmat}
        1         &           &        &              &   \\
        \ell_{21} & 1         &        &              &   \\
        \ell_{31} & \ell_{32} & 1      &              &   \\
        \vdots    & \vdots    & \ddots & \ddots       &   \\
        \ell_{n1} & \ell_{n2} & \ldots & \ell_{n,n-1} & 1 \\
      \end{pmat}
\end{align*}

And so $L$ encodes all the moves made in Gaussian elimination!

\Algorithm{Solving linear systems with LU decomposition}\label{db12db5}

Given a linear system $Ax=b$, with $A$ as a real non-singular matrix, apply
\href{f8b347e}{LU decomposition} first:
$$
  A=LU
$$

Then, use \href{a181277}{forward substitution} to solve for $y$ in
$$
  Ly=b
$$

And then, use \href{a0fa0a9}{backward substitution} to solve for $x$ in
$$
  Ux=y
$$
