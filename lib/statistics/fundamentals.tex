\subsection{Fundamentals}\label{e028258}

For this entire chapter, given any events $A$ and $B$, we will denote the event
$A\cap B$ as $AB$.

\Theorem{Multinomial Coefficients}\label{ce33c96}

The (natural) number computed by
$$
  \binom{n}{k_1,k_2,\ldots,k_r}:=\frac{n!}{k_1!k_2!\cdots k_r!}
$$

gives the number of possible divisions of $n$ distinct objects into $r$ groups
of respective sizes $\iter{k_1}{k_r}$, where $k_1+\ldots+k_r=n$.

\Theorem{Multinomial Theorem}\label{fa4d805}

For natural numbers $\iter{x_1}{x_r}$, and using notation from
\autoref{ce33c96},
$$
  (x_1+\ldots+x_r)^n
  =\sum_{(\iter{k_1}{k_r}):k_1+\ldots+k_r=n}\binom{n}{k_1,k_2,\ldots,k_r}
  x_1^{k_1}x_2^{k_2}\ldots x_r^{k_r}
$$

That is, the RHS is the sum across all possible values of $(\iter{k_1}{k_r})$
such that their sum is $n$, and $k_i\geq0$ for all $i=\iter1r$.

\Definition{Sample space}\label{dcc8547}

This is defined as the set of all possible outcomes. Usually denoted by $S$.

For example, the sample space of a coin toss can be written as the set
$\{H,T\}$.

Sample spaces can be infinite too, such as when a coin is tossed until it lands
on heads:
$$
  S=\{(H),(T,H),(T,T,H),(T,T,T,H),\ldots\}
$$

\Definition{Mutually exclusive events}\label{a16826f}

Two events $E$ and $F$ are mutually exclusive if $E\cap F=\emptyset$.

\Definition{Complement of events}\label{a89283f}

The complement of event $E$, denoted as $E^c$ is the event consisting of all
outcomes that are not in $E$.

\Proposition{Properties of event operations}\label{fc3d98b}

\begin{itemize}
  \item $\cap$ and $\cup$ are commutative.
  \item $\cap$ and $\cup$ are associative.
  \item $\cap$ and $\cup$ are distributive both ways.
        \begin{align*}
          (E\cup F)G &=(E\cap G)\cup(F\cap G) \\
          (E\cap F)G &=(E\cup G)\cap(F\cup G)
        \end{align*}
\end{itemize}

% Proof probably requires some definitions from Set Theory, which we don't have
% yet

\Theorem{De Morgan's Law}\label{c28492b}

\begin{enumerati}
  \item The complement of unions is the intersection of complements:
  $$
    \biggl(\bigcup_{i=1}^nE_i\biggr)^c=\bigcap_{i=1}^nE_i^c
  $$
  \item And the complement of intersections is the intersection of unions:
  $$
    \biggl(\bigcap_{i=1}^nE_i\biggr)^c=\bigcup_{i=1}^nE_i^c
  $$
\end{enumerati}

\begin{proof}
  Let $I:=\{\iter1n\}$ be the index set.

  \proofp{(i)} Let $x\in(\bigcup_{i=1}^nE_i)^c$. Then
  \begin{align*}
    x\notin\bigcup_{i=1}^nE_i
     &\implies x\notin E_i,\with{\forall i\in I} \\
     &\implies x\in E_i^c,\with{\forall i\in I}  \\
     &\implies x\in\bigcap_{i=1}^nE_i^c
  \end{align*}

  Hence we have established that
  $(\bigcup_{i=1}^nE_i)^c\subseteq\bigcap_{i=1}^nE_i^c$

  Now let $x\in\bigcap_{i=1}^nE_i^c$. Then
  \begin{align*}
    x\in E_i^c,\with{\forall i\in I}
     &\implies x\notin E_i,\with{\forall i\in I}      \\
     &\implies x\notin\bigcup_{i=1}^nE_i              \\
     &\implies x\in\biggl(\bigcup_{i=1}^nE_i\biggr)^c
  \end{align*}

  Hence shown that $\bigcap_{i=1}^nE_i^c\subseteq(\bigcup_{i=1}^nE_i)^c$, and
  so altogether we have
  $$
    \biggl(\bigcup_{i=1}^nE_i\biggr)^c=\bigcap_{i=1}^nE_i^c
  $$

  \proofp{(ii)} We can apply (i) to help us.
  \begin{align*}
    \biggl(\bigcup_{i=1}^nE_i^c\biggr)^c
     &=\bigcap_{i=1}^n(E_i^c)^c\desc{by (i)}                      \\
     &=\bigcap_{i=1}^nE_i                                         \\
    \bigcup_{i=1}^nE_i^c
     &=\biggl(\bigcap_{i=1}^nE_i\biggr)^c\desc{invert both sides}
  \end{align*}
\end{proof}
