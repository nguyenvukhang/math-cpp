\subsection{Definitions}\label{e9f040b}

\Definition{Sigmoidal function}\label{cf39bf4}

A function $\sigma:\R\to\R$ is called a sigmoidal function if it satisfies
$$
  \lim_{x\to L}\sigma(x)=\begin{cases}
    0 & \text{if } L=-\infty \\
    1 & \text{if } L=\infty  \\
  \end{cases}
$$

\Definition{Tauber-Wiener function}\label{e65fc0e}

If a function $g:\R\to\R$ satisfies that all the linear combinations
$$
  \sum_{i=1}^Nc_ig(\lambda_ix+\theta_i)\with{(\lambda_i,\theta_i,c_i\in\R,\ i=\iter1N)}
$$

are \href{e14819a}{dense} in every $C[a,b]$, then $g$ is called a
\textbf{Tauber-Wiener} function. We will subsequently write $g\in\TW$.

\Remark{Compact set in a Banach space}\label{afe68fa}

Suppose that $(X,\norm{\,\cdot\,}_X)$ is a \href{f894cb0}{Banach space}. Then
$V\subseteq X$ is a compact set in $X$ if for every sequence $\{x_n\}_{n\in\N}$
with all $x_n\in V$, there is a subsequence $\{x_{n_k}\}$ which converges to
some element $x\in V$.

It is well known that if $V\subseteq X$ is a compact set in $X$, then for any
$\delta>0$, there is a $\delta$-net $N(\delta)=\{\iter{x_1}{x_{n(\delta)}}\}$
which converges to some element $x\in V$, with all $x_i\in V$ for all
$i=\iter1{n(\delta)}$.

% Should it be for every x in V?
% We'll wait to see this being used in the proofs.
That is, for every $x\in X$, there is some $x_i\in N(\delta)$ such that
$\norm{x_i-x}_X<\delta$.

\Notation{Paper-specific notation}\label{aa7beb7}

For the rest of this paper's analysis, we will use these global variables:
\begin{itemize}
  \item $X:=$ \href{f894cb0}{Banach space} over the field $\R$ with norm
        $\norm{\,\cdot\,}_X$, where $\norm{\,\cdot\,}_X:X\to\R$.
  \item $K:=$ a compact set.
  \item $C(K):=$ the Banach space of all continuous functions defined on $K$,
        with norm
        $$
          \norm{f}_{C(K)}:=\max_{x\in K}|f(x)|
        $$

        That is, if $f\in C(K)$, then $f:K\to\R$ is continuous.
  \item $\TW:=$ set of all \href{e65fc0e}{Tauber-Wiener} functions.
  \item $\mathcal S(\R^n):=$ Schwartz functions in tempered distribution theory.
        (i.e. rapidly decreasing and infinitely differentiable functions)
  \item $\mathcal S'(\R^n):=$ Tempered distributions. (i.e. linear continuous
        functionals defined on $\mathcal S(\R^n)$)
  \item $C^\infty(\R^n)$:= Infinitely differentiable functions.
  \item $C^\infty_c(\R^n):=$ Infinitely differentiable functions with compact
        support in $\R^n$.
  \item $C_p[-1,1]^n:=$ All periodic functions with period two with respect
        to every variable $x_i$, with $i=\iter1n$.
\end{itemize}
