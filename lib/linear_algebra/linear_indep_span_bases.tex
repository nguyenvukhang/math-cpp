\subsection{Linear independence, span, and bases}\label{d63fa20}

\Lemma{Steinitz exchange lemma}\label{afcb195}

Let $U,W$ be finite subsets of a \href{fc83050}{vector space} $V$. If $U$ is a
set of \href{c133a44}{linearly independent} vectors, and $W$
\href{ac574be}{spans} $V$, then
\begin{enumerati}
  \item $|U|\leq|W|$
  \item There is a set $W'\subseteq W$ with $|W'|=|W|-|U|$ such that $U\cup W'$
        spans $V$.
\end{enumerati}

\begin{proof}
  \def\W#1#2{\iter{w_{#1}}{w_{#2}}}
  \def\U#1#2{\iter{u_{#1}}{u_{#2}}}

  Suppose $U=\{\U1m\}$ and $W=\{\W1n\}$. We wish to show that $m\leq n$, and
  that after rearranging $w_j$ if necessary, the set $\{\U1m,\W{m+1}n\}$ spans
  $V$. We proceed by induction on $m$.

  For the base case, suppose $m=0$. Then the claim holds because there are no
  vectors $u_i$, and the set $\{\W1n\}$ spans $V$ by the hypothesis.

  For the inductive step, assume that the proposition is true for $m-1$. By the
  inductive hypothesis, we may reorder the $w_i$ so that $\{\U1{m-1},\W{m}n\}$
  spans $V$. Since $u_m\in V$, there exists coefficients $\iter{a_1}{a_n}$ such
  that
  \begin{equation*}
    u_m=\sum_{i=1}^{m-1}a_iu_i+\sum_{j=m}^na_jw_j\Tag{*}
  \end{equation*}

  At least one of the $a_j$ must be non-zero, otherwise $(*)$ contradicts the
  \href{c133a44}{linear independence} of $\{\U1m\}$. It follows that $m\leq n$.

  Next, by reordering $\iter{a_mw_m}{a_nw_n}$ if necessary, we may assume that
  $a_m$ is non-zero. Therefore, we have
  $$
    w_m=\frac1{a_m}\biggr(
    u_m-\sum_{i=1}^{m-1}a_iu_i-\sum_{j=m+1}^{n}a_jw_j
    \biggl)
  $$

  In other words, $w_m$ is in the span of $\{\U1{m},\W{m+1}n\}$. Since this
  span contains each of the vectors $\U1{m-1},\W{m}n$, by the inductive
  hypothesis it contains $V$.
\end{proof}

\Lemma{Bound lemma}\label{e054922}

In a (finite-dimensional) \href{fc83050}{vector space} $V$ with a basis of
length $n\in\N_0$, a set of $n+1$ vectors in $V$ is always
\href{c133a44}{linearly dependent}.

\begin{proof}
  Since $V$ has a basis of length $n$, it has a spanning set of length $n$. By
  \href{d8487b6}{this lemma}, all linearly independent lists have length $\leq
  n$. Hence if a list has length $n+1$, it must be linearly dependent.
\end{proof}

\begin{proof}
  \def\V{\{\iter{v_1}{v_{n+1}}\}}
  \def\Z#1{\{\iter{z_#1}{z_n}\}}

  Let $P(N)$ be the statement that in a vector space with a basis of length
  $\leq N-1$, any set of $N$ vectors is linearly dependent.

  With $P(1)$, the statement reads ``In a vector space with a basis of length
  0, any set of 1 vector is linearly dependent". The vector space with an empty
  basis is $\{0\}$, and the only vector in it is $0$, so the only set that
  contains 1 vector is $\{0\}$, which is trivially linearly dependent.

  Now on to the inductive step. We assume that $P(n)$ is true for some
  $n\in\N_0$, and try to show that this implies that $P(n+1)$ it true.

  Let $\Z1$ be a basis of $V$. Consider an arbitrary set of $n+1$ vectors
  $\V\subset V\sans0$ (if any $v_i=0$, then this set is linearly dependent and
  we are done).

  As $v_1\neq0$, we assume that $v_1=\lambda_1z_1+y_1$, with $\lambda_1\neq0$
  and $y_1\in U:=\Span\Z2$. Note that if instead all vectors in $\V$ can be
  written as a linear combination of $\Z2$ alone, then by $P(n)$ they are
  linearly dependent and we are done.

  Next, for each $j=\iter2{n+1}$, there exist $\lambda_j\in\R$ and $y_j\in U$
  such that
  $$
    v_j=\lambda_jz_1+y_j
  $$

  This just comes from $\Z1$ being a basis of $V$. Therefore, for each
  $j=\iter2{n+1}$,
  \begin{align*}
    w_j &:=\lambda_1v_j-\lambda_jv_1                              \\
        &=\lambda_1(\lambda_jz_1+y_j)-\lambda_j(\lambda_1z_1+y_1) \\
        &=\lambda_1v_j-\lambda_jv_1\in U
  \end{align*}

  By the induction hypothesis applied to $U$, the set of $n$ vectors
  $\{\iter{w_2}{w_{n+1}}\}$ are linearly dependent.

  Hence, there exist scalars $\{\iter{\mu_2}{\mu_{n+1}}\}$, not all zero, such
  that
  \begin{equation*}
    0 =\sum_{j=2}^{n+1}\mu_jw_j
    =\sum_{j=2}^{n+1}\mu_j(\lambda_1v_j-\lambda_jv_1)
    =\sum_{j=2}^{n+1}\lambda_1\mu_jv_j+\biggl(-\sum_{j=2}^{n+1}\mu_j\lambda_j\biggr)v_1
  \end{equation*}

  As $\lambda_1\neq0$ and $\mu_j\neq0$ for at least one $j$, we have just shown
  that there exists a non-trivial linear combination of $\V$ that gets 0. Hence
  $\V$ is linearly dependent, i.e. $P(n+1)$ is true.

  This completes the inductive step and hence the inductive proof.
\end{proof}

\Theorem{Uniqueness of basis size}\label{e38814b}

Let $V$ be a finite-dimensional vector space. Let $U,W\subset V$ be
\href{db2477b}{bases} of $V$. Then
$$
  |U|=|W|
$$

and with this uniqueness, we denote $\href{cd4284b}{\dim V}:=|U|$.

\begin{proof}
  Since $U$ is a basis of $V$ with length $|U|$, if $|U|<|W|$, then by the
  \href{e054922}{Bound lemma}, $W$ is linearly dependent. This contradicts the
  fact that $W$ is a basis, and hence we must have $|U|\geq |W|$.

  By symmetry, we also have $|U|\leq |W|$. Hence $|U|=|W|$.
\end{proof}

\Theorem{Basis is smallest spanning set}\label{b0af3c1}

Let $V$ be a finite-dimensional \href{fc83050}{vector space} with a basis of
size $n$. Then any subset $S$ of $V$ with $|S|<n$ does not span $V$.

\begin{proof}
  First, we remove vectors from $S$ that can be expressed as a linear
  combination of other vectors in $S$ until we obtain a linearly independent
  subset $S'$. Observe that $|S'|<|S|<n$ and hence $|S'|\neq n$. Also note that
  $\Span S'=\Span S$.

  By \autoref{e38814b}, $S'$ is not a basis of $V$. Since $S'$ is linearly
  independent and not a basis, it cannot span $V$. Hence
  $$
    \Span S=\Span S'\neq V
  $$

  Hence $S$ does not span $V$.
\end{proof}

\Lemma{Unique representation and linear independence}\label{af98c68}

Let $B$ be a subset of the \href{fc83050}{vector space} $V$ (over field
$\mathcal F$). Then the following are equivalent:
\begin{enumerati}
  \item $B$ is \href{c133a44}{linearly independent}.
  \item Each $v\in\href{ac574be}{\Span}B$ has a unique representation as a
        linear combination of elements in $B$.
\end{enumerati}

\begin{proof}
  ($\implies$) Let $v\in\Span B$ have two representations
  $$
    v=\sum_{i=1}^p\lambda_ib_i=\sum_{i=1}^p\mu_ib_i
  $$

  where $\iter{\lambda_1}{\lambda_p},\,\iter{\mu_1}{\mu_p}\in\mathcal F$ and
  $\iter{b_1}{b_p}\in B$. Then
  $$
    0=\sum_{i=1}^p(\lambda_i-\mu_i)b_i
  $$

  Now since $B$ is \href{c133a44}{linearly independent}, we have
  $$
    \lambda_i-\mu_i=0,\with{\text{for }i=\iter1p}
  $$

  and hence the representation is unique.

  ($\impliedby$) Assume that $B$ is not linearly independent. Our goal is to
  show that there exists a $v\in\Span B$ with a non-unique representation.

  WLOG, assume that $\iter{b_1}{b_n}$ are pairwise distinct (otherwise the
  non-unique representation is trivial), and there is a collection
  $\iter{\lambda_1}{\lambda_n}\in\mathcal F$ such that
  $$
    0=\sum_{i=1}^n\lambda_ib_i
  $$

  By linear dependence, at least one $\lambda_i$ is non-zero. WLOG, let
  $\lambda_1\neq0$. Then
  $$
    1\cdot b_1=b_1=\sum_{i=2}^n-\frac{\lambda_i}{\lambda_1}b_i
  $$

  are two distinct representations of $b_1\in\Span B$.
\end{proof}

\Theorem{$n$ linearly independent vectors span $n$-dimensional spaces}\label{c8f8fd3}

Let $V$ be an $n$-dimensional vector space (in the sense that its bases all
have size $n$). A set of $n$ \href{c133a44}{linearly independent} vectors is a
\href{db2477b}{basis} of $V$. Likewise, $n$ vectors that \href{ac574be}{span}
$V$ is a basis of $V$. That is,
$$
  \iter{z_1}{z_n}\textit{ linearly independent} \iff
  \Span\{\iter{z_1}{z_n}\}=V \iff
  \{\iter{z_1}{z_n}\}\textit{ basis of }V
$$

\begin{proof}
  That $Z$ being a basis implies linear independence and span follows from the
  definition of a basis.

  \proofp{linear independence $\implies$ basis}

  Let $Z:=\{\iter{z_1}{z_n}\}$ be linearly independent. Then for all $v\in
  V\setminus Z$, the \href{e054922}{bound lemma} implies that $Z\cup\{v\}$ is
  linearly dependent (as a set of $n+1$ elements in an $n$-dimensional vector
  space). Hence, all vectors $v\in V\setminus Z$ are linear combinations of
  vectors in $Z$, and hence $Z$ is a basis of $V$.

  \proofp{span $\implies$ basis}

  Assume on the contrary that $\{\iter{z_1}{z_n}\}$ is linearly dependent. Then
  there exists a collection $\iter{\lambda_1}{\lambda_n}$ such that
  $$
    0=\sum_{i=1}^n\lambda_iz_i
  $$

  By linear dependence, at least one $\lambda_i$ is non-zero. WLOG, let
  $\lambda_1\neq0$. Then
  $$
    z_1=\sum_{i=2}^n-\frac{\lambda_i}{\lambda_1}z_i
  $$

  and hence $z_1\in\Span\{\iter{z_2}{z_n}\}$. But then this means that
  $$
    \Span\{\iter{z_2}{z_n}\}=\Span\{\iter{z_1}{z_n}\}=V
  $$

  contradicting that all bases of $V$ have size $n$. Hence we must have that
  $Z$ is linearly independent, making it a basis.
\end{proof}
