\subsection{Basics}\label{fbd96f6}

\Theorem{$\sqrt2$ is an irrational number}\label{c2585a1}

Let $x\in\href{d52c6b7}{\R}$ be a number such that $x^2=2$. Then
$x\notin\href{d52c6b7}{\Q}$.

\begin{proof}
  Let $\sqrt2:=x$ in the above statement. Suppose $\sqrt2$ is rational. Then we
  can write
  $$
    \sqrt2=\frac ab
  $$

  where $a$ and $b$ are integers with no common factor other than 1. Then by
  assumption,
  $$
    2=\frac{a^2}{b^2}
  $$

  and hence
  $$
    2b^2=a^2
  $$

  This says that $a^2$ is even. So $a$ is also even, and $a=2k$ for some
  integer $k$. Then we get
  $$
    2b^2=4k^2
  $$

  So
  $$
    b^2=2k^2
  $$

  But this says that $b^2$ is even, so $b$ is even. It follows that 2 is a
  common factor for $a$ and $b$. This contradicts our assumption of $a$ and
  $b$, and hence $\sqrt2$ is not rational.
\end{proof}

\Principle{Well-ordering Property of $\mathbb N$}\label{cd7c4d1}

Every non-empty subset $S$ of $\href{d52c6b7}{\N}$ has a \textit{least (or
minimum)} element. Formally,
$$
  \forall S\subseteq\N,\ \exists m\in S:\ \forall s\in S,\ m\leq s
$$

Note that $S$ may not have a largest element.

\Theorem{Induction on natural numbers}\label{a824f8c}

Let $S\subseteq\N$. If we have
\begin{enumerati}
  \item $1\in S$, and
  \item for every $k\in\N$, $k\in S\implies k+1\in S$.
\end{enumerati}

Then $S=\N$.

\begin{proof}
  Suppose that $S\neq\N$. Then its complement $\N\setminus S\neq\emptyset$

  By the \href{cd7c4d1}{well-ordering property of $\N$}, there exists a least
  element $m\in\N\setminus S$.

  By (i), we have $m\neq 1$ and hence $m\geq2$. Thus, $m-1\in\N$. Since $m$ is
  the smallest natural number \textit{not} in $S$, we have $m-1\in S$. But by
  (ii), $m=(m-1)+1\in S$, which is a contradition to $m\in\N\setminus S$.
\end{proof}

\Theorem{Principle of Mathematical Induction}\label{b51ca45}

For each $n\in\N$, let $P(n)$ be a statement about $n$. Suppose that
\begin{enumerati}
  \item $P(1)$ is true, and
  \item for every $k\in\N$, if $P(k)$ is true, then $P(k+1)$ is true.
\end{enumerati}

Observe that $1$ can be replaced with any natural number $n_0$, but we would
have only proved that $P$ is true for all natural numbers $\geq n_0$.

\begin{proof}
  Apply \href{a824f8c}{induction on natural numbers} on the set
  $$
    \set{n\in\N}{P(n)\text{ is true}}
  $$
\end{proof}
