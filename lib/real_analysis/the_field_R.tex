\subsection{The field $\mathbb R$}\label{ff004fd}

\Remark{$\mathbb R$ with standard addition and multiplication forms a field}\label{bf61f02}

Let addition $+:\R\times\R\to\R$ and multiplication $\,\cdot\,:\R\times\R\to\R$
be the standard addition and multiplication we are all first taught in grade
school.

Then $(\R,+,\,\cdot\,)$ satisfies all 9 \href{aec6040}{Field Axioms} and thus
forms a field.

\Proposition{Order properties of $\mathbb R$}\label{d49c63e}

There is a \href{a3a60e1}{binary relation} $>$ on $\R$ which has the following
properties (with $a,b,c\in\R$):

\begin{enumerati}
  \item [(\textbf{O1})] If $a>b$, then $a+c>b+c$.
  \item [(\textbf{O2})] If $a>0$ and $b>0$, then $a\cdot b>0$.
  \item [(\textbf{O3})] \textit{(Trichotomy Property)} If $a,b\in\R$, then
  exactly one of the following holds:
  $$
    a>b,\quad a=b,\quad b>a
  $$
  \item [(\textbf{O4})] \textit{(Transitive Property)} If $a>b$ and $b>c$, then
  $a>c$.
\end{enumerati}

Note that $a<b$ is, by definition, the same as $b>a$.

\Theorem{}\label{b069294}

If $a\in\R$ is such that $0\leq a<\epsilon$ for every positive number
$\epsilon$, then $a=0$.

\begin{proof}
  Since $a\geq0$, by definition either $a>0$ or $a=0$. Suppose $a>0$. Then
  $\frac a2>0$.

  Now let $\epsilon:=\frac a2$. Then by assumption, $a<\epsilon=\frac a2$
  \begin{align*}
    a<\frac a2 &\implies 2\cdot a<2\cdot\frac a2=a \\
               &\implies 2a-a<0                    \\
               &\implies a<0
  \end{align*}

  This contradicts that $a>0$. Hence we must have $a=0$.
\end{proof}

\Theorem{Triangle inequality for $\mathbb R$}\label{f1288ad}

For all $a,b\in\R$, we have
$$
  |a+b|\leq|a|+|b|
$$

\begin{proof}
  We have $-|a|\leq a\leq|a|$ and $-|b|\leq b\leq|b|$. Adding, we have
  \begin{align*}
    -(|a|+|b|)\leq a+b\leq |a|+|b|
  \end{align*}

  which implies that $|a+b|\leq|a|+|b|$.
\end{proof}

\Corollary{Corollaries of triangle inequality for $\mathbb R$}\label{f699f4d}

\begin{enumerati}
  \item $\big||a|-|b|\big|\leq|a-b|$
  \item $|a-b|\leq|a|+|b|$
  \item $\big||a|-|b|\big|\leq|a+b|$
\end{enumerati}

\begin{proof}
  By the \href{f1288ad}{triangle inequality},
  $$
    |a|=|(a-b)+b|\leq|a-b|+|b|
  $$

  So
  \begin{equation*}
    |a|-|b|\leq|a-b|\Tag{*}
  \end{equation*}

  By symmetry, we also have
  $$
    |b|-|a|\leq|b-a|
  $$

  which can be rewritten as
  \begin{equation*}
    -(|a|-|b|)\leq|a-b|\Tag{**}
  \end{equation*}

  With $(*)$ and $(**)$, we have (i).

  (ii) is obtained by using $-b$ in the place of $b$ in the
  \href{f1288ad}{triangle inequality}

  (iii) follows from using $-b$ in the place of $b$ in (i).
\end{proof}
