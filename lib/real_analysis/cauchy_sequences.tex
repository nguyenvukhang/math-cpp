\subsection{Cauchy sequences}\label{d1034e6}

\Theorem{Convergent sequences are Cauchy}\label{a1534e3}

Every \href{de3e28a}{convergent} sequence is \href{a8f670d}{Cauchy}.

(The \href{a537c7e}{converse} is also true.)

\begin{proof}
  Suppose $x_n\to x$. Let $\epsilon>0$. Then there exists $K\in\N$ such that
  $$
    |x_n-x|<\frac\epsilon2,\with{\forall n\geq K}
  $$

  It follows that for all $n,m\geq K$,
  \begin{align*}
    |x_n-x_m| &=|(x_n-x)-(x_m-x)|                                            \\
              &\leq|x_n-x|+|x_m-x|\desc{\href{f1288ad}{triangle inequality}} \\
              &<\frac\epsilon2+\frac\epsilon2                                \\
              &=\epsilon
  \end{align*}

  Thus $\{x_n\}$ is Cauchy.
\end{proof}

\Theorem{Cauchy sequences are bounded}\label{e6e340b}

Every \href{a8f670d}{Cauchy sequence} is \href{d5ed299}{bounded}.

\begin{proof}
  Take $\epsilon:=1$. Then since $\{x_n\}$ is Cauchy, there exists $K\in\N$
  such that
  $$
    |x_n-x_m|<1,\with{\forall n,m\geq K}
  $$

  In particular, putting $m:=K$, we obtain
  \begin{equation*}
    |x_n-x_K|<1,\with{\forall n\geq K}\Tag{*}
  \end{equation*}

  It follows that for $n\geq K$, we have
  \begin{align*}
    |x_n| &=|(x_n-x_K)+x_K|                                              \\
          &\leq|x_n-x_K|+|x_K|\desc{\href{f1288ad}{triangle inequality}} \\
          &<1+|x_K|\desc{by $(*)$}
  \end{align*}

  Let $M:=\max\{|x_1|,|x_2|,\ldots,|x_K|,1+|x_K|\}$. Then
  $$
    |x_n|\leq M,\with{\forall n\in\N}
  $$

  so $\{x_n\}$ is bounded.
\end{proof}

\Theorem{Cauchy criterion for sequences}\label{a537c7e}

Every \href{a8f670d}{Cauchy sequence} is \href{de3e28a}{convergent}.

(The \href{a1534e3}{converse} is also true.)

\begin{proof}
  Let $\{x_n\}$ be a \href{a8f670d}{Cauchy sequence}. By \autoref{e6e340b},
  $\{x_n\}$ is bounded. By \href{d277ad0}{Bolzano-Weierstrass}, it has a
  convergent subsequence $\{x_{n_k}\}$. Let $x:=\lim_{n\to\infty}x_{n_k}$.

  We claim that $x_n\to x$.

  Let $\epsilon>0$ be given. Since $\{x_n\}$ is Cauchy, there exists $K_1\in\N$
  such that
  \begin{equation*}
    |x_n-x_m|<\frac\epsilon2,\with{\forall n,m\geq K_1}\Tag{*}
  \end{equation*}

  Since $x_{n_k}\to x$, there exists $K_2\in\N$ such that $K_2\geq K_1$ and
  \begin{equation*}
    |x_{n_k}-x|<\frac\epsilon2,\with{\forall k\geq K_2}
  \end{equation*}

  In particular,
  \begin{equation*}
    |x_{n_{K_2}}-x|<\frac\epsilon2\Tag{**}
  \end{equation*}

  Note that
  $$
    K_2\geq K_1\implies n_{K_2}\geq n_{K_1}\geq K_1
  $$

  and hence $n_{K_2}$ satisfies the role of $m$ in $(*)$.

  Thus for all $n\geq K_1$, one has
  \begin{align*}
    |x_n-x|
     &=|(x_n-x_{n_{K_2}})+(x_{n_{K_2}}-x)|                                            \\
     &\leq|x_n-x_{n_{K_2}}|+|x_{n_{K_2}}-x|\desc{\href{f1288ad}{triangle inequality}} \\
     &<\frac\epsilon2+\frac\epsilon2\desc{by $(*)$ and $(**)$}                        \\
     &=\epsilon
  \end{align*}

  Hence $\{x_n\}\to x$.
\end{proof}

\Theorem{Contractive → Cauchy}\label{ac20bfc}

Every \href{d5c8fb8}{contractive sequence} is \href{a8f670d}{Cauchy} (and so it
is convergent).

\begin{proof}
  Suppose that $\{x_n\}$ is a contractive sequence. Given some fixed
  $C\in(0,1)$, we have
  $$
    |x_{n+2}-x_{n+1}|\leq C|x_{n+1}-x_n|,\with{\forall n\in\N}
  $$

  By applying the above inequality repeatedly, we obtain for all $n\geq2$,
  \begin{align*}
    |x_{n+1}-x_n| &\leq C|x_{n}-x_{n-1}|     \\
                  &\leq C^2|x_{n-1}-x_{n-2}| \\
                  &\vdots                    \\
                  &\leq C^{n-1}|x_2-x_1|
  \end{align*}

  Now if $m>n\geq2$, then
  \begin{align*}
    |x_{m}-x_{n}|
     &=|(x_{m}-x_{m-1})+(x_{m-1}-x_{m-2})+\ldots+(x_{n+1}-x_{n})|                          \\
     &\leq|x_{m}-x_{m-1}|+\ldots+|x_{n+1}-x_{n}|\desc{\href{f1288ad}{triangle inequality}} \\
     &\leq(C^{m-2}+C^{m-3}+\ldots+C^{n-1})|x_2-x_1|                                        \\
     &=C^{n-1}(1+C+\ldots+C^{m-n-1})|x_2-x_1|                                              \\
     &=C^{n-1}\cdot\frac{1-C^{m-n-1}}{1-C}|x_2-x_1|                                        \\
     &\leq\frac{C^{n-1}}{1-C}|x_2-x_1|\Tag{*}
  \end{align*}

  Now let $\epsilon>0$ be given. Since $0<C<1$, we have $C^{n-1}\to0$, and thus
  the expression above also $\to0$. Hence there exists $K\in\N$ such that
  \begin{equation*}
    \frac{C^{n-1}}{1-C}|x_2-x_1|<\epsilon,\with{\forall n\geq K}\Tag{**}
  \end{equation*}

  Then now with $(*)$ and $(**)$, we have that for all $m>n\geq K$,
  $$
    |x_{m}-x_{n}|\leq\frac{C^{n-1}}{1-C}|x_2-x_1|<\epsilon
  $$

  Thus $\{x_n\}$ is \href{a8f670d}{Cauchy}.
\end{proof}
