\subsection{Subsequences}\label{ca43129}

\Theorem{Monotone Convergence Theorem}\label{cc11aa4}

If $\{x_n\}$ is \href{feae1b2}{monotone} and bounded, then $\{x_n\}$
\href{de3e28a}{converges}. In this case,
$$
  \lim_{n\to\infty}x_n=\begin{cases}
    \sup\set{x_n}{n\in\N} & \text{if $\{x_n\}$ is increasing} \\[0.5em]
    \inf\set{x_n}{n\in\N} & \text{if $\{x_n\}$ is decreasing}
  \end{cases}
$$

\begin{proof}
  \def\limn{\displaystyle\lim_{n\to\infty}}

  \paragraph{Case 1: $\{x_n\}$ is increasing and bounded.} Let
  $S:=\set{x_n}{n\in\N}$. Then $S\neq0$ and $S$ is bounded above. Thus by the
  \href{f330cf9}{supremum property} of $\R$, $x:=\sup S$ exists. We shall prove
  that $\limn x_n=x$.

  Let $\epsilon>0$ be given. Since $x=\sup S$, $x-\epsilon$ is not an upper
  bound of $S$. So there exists $x_K\in S$ ($K\in\N$) such that
  $x_K>x-\epsilon$. Thus $0\leq x-x_K<\epsilon$.

  Since $\{x_n\}$ is increasing, $x_K\leq x_n$ for all $n\geq K$. It follows
  that for all $n\geq K$, we have
  $$
    0\leq x-x_n\leq x-x_K<\epsilon
  $$

  So $|x-x_n|<\epsilon$ for all $n\geq K$, and this says that $\limn x_n=x$.

  \paragraph{Case 2: $\{x_n\}$ is decreasing and bounded.}

  Use similar reasoning or consider the sequence $\{-x_n\}$
\end{proof}

\Corollary{Monotone Convergence Theorem*}\label{c28d9a9}
%+Limit of bounded increasing sequence is its supremum
%+Limit of bounded decreasing sequence is its infimum

\begin{enumerati}
  \def\limn{\displaystyle\lim_{n\to\infty}}
  \item If $\{x_n\}$ is \href{feae1b2}{increasing} and \href{d5ed299}{bounded
        above}, then $\{x_n\}$ \href{de3e28a}{converges} and
  $$
    \limn x_n=\sup\set{x_n}{n\in\N}
  $$
  \item If $\{x_n\}$ is decreasing and bounded below, then $\{x_n\}$ converges
        and
  $$
    \limn x_n=\inf\set{x_n}{n\in\N}
  $$
\end{enumerati}

\begin{proof}
  Refer to the proof of the \href{cc11aa4}{theorem} to which this is a corollary of.
\end{proof}

\Proposition{Basic properties of subsequences}\label{bfa595b}

\begin{enumerati}
  \item If $\{y_k\}=\{x_{n_k}\}$ is a \href{c6b3a49}{subsequence} of $\{x_n\}$,
        then $k\leq n_k$.
  \item If $\{y_k\}=\{x_{n_k}\}$ is a subsequence of $\{x_n\}$ and
        $\{z_\ell\}=\{y_{k_\ell}\}$ is a subsequence of $\{y_k\}$, then
        $\{z_\ell\}$ is a subsequence of $\{x_n\}$.
  $$
    z_\ell=x_{n_{k_\ell}}
  $$

  (A subsequence of a subsequence is still a subsequence of the original
  sequence)
\end{enumerati}

\Theorem{Subsequence convergence}\label{da6e7f5}
%+Convergence of subsequences

If $\{x_n\}$ \href{de3e28a}{converges} to $x$, then any
\href{c6b3a49}{subsequence} also converges to $x$.

Formally, if $\{x_{n_k}\}$ is a subsequence of $\{x_n\}$, then
$$
  \lim_{n\to\infty}x_n=x\implies\lim_{k\to\infty}x_{n_k}=x
$$

\begin{proof}
  Let $\epsilon>0$ be given. Since $x_n\to x$, there exists $K\in\N$ such that
  \begin{equation*}
    |x_n-x|<\epsilon,\with{\forall n\geq K}\Tag{*}
  \end{equation*}

  Since $\{x_{n_k}\}$ is a subsequence of $\{x_n\}$, we have $k\leq n_k$ for
  all $k\in\N$. Thus
  $$
    k\geq K\implies n_k\geq K\stackrel{(*)}{\implies}|x_{n_k}-x|<\epsilon
  $$
\end{proof}

\Corollary{Subsequence convergence*}\label{b3182b1}
%+Convergence of subsequences*

\begin{enumerati}
  \item If $\{x_n\}$ has a \href{c6b3a49}{subsequence} which is divergent, then
        $\{x_n\}$ diverges.
  \item If $\{x_n\}$ has two subsequences whose limits are not equal, then
        $\{x_n\}$ diverges.
\end{enumerati}

\begin{proof}
  These are contrapositive statements of \autoref{da6e7f5}.
\end{proof}

\Theorem{Monotone Subsequence Theorem}\label{dddb70e}

Every \href{b5fa0e4}{sequence} has a \href{feae1b2}{monotone}
\href{c6b3a49}{subsequence}.

\begin{proof}
  \def\xn{\{x_n\}}

  Let $\xn$ be a sequence. We call a term $x_p$ a \textbf{peak term} of $\xn$
  if
  $$
    x_p\geq x_n\quad(\forall n\geq p)
  $$

  That is, all terms after $x_p$ never go above $x_p$ again. Then there are
  only two cases:

  \paragraph{Case 1: $\xn$ has infinitely many peak terms.}

  Then the subsequence formed by all the peak terms form a decreasing
  subsequence of $\xn$.

  \paragraph{Case 2: $\xn$ has finitely many peak terms.}

  Let $x_{p_1},x_{p_2},\ldots,x_{p_j}$ be \textbf{all} the peak terms.

  Let $n_1=p_j+1$ be the first term after the last peak term.

  Since $x_{n_1}$ is not a peak term. $\implies\exists n_2>n_1$ such that
  $x_{n_1}<x_{n_2}$.

  Since $x_{n_2}$ is not a peak term, $\implies\exists n_3>n_2$ such that
  $x_{n_2}<x_{n_3}$.

  Continuing indefinitely, we can form an increasing subsequence $\{x_{n_k}\}$.
\end{proof}

\Theorem{Bolzano-Weierstrass Theorem}\label{d277ad0}

Every \href{d5ed299}{bounded sequence} has a \href{de3e28a}{convergent}
\href{c6b3a49}{subsequence}.

\begin{proof}
  \def\xn{\{x_n\}}
  \def\xnk{\{x_{n_k}\}}

  Let $\xn$ be a bounded sequence. By the \href{dddb70e}{Monotone Subsequence
  Theorem}, $\xn$ has a monotone subsequence $\xnk$.

  Since $\xn$ is bounded, so is $\xnk$.

  Since $\xnk$ is both monotone and bounded, it follows from the
  \href{cc11aa4}{Monotone Convergence Theorem} that $\xnk$ converges.
\end{proof}

\Theorem{}\label{d350704}

Let $\{x_n\}$ be a bounded sequence and let
$M:=\displaystyle\limsup_{n\to\infty}x_n$ (see: \href{f4f2af4}{$\limsup$}).

\begin{enumerati}
  \item For each $\epsilon>0$, there are at most finitely many $n$'s such that
        $x_n\geq M+\epsilon$.

  Equivalently, for each $\epsilon>0$, there exists $K=K(\epsilon)\in\N$ such
  that
  $$
    x_n<M+\epsilon,\with{\forall n\geq K}
  $$

  \item For each $\epsilon>0$, there are infinitely many $n$'s such that
        $x_n>M-\epsilon$.
\end{enumerati}

(We state \href{b4ab746}{here} that the converse of this theorem is also true.)

\begin{proof}
  \def\X{\Set{x_k}{k\geq n}}

  First, we will write proofs based on the main definition of $\limsup$.

  Define the sequence $\{y_n\}$ by $y_n:=\sup\X$.

  \proofp{(i)} Since $M=\displaystyle\lim_{n\to\infty}y_n$, given any
  $\epsilon>0$, there exists $K\in\N$ such that for all $n\geq K$, we have
  $$
    |y_n-M|<\epsilon,
  $$

  which implies
  $$
    y_n-M<\epsilon\implies y_n<M+\epsilon
  $$

  And from the definition of $y_n$, clearly $x_n\leq y_n$, since $x_n\in\X$.
  Hence we have
  $$
    x_n<M+\epsilon
  $$

  \proofp{(ii)} Since $M=\displaystyle\lim_{n\to\infty}y_n$, given any
  $\epsilon>0$, there exists $K\in\N$ such that for all $n\geq K$, we have
  $$
    |y_n-M|<\epsilon,
  $$

  which implies
  $$
    -(y_n-M)<\epsilon\implies y_n>M-\epsilon
  $$

  Hence $M-\epsilon$ is not an upper bound for $\X$, since $y_n$ is its lowest
  upper bound. Hence, for each $n\geq K$ there exists $\bar
  x\in\X\subseteq\Set{x_n}{n\in\N}$ such that $\bar x>M-\epsilon$.

  ---

  Next, we will write proofs based on the alternative definition of $\limsup$.
  We first define $S(x_n)$ to be the set of all subsequential limits of
  $\{x_n\}$.

  \proofp{(i)} Suppose (i) is false. Then there exists $\epsilon>0$ such that
  there are infinitely many $n$'s such that $x_n\geq M+\epsilon$. We now choose
  a subsequence $\{x_{n_k}\}$ from these terms. Then
  $$
    x_{n_k}\geq M+\epsilon,\with{\forall k\in\N}
  $$

  Since $\{x_{n_k}\}$ is bounded, by \href{d277ad0}{Bolzano-Weierstrass}, it
  has a convergent subsequence $x_{n_{k_\ell}}\to x$, for some $x\in\R$. Note
  that $x\geq M+\epsilon$, since \href{d88455d}{$\geq$ is preserved} upon
  taking limits.

  But this is saying that $x\in S(x_n)$, but also that $x>M=\sup S(x_n)$. This
  contradicts the property that $\sup S(x_n)$ is an upper bound of $S(x_n)$.
  This proves (i).

  \proofp{(ii)} Suppose (ii) is false. Then there exists $\epsilon>0$ such that
  there are only finitely many $n$'s such that $x_n>M-\epsilon$. Equivalently,
  this means that there exists $K\in\N$ such that $x_n\leq M-\epsilon$ for all
  $n\geq K$.

  It follows readily that all subsequential limits are $\leq M-\epsilon$. So
  $M-\epsilon$ is an upper bound for $S(x_n)$, but this contradicts the
  property that $\sup S(x_n)$ is the least upper bound for $S(x_n)$. This
  proves (ii).
\end{proof}

\Theorem{}\label{b4ab746}

The converse of \autoref{d350704} is true: if $M\in\R$ satisfies conditions (i)
and (ii), then $M=\displaystyle\limsup_{n\to\infty}x_n$ (see:
\href{f4f2af4}{$\limsup$}).

\begin{proof}
  Let $\epsilon>0$ be arbitrary. Since $M$ satisfies (i), there exists
  $K_1\in\N$ such that
  $$
    n\geq K_1\implies x_n<M+\epsilon
  $$

  Since $M$ satisfies (ii), there exists $K_2\in\N$ such that
  $$
    n\geq K_2\implies x_n>M-\epsilon
  $$

  So we can pick $K:=\max\{K_1,K_2\}$, so that we have for all $n\geq K$,
  $$
    -\epsilon<x_n-M<\epsilon\implies|x_n-M|<\epsilon
  $$

  which says that $\displaystyle\lim_{n\to\infty}x_n=M$.

  Hence every subsequence of $\{x_n\}$ goes to $M$ (\autoref{da6e7f5}).
  Therefore the set of subsequential limits of $\{x_n\}$ is a singleton
  $\{M\}$, and finally $M=\displaystyle\limsup_{n\to\infty}x_n$ by
  \href{f4f2af4}{definition}.
\end{proof}

\Theorem{}\label{ea49e1a}

Let $\{x_n\}$ be a \href{d5ed299}{bounded sequence} and let
$m:=\displaystyle\liminf_{n\to\infty}x_n$ (see: \href{f4f2af4}{$\liminf$}).

\begin{enumerati}
  \item For each $\epsilon>0$, there are at most finitely many $n$'s such that
        $x_n\leq m-\epsilon$.

  Equivalently, for each $\epsilon>0$, there exists $K=K(\epsilon)\in\N$ such
  that
  $$
    x_n>m-\epsilon,\with{\forall n\geq K}
  $$

  \item For each $\epsilon>0$, there are infinitely many $n$'s such that
        $x_n<m+\epsilon$.
\end{enumerati}

\begin{proof}
  Similar to \autoref{d350704}.
\end{proof}

\Lemma{Existence of limit on limsup and liminf}\label{ea8320c}

If the \href{b5fa0e4}{sequence} $\{x_n\}$ \href{de3e28a}{converges} to a
\href{e565120}{limit} $\bar x$, then
$$
  \limsup_{n\to\infty}x_n=\liminf_{n\to\infty}x_n=\bar x\desc{see: \href{f4f2af4}{$\liminf$/$\limsup$}}
$$

\begin{proof}
  If $x_n\to\bar x$, then by \autoref{da6e7f5}, every
  \href{c6b3a49}{subsequence} of $\{x_n\}$ also converges to $\bar x$.

  Hence the set of all subsequential limits of $\{x_n\}$ is a singleton,
  $\{\bar x\}$, and the supremum and infimum of this set is clearly $\bar x$.
  It follows from \href{f4f2af4}{definition} that
  $$
    \limsup_{n\to\infty}x_n=\liminf_{n\to\infty}x_n=\bar x
  $$
\end{proof}

\Theorem{Coincidence of limit, limsup, and liminf}\label{ccbc3b1}

Let $\{x_n\}$ be a \href{d5ed299}{bounded sequence}. Then it converges (to
$\bar x$) if and only if
$$
  \limsup_{n\to\infty}x_n=\liminf_{n\to\infty}x_n(=\bar x)\desc{see: \href{f4f2af4}{$\liminf$/$\limsup$}}
$$

In short, $\displaystyle\lim_{n\to\infty}x_n=\bar x
\iff\limsup_{n\to\infty}=\liminf_{n\to\infty}=\bar x$

\begin{proof}
  ($\implies$) This follows from \autoref{ea8320c}.

  ($\impliedby$) Let $\displaystyle\bar x=
  \limsup_{n\to\infty}x_n=\liminf_{n\to\infty}x_n$. Let $\epsilon>0$ be given.

  By \autoref{d350704} and \autoref{ea49e1a}, there exist $K_1,K_2\in\N$ such
  that
  \begin{gather*}
    n\geq K_1\implies x_n<\bar x+\epsilon\text{, and} \\
    n\geq K_2\implies x_n>\bar x-\epsilon
  \end{gather*}

  Put $K:=\max\{K_1,K_2\}$. Then for all $n\geq K$, we enjoy both results:
  $$
    \bar x-\epsilon<x_n<\bar x+\epsilon
  $$

  Which then gives $|x_n-\bar x|<\epsilon$. Hence
  $\displaystyle\lim_{n\to\infty}x_n=\bar x$.
\end{proof}

\Theorem{}\label{a19e9f7}

Let $\{x_n\}$ and $\{y_n\}$ be \href{d5ed299}{bounded sequences} such that
$x_n\leq y_n$ for every $n\in\N$. Then we have
$$
  \limsup_{n\to\infty}x_n\leq \limsup_{n\to\infty}y_n\desc{see: \href{f4f2af4}{$\liminf$/$\limsup$}}
$$

and
$$
  \liminf_{n\to\infty}x_n\leq \liminf_{n\to\infty}y_n
$$

\begin{proof}
  Let $x$ be a \href{fd942fa}{subsequential limit} of $\{x_n\}$, and let
  $\{x_{n_k}\}$ be a subsequence of $\{x_n\}$ such that $x_{n_k}\to x$. Consider
  the corresponding subsequence $\{y_{n_k}\}$ of $\{y_n\}$. Since $\{y_{n_k}\}$
  is bounded, by \href{d277ad0}{Bolzano-Weierstrass}, it has a convergent
  subsequence $\{y_{n_{k_\ell}}\}$.

  Then since $\{x_{n_{k_\ell}}\}\leq\{y_{n_{k_\ell}}\}$ for all $\ell$, we have
  \begin{align*}
    x &=\lim_{k\to\infty}x_{n_k}\desc{by definition}                                     \\
      &=\lim_{\ell\to\infty}x_{n_{k_\ell}}\desc{\href{da6e7f5}{subsequence convergence}} \\
      &\leq\lim_{\ell\to\infty}y_{n_{k_\ell}}\desc{\href{d88455d}{$\leq$ preserved}}     \\
      &\leq\limsup_{n\to\infty}y_n\desc{\href{f4f2af4}{by definition}}
  \end{align*}

  Since $x$ was arbitrarily chosen, we have just shown that
  $\displaystyle\limsup_{n\to\infty}y_n$ is an upper bound for the $S(x_n)$,
  the set of all subsequential limits of $\{x_n\}$. It follows that
  $\displaystyle\limsup_{n\to\infty}x_n\leq\limsup_{n\to\infty}y_n$.

  The proof of the remaining inequality is left as an exercise.
\end{proof}

\Result{$\limsup$ and $\liminf$ of a bounded sequence}\label{d98d9e2}

Let $\{x_n\}$ be a \href{d5ed299}{bounded sequence}.

\begin{enumerati}
  \item There exists a subsequence of $\{x_n\}$ which converges to
        $\displaystyle\limsup_{n\to\infty}x_n$.
  \item There exists a subsequence of $\{x_n\}$ which converges to
        $\displaystyle\liminf_{n\to\infty}x_n$.
\end{enumerati}

\begin{proof}
  \proofp{(i)} Let $M:=\displaystyle\limsup_{n\to\infty}x_n$. Let $\epsilon_1>0$
  be given. Then by \autoref{d350704}, there exists $K_1\in\N$ such
  that
  $$
    n\geq K_1\implies x_n<M+\epsilon_1
  $$

  But by \autoref{d350704}, there are infinitely many $n$'s such that
  $M-\epsilon_1<x_n$. Fix one such $n_1\in\N$ such that $n_1>K_1$, and now we
  have
  $$
    M-\epsilon_1<x_{n_1}<M+\epsilon_1\implies|x_{n_1}-M|<\epsilon_1
  $$

  What we've just done was to show a method to pick a term $x_{n_1}$ based on
  $\epsilon_1$. with the property $|x_{n_1}-M|<\epsilon_1$.

  Using this method, we can have a sequence $\{\epsilon_k\}$ chosen to be
  $$
    \{\epsilon_k\}:=\left\{1,\frac12,\frac14,\ldots\right\}
  $$

  which determines the sequence
  $$
    \{x_{n_k}\}:=\left\{x_{n_1},x_{n_2},\ldots\right\}
  $$

  chosen using the method above. Observe that this sequence has two properties:
  \begin{enumerati}
    \item it is a subsequence of $\{x_n\}$.
    \item it converges to $M$. By construction we have
          $M-\epsilon_k<x_{n_k}<M+\epsilon_k$, and by choice of
          $\{\epsilon_k\}$ we have
    $$
      \lim_{k\to\infty}M-\epsilon_k=\lim_{k\to\infty}M+\epsilon_k=M
    $$

    so by the \href{c3364d9}{Squeeze Theorem}, we have
    $$
      \lim_{k\to\infty}x_{n_k}=M
    $$
  \end{enumerati}

  Hence we've just found a subsequence $\{x_{n_k}\}$ of $\{x_n\}$ that
  converges to $M=\displaystyle\limsup_{n\to\infty}x_n$.
\end{proof}
