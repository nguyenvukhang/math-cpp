\chapter{\texttt{defs::real\_analysis}}\label{ded38cd}

\begin{toc}
  \citem{b6631c3} % Basics
  \citem{b81c8db} % Bounds, Supremum, Infimum
  \citem{f5b7518} % Sets
  \citem{d310a77} % Sequences and limits
  \citem{aff9ce6} % Subsequences
  \citem{dda0b11} % Cauchy sequences
  \citem{e1fcd30} % Divergence
  \citem{f9277e5} % Series
  \citem{a052053} % Closure
  \citem{db384c3} % Metric spaces
  \citem{d08d5ae} % Measure Theory
\end{toc}

\subsection{Basics}\label{b6631c3}

\Definition{Number systems}\label{d52c6b7}

\begin{enumerati}
  \item $\N:=$ set of all \textbf{natural} numbers $\{1,2,3,\ldots\}$
  \item $\Z:=$ set of all \textbf{integers} $\{\ldots,-2,-1,0,1,2,\ldots\}$
  \item $\Q:=$ set of all \textbf{rational} numbers $\Set{\dfrac pq}{p,q\in\Z,q\neq0}$
  \item $\mathbb R:=$ set of all \textbf{real} numbers
  \item $\mathbb C:=$ set of all \textbf{complex} numbers
\end{enumerati}

We have
$$
  \N\subseteq\Z\subseteq\Q\subseteq\R\subseteq\C.
$$

The set of \textbf{irrational} numbers is denoted by $\R\setminus\Q$.

\Definition{Binary operation}\label{d526017}

A \textit{binary operation} on a set $S$ is a function that maps an ordered
pair elements of $S$ to an element of $S$. For example, $+$ as a binary
operation on $\R$ is really a function that takes two values $a,b\in\R$ and
maps it to $a+b\in\R$.

Formally, any function $f$ such that $f:S\times S\to S$, we call $f$ a binary
operation on $S$.

\Definition{Binary relation}\label{a3a60e1}

A \textit{binary relation} associates elements of one set, called the domain,
with elements of another set, called the codomain.

A binary relation over sets $X$ and $Y$ is defined as a subset of
$$
  \Set{(x,y)}{x\in X,y\in Y},
$$

where we say $x\in X$ is related to $y\in Y$ if and only if the pair $(x,y)$
belong to the set which defines the binary relation. Here, $X$ is the domain
and $Y$ is the codomain.

An example of a binary relation is $<$, and we use it by writing ``$a<b$" to
denote that $a$ is related to $b$ is a particular way.

\Definition{Field}\label{aec6040}

A set $\mathcal F$ together with two \href{d526017}{binary operations}, $+$ and
$\cdot$ on $\mathcal F$ such that the \textbf{Field Axioms} are met is called a
\textit{field}.

The field axioms are, for all $a,b,c\in\mathcal F$:
\begin{itemize}
  \item [(\textbf{A1})] \textit{(Commutativity)} $a+b=b+a$
  \item [(\textbf{A2})] \textit{(Associativity)} $(a+b)+c=a+(b+c)$
  \item [(\textbf{A3})] \textit{(Existence of additive identity)} $\exists0\in\mathcal F:\ a+0=a$
  \item [(\textbf{A4})] \textit{(Existence of additive inverse)} $\forall
        x\in\mathcal F,\ \exists y\in\R:\ x+y=0$
  \item [(\textbf{M1})] \textit{(Commutativity)} $a\cdot b=b\cdot a$
  \item [(\textbf{M2})] \textit{(Associativity)} $(a\cdot b)\cdot c=a\cdot
        (b\cdot c)$
  \item [(\textbf{M3})] \textit{(Existence of multiplicative identity)} $\exists1\in\mathcal F:\ a\cdot1=a$
  \item [(\textbf{M4})] \textit{(Existence of multiplicative inverse)} $\forall
        x\in\mathcal F\sans0,\ \exists z\in\mathcal F:\ x\cdot z=1$
  \item [(\textbf{D})] \textit{(Distributivity of multiplication over addition)}
        $a\cdot(b+c)=(a\cdot b)+(a\cdot c)$
\end{itemize}

Because of the (provable) uniqueness of $y$ and $z$, we usually write $y$ as
$-x$, and $z$ as $x^{-1}$ or $1/x$.

\subsection{Bounds, Supremum, Infimum}\label{b81c8db}

\Definition{Boundedness}\label{e4698be}

A non-empty set $S\subseteq\R$ is said to be \textbf{bounded above} if there
exists some $M\in\R$ such that
$$
  x\leq M\with{\forall x\in S}
$$

Such an $M$ is called an \textbf{upper bound} of $S$.

On the other hand, $S$ is said to be \textbf{bounded below} if there exists
some $m\in\R$ such that
$$
  m\leq x\with{\forall x\in S}
$$

Such an $m$ is called a \textbf{lower bound} of $S$.

If $S$ is both bounded above and bounded below, then we simply call it
\textbf{bounded}.

Equivalently, a set $S$ is bounded if there exists $M\geq0$ such that
$$
  |x|\leq M\with{\forall x\in S}
$$

\Definition{Maximum and minimum of a subset of $\mathbb R$}\label{c3ec51c}

For a non-empty $S\subseteq\R$, one defines the maximum of $S$ to be the
(necessarily unique) number $M$ such that
\begin{enumerati}
  \item $M\in S$, and
  \item $x\leq M$ for all $x\in S$.
\end{enumerati}

Similarly, the \textbf{minimum} of $S$ is the (necessarily unique) number $m$
such that
\begin{enumerati}
  \item $m\in S$, and
  \item $m\leq x$ for all $x\in S$.
\end{enumerati}

\Definition{Supremum}\label{e6981e1}

Let $E\subseteq\R$ be non-empty. A real number $M\in\R$ is called the
\textbf{supremum} of $E$ (we write $\sup E$) if
\begin{enumerati}
  \item $M$ is an \href{e4698be}{upper bound} of $E$, and
  \item if $M'$ is an upper bound of $E$, then $M\leq M'$.
\end{enumerati}

\Definition{Infimum}\label{ff16df6}

Let $E\subseteq\R$ be non-empty. A real number $m\in\R$ is called the
\textbf{infimum} of $E$ (we write $\inf E$) if
\begin{enumerati}
  \item $m$ is a \href{e4698be}{lower bound} of $E$, and
  \item if $m'$ is a lower bound of $E$, then $m'\leq m$.
\end{enumerati}

\subsection{Sets}\label{f5b7518}

\Definition{Dense sets}\label{e929c5e}

The set $D\subseteq\R$ is said to be \textbf{dense} in $\R$ if for any
$a,b\in\R$ with $a<b$, we have $D\cap(a,b)\neq\emptyset$.

In other words, $\exists x\in D$ such that $a<x<b$.

\Definition{Intervals}\label{c65e94a}

An \textbf{interval} is a subset $I$ of $\R$ with the following (equivalent)
properties:
\begin{itemize}
  \item if $x\leq t\leq y$ and $x,y\in I$, then $t\in I$.
  \item if $x,y\in I$ and $x\leq y$, then $[x,y]\subseteq I$.
\end{itemize}

\nextsection
\subsection{Sequences and limits}\label{d310a77}

\Definition{Sequence}\label{b5fa0e4}

A \textbf{sequence} in $\R$ is a function $X:\N\to\R$.

The numbers $\set{X(n)}{n\in\N}$ are called the \textbf{terms} of the sequence.
For each $n\in\N$, $X(n)$ is called the $n$-th term of the sequence.

\paragraph{Notation.}

We usually write $x_n$ for $X(n)$ and denote the sequence $X$ by any one of
$$
  \{x_n\},\ \{x_n\}_{n=1}^\infty,\ \{x_n\}_{n\in\N},\ \{x_n\}_\N
$$

\Definition{Bounded sequence}\label{d5ed299}

The boundedness of a \href{b5fa0e4}{sequence} $\{x_n\}$ is determined by the
set
$$
  \set{x_n}{n\in\N}
$$

used in definitions stated \href{e4698be}{here}.

\Definition{Constant sequence}\label{d661313}

A constant sequence is of the form
$$
  \{c,c,c,\ldots\}
$$

for some constant $c\in\R$.

\Definition{Neighborhood}\label{ba35f12}
%+Deleted neighborhood

Let $a\in\mathcal F$ (where $\mathcal F$ is either $\R$ or $\C$) and
$\epsilon>0$. The \textbf{$\epsilon$-neighborhood of $a$} is the set
$$
  B_\epsilon(a):=\set{x\in\mathcal F}{|x-a|<\epsilon}
$$

or alternatively, $(a-\epsilon,a+\epsilon)$.

This should be thought of as the set containing elements within a certain
distance from $a$, with the distance quantified by $\epsilon$. $B$ is for ball,
since this notion of a distance bound can be generalized up to higher
dimensions.

We define the \textbf{$\epsilon$-deleted neighborhood} of $a$ as the set
$$
  B^*_\epsilon(a):=B_\epsilon(a)\sans a
$$

\Notation{Dependent variables}\label{a2d79c4}

In the context of real analysis, given some $\epsilon>0$, when we write that
there exists $K=K(\epsilon)\in\N$, we are saying that $K$ depends on
$\epsilon$.

\Definition{Limit}\label{e565120}

We say that $\bar x$ is the \textbf{limit} of the \href{b5fa0e4}{sequence}
$\{x_n\}$ if for every $\epsilon>0$, there exist
$K=\href{a2d79c4}{K(\epsilon)}\in\N$ such that
$$
  n\geq K\implies|x_n-\bar x|<\epsilon
$$

or equivalently,
$$
  \forall n\geq K,\ |x_n-\bar x|<\epsilon
$$

or,
$$
  \forall n\geq K,\ x_n\in B_\epsilon(\bar x)
$$

\Notation{Landau notation}\label{ab54b3a}

Let $\{x_n>0\},\ \{y_n>0\}$ be strictly positive sequences for large values of
$n$. Then we define $o$ and $O$ such that
$$
  \begin{array}{r c l}
    x_n=o(y_n) & \iff & \displaystyle\lim_{n\to\infty}\frac{x_n}{y_n}=0 \\[0.7em]
    x_n=O(y_n) & \iff & \exists C>0,\ \forall n\in\N:x_n\leq Cy_n
  \end{array}
$$

This is also known as ``little $o$" and ``big $O$" notation. Also,
$$
  x_n=o(y_n)\equiv x_n\ll y_n
$$

In the case where both $\{x_n\}$ and $\{y_n\}$ blow up to infinity,
$x_n=o(y_n)$ means that when $n$ is large, $x_n$ is much smaller than $y_n$,
even though $x_n$ itself is large.

\Definition{Convergence}\label{de3e28a}

\begin{enumerati}
  \item If $\bar x$ is the limit of $\{x_n\}$, then we also say that $\{x_n\}$
        \textbf{converges} to $\bar x$, and we write
  $$
    \lim_{n\to\infty}x_n=\bar x
  $$

  or `` $x_n\to\bar x$ as $n\to\infty$ " or simply `` $x_n\to\bar x$ ".
  \item We say that a sequence $\{x_n\}$ \textbf{converges} if it converges to
        a (finite) limit $\bar x\in\R$; and that it \textbf{diverges} if it
        does not converge (to a finite limit).
\end{enumerati}

\Definition{$\epsilon$–$\delta$ definition of limit}\label{d2d461a}

Let $f:A\to\R$, where $\emptyset\neq A\subseteq R$. Let $c$ be a
\href{b0219cd}{cluster point} of $A$. We say that a $L\in\R$ is the
\textbf{limit of} $f$ \textbf{at} $x=c$ and write
$$
  \lim_{x\to c}f(x)=L
$$

if for every $\epsilon>0$, there exists
$\delta=\href{a2d79c4}{\delta(\epsilon)}>0$ such that for all $x\in A$ and
$0<|x-c|<\delta$,
$$
  |f(x)-L|<\epsilon
$$

or equivalently,
\begin{itemize}
  \item $x\in A\land0<|x-c|<\delta\implies|f(x)-L|<\epsilon$
  \item $x\in A\sans c\land|x-c|<\delta\implies|f(x)-L|<\epsilon$
\end{itemize}

In this case, we say that $f$ \textbf{converges} to $L$ at $x=c$. We sometimes
also write $f(x)\to L$ as $x\to c$.

If the limit of $f$ at $x=c$ does not exist (in $\R$), we say that $f$
\textbf{diverges} at $x=c$.

\subsection{Subsequences}\label{aff9ce6}

\Definition{Monotone sequences}\label{feae1b2}
%+Decreasing sequences
%+Increasing sequences

We say that a sequence $\{x_n\}$ is
\begin{itemize}
  \item \textbf{increasing} if
        $$
          x_1\leq x_2\leq\ldots
        $$
  \item \textbf{decreasing} if
        $$
          x_1\geq x_2\geq\ldots
        $$
  \item \textbf{monotone} if it is either increasing or decreasing.
\end{itemize}

\Definition{Subsequences}\label{c6b3a49}

Let $\{x_n\}$ be a sequence and let
$$
  n_1<n_2<\ldots<n_k<\ldots
$$

be a strictly increasing sequence of natural numbers. We call the sequence
$$
  \{x_{n_k}\}=\{x_{n_1},x_{n_2},\ldots,x_{n_k},\ldots\}
$$

a \textbf{subsequence} of $\{x_n\}$.

Note that the $k$-th term of the subsequence is the $n_k$-th term of the
original sequence.

\Definition{Subsequential limit}\label{fd942fa}

Let $\{x_n\in\R\}$ \href{b5fa0e4}{sequence}. $\bar x\in\R$ is called a
\textbf{subsequential limit} of $\{x_n\}$ if $\{x_n\}$ has a subsequence
$\{x_{n_k}\}$ which \href{de3e28a}{converges} to $\bar x$.

\Definition{Limit superior/limit inferior}\label{f4f2af4}

Let $\{x_n\in\R\}$ \href{b5fa0e4}{sequence}. We define

\begin{enumerati}
  \item the \textbf{limit superior} of $\{x_n\}$ to be
  $$
    \limsup_{n\to\infty}x_n:=\lim_{n\to\infty}\sup\Set{x_k}{k\geq n}
  $$
  \item the \textbf{limit inferior} of $\{x_n\}$ to be
  $$
    \liminf_{n\to\infty}x_n:=\lim_{n\to\infty}\inf\Set{x_k}{k\geq n}
  $$
\end{enumerati}

Alternatively, we can also define them by
\begin{align*}
  \limsup_{n\to\infty}x_n &:=\sup S(x_n) \\
  \liminf_{n\to\infty}x_n &:=\inf S(x_n)
\end{align*}

where $S(x_n)$ is the set of all \href{fd942fa}{subsequential limits} of
$\{x_n\}$.

\subsection{Cauchy sequences}\label{dda0b11}

\Definition{Cauchy sequence}\label{a8f670d}

A sequence $\{x_n\}$ is called a \textbf{Cauchy sequence} if for every
$\epsilon>0$, there exists $K=K(\epsilon)\in\N$ such that
$$
  |x_n-x_m|<\epsilon,\with{\forall n,m\geq K}
$$

That is, past a certain point, the difference between any two terms is
arbitrarily small.

Note that the constraint on $m,n$ can be re-worded as $m>n\geq K$ without loss
of generality.

\Definition{Contractive sequences}\label{d5c8fb8}

A sequence $\{x_n\}$ is said to be contractive if there exists a \textbf{fixed}
$C\in\R$ with $0<C<1$ such that
$$
  |x_{n+2}-x_{n+1}|\leq C|x_{n+1}-x_n|,\with{\forall n\in\N}
$$

Note that the sequence $\{x_n\}$ given by $x_n:=\frac1n$ is an example of a
non-contractive convergence sequence, since given any $C$ such that $0<C<1$, we
have
\begin{align*}
  |x_{n+2}-x_{n+1}|                        &\leq C|x_{n+1}-x_n|                    \\
  \iff\left|\frac1{n+2}-\frac1{n+1}\right| &\leq C\left|\frac1{n+1}-\frac1n\right| \\
  \iff\frac1{(n+2)(n+1)}                   &\leq C\left(\frac1{n(n+1)}\right)      \\
  \iff\frac{n}{n+2}                        &\leq C
\end{align*}

But since the limit of the LHS as $n\to\infty$ is 1, we can always pick an $n$
sufficiently large such that it is greater than $C$.

\subsection{Divergence}\label{e1fcd30}

\Definition{Properly divergent sequences}\label{eb71424}

We say that a \href{b5fa0e4}{sequence} $\{x_n\}$ tends to $\infty$ if for every
$M>0$, there exists $K=K(M)\in\N$ such that
$$
  x_n>M,\with{\forall n\geq K}
$$

In this case, we write $\displaystyle\lim_{n\to\infty}x_n=\infty$.

On the other hand, $\{y_n\}$ tends to $-\infty$ if for every $M<0$, $\exists
K=K(M)\in\N$ such that
$$
  y_n<M,\with{\forall n\geq K}
$$

In this case, we write $\displaystyle\lim_{n\to\infty}y_n=-\infty$.

We call a sequence $\{z_n\}$ \textbf{properly divergent} if either it
$z_n\to\infty$ or $z_n\to-\infty$.

\subsection{Series}\label{f9277e5}

\Definition{Series}\label{d659804}

A series is an infinite sum, represented by an infinite expression of the form
$$
  a_1+a_2+a_3+\ldots
$$

where $\{a_n\}$ is a \href{b5fa0e4}{sequence} of terms. Using summing notation,
the above series is written as
$$
  \sum_{k=1}^\infty a_k
$$

\Definition{Partial sums}\label{a835138}

Given a \href{d659804}{series} $\displaystyle\sum_{k=1}^\infty a_k$, its $n$-th
\textbf{partial sum} is given by
$$
  s_n:=\sum_{k=1}^na_k=a_1+a_2+\ldots+a_n
$$

\Definition{Harmonic series}\label{c9bddda}

The harmonic series is the infinite series
$$
  \sum_{n=1}^\infty\frac1n=1+\frac12+\frac13+\frac14+\ldots
$$

\Definition{Convergence of a series}\label{f8901df}
%+Series convergence

If the sequence of \href{a835138}{partial sums} $\{s_n\}$ of the series
$\sum_{k=1}^\infty a_k$ converges to a number $S\in\R$, we say that the series
$\sum_{k=1}^\infty a_k$ converges (to $S$) and write
$$
  \sum_{k=1}^\infty a_k=\lim_{n\to\infty}s_n=S
$$

We also call $S$ the \textbf{sum} of $\sum_{k=1}^\infty a_k$.

\Definition{Absolute convergence}\label{f823d65}

We say that the \href{d659804}{series} $\sum_{k=1}^\infty a_k$ converges
\textbf{absolutely} if the series $\sum_{k=1}^\infty|a_k|$
\href{f8901df}{converges}.

\Definition{Conditional convergence}\label{bc12578}

We say that the \href{d659804}{series} $\sum_{k=1}^\infty a_k$ converges
\textbf{conditionally} if
\begin{itemize}
  \item The series $\sum_{k=1}^\infty a_k$ \href{f8901df}{converges}, and
  \item The series $\sum_{k=1}^\infty|a_k|$ diverges.
\end{itemize}

\Definition{Non-negative series}\label{b6cffeb}

A series $\sum_{n=1}^\infty a_n$ is called a (eventually) non-negative series
if each term $a_k\geq0$ for all $k$ sufficiently large. Here, $a_k\geq0$ for
all $k$ sufficiently large means there exists $K\in\N$ such that $a_k\geq0$ for
all $k\geq K$.

\Definition{Positive series}\label{c09906a}

A series $\sum_{n=1}^\infty a_n$ is called a (eventually) positive series if
each term $a_k>0$ for all $k$ sufficiently large.

\Definition{Geometric series}\label{ae21a85}

A geometric series is determined by its initial value $a\in\R$ and its common
ratio $r\in\R$. It is given by
$$
  \sum_{n=0}^\infty ar^n=a+ar+ar^2+\ldots
$$

\Definition{$p$-series}\label{cccc2e8}

The $p$-series is the series of the form
$$
  \sum_{n=1}^\infty\frac1{n^p}
$$

\Definition{Rearrangement of a series}\label{a58ff93}

A series $\sum_{k=1}^\infty b_k$ is called a \textbf{rearrangement} of the
series $\sum_{k=1}^\infty a_k$ if there is a \href{d205f32}{bijection}
$f:\N\to\N$ such that
$$
  b_n=a_{f(n)}\with{(n\in\N)}
$$

\subsection{Closure}\label{a052053}

\Definition{Dense}\label{e14819a}

Informally, a subset $A$ of a topological space $X$ is said to be
\textbf{dense} in $X$ if every point of $X$ either belongs to $A$ or else is
arbitrarily ``close" to a member of $A$.

A subset $A$ if a topological space $X$ is said to be a dense subset of $X$ if
any of the following equivalent conditions are satisfied:
\begin{enumerati}
  \item The smallest \href{deadb92}{closed subset} of $X$ containing $A$ is $X$
        itself.
  \item The closure of $A$ in $X$ is equal to $X$. ($\Cl_XA=X$).
  \item Every point in $X$ either belongs to $A$ or is a \href{b0219cd}{cluster
        point} of $A$.
\end{enumerati}

\Definition{Point of closure}\label{f928932}

For $S$ as a subset of a Euclidean space, $x$ is a point of closure of $S$ if
every open ball centered at $x$ contains a point of $S$ (this point can be $x$
itself).

\Definition{Closure}\label{a07ff74}

The closure of a subset $S$ of points in a topological space can be defined
using any of the following equivalent definitions:
\begin{enumerati}
  \item $\Cl S$ is the set of all \href{f928932}{points of closure} of $S$.
  \item $\Cl S$ is the set $S$ together with all of its \href{f928932}{limit
  points}.
  \item $\Cl S$ is the intersection of all closed sets containing $S$.
  \item $\Cl S$ is the smallest closed set containing $S$.
  \item $\Cl S$ is the union of $S$ and its boundary $\partial S$
\end{enumerati}

\Definition{Open sets}\label{dd04b4d}

A subset $U$ of a metric space $(M,d)$ is called open if for any point $x$ in
$U$, there exists a real number $\epsilon>0$ such that any point $y\in M$
satisfying $d(x,y)<\epsilon$ belongs to $U$.

Equivalently, $U$ is open if every point $U$ has a neighborhood contained in
$U$.

An example of a metric space is $(\R^2,\norm{\,\cdot\,})$.

\Definition{Closed sets}\label{deadb92}

A subset $A$ of a \href{de3c1b1}{topological space} $(X,\mathcal T)$ is closed
if its complement $X\setminus A$ is an \href{dd04b4d}{open} subset of
$(X,\mathcal T)$

A set $A$ is closed in $X$ if and only if it is equal to its closure $\Cl A$ in
$X$.

Yet another equivalent definition is that a set is closed if and only if it
contains all of its boundary points.

\nextsection
\subsection{Metric spaces}\label{db384c3}

\Definition{Metric}\label{d23883d}
%+Distance function

A \textbf{metric} $d$ defined on a given set $M$ is a function
$$
  d:M\times M\to\R
$$

satisfying the following axioms for all points $x,y,z\in M$:
\begin{itemize}
  \item[(\textbf{M1})] The distance from a point to itself is zero:
        $$
          d(x,x)=0
        $$
  \item[(\textbf{M2})] \textit{(Positivity)} The distance between two
        distinct points is always positive:
        $$
          x\neq y\implies d(x,y)>0
        $$
  \item[(\textbf{M3})]\textit{(Symmetry)} The distance from $x$ to $y$ is the
        same as the distance from $y$ to $x$:
        $$
          d(x,y)=d(y,x)
        $$
  \item[(\textbf{M4})] The \href{f1288ad}{triangle inequality} holds:
        $$
          d(x,z)\leq d(x,y)+d(y,z)
        $$
\end{itemize}

Note that by taking all axioms except for (\textbf{M2}), one can show that the
distance is always non-negative:
$$
  0=d(x,x)\leq d(x,y)+d(y,x)=2d(x,y)
$$

Therefore (\textbf{M2}) can be weakened to $[x\neq y\implies d(x,y)\neq0]$, and
combined with (\textbf{M1}) to make
$$
  d(x,y)=0\iff x=y
$$

\Definition{Metric space}\label{dfe585e}

A metric space is an ordered pair $(M,d)$ where $M$ is a set and $d$ is a
\href{d23883d}{metric} on $M$.

\Definition{Complete metric space}\label{beab911}
%+Cauchy space

A \href{dfe585e}{metric space} $(M,d)$ is called \textit{complete} (or a
\textit{Cauchy space}) if every \href{a8f670d}{Cauchy sequence} of points in
$M$ has a \href{e565120}{limit} that is also in $M$.

Note that in the place of $\left|\,\cdot\,\right|$ in those references, we now
use $d$.

\Definition{Banach space}\label{f894cb0}

A Banach space is an ordered pair $(X,\norm{\,\cdot\,})$ where
$(X,\norm{\,\cdot\,})$ is a \href{beab911}{complete metric space},
$\norm{\,\cdot\,}$ is a \href{e0fff96}{norm}, and $X$ is a
\href{fc83050}{vector space}.

In short, it's a complete normed vector space.

Thus, a Banach space is a vector space with a \href{dfe585e}{metric} that
allows the computation of vector length and distance between vectors and is
complete in the sense that a \href{a8f670d}{Cauchy sequence} of vectors always
converges to a well-defined \href{e565120}{limit} that is within the space.

\subsection{Measure Theory}\label{d08d5ae}

\Definition{$\sigma$-algebra}\label{da9a47a}

Let $X$ be a given set, and let $\mathcal P(X)$ be its power set. Then a subset
$\Sigma\subseteq\mathcal P(X)$ is called a $\sigma$-algebra if and only if it
satisfies the following properties:
\begin{itemize}
  \item[(\textbf{S1})] $X\in\Sigma$. In the following context, $X$ is considered
        to be the universal set.
  \item[(\textbf{S2})] $\Sigma$ is closed under complementation: for all
        $A\in\Sigma$, we have $X\setminus A\in\Sigma$.
  \item[(\textbf{S3})] $\Sigma$ is closed under countable unions: for all
        $A_1,A_2,\ldots\in\Sigma$, we have $A_1\cup A_2\cup\ldots\in\Sigma$.
\end{itemize}

From these properties, it follows that the $\sigma$-algebra is also closed
under countable intersections (by applying \href{c28492b}{De Morgan's laws}).

\Definition{Measure}\label{f7f1db2}

Let $X$ be a set, and $\Sigma$ a $\sigma$-algebra over $X$. A set function
$\mu$ from $\Sigma$ to the extended real number line is called a
\textbf{measure} if the following conditions hold:
\begin{itemize}
  \item[(\textbf{M1})] $\mu(\emptyset)=0$.
  \item[(\textbf{M2})] \textit{(Non-negativity)} For all $E\in\Sigma$,
        $\mu(E)\geq0$.
  \item[(\textbf{M3})] \textit{(Countable additivity/$\sigma$-additivity)} For
        all countable collections $\{E_k\}_{k=1}^\infty$ of pairwise disjoint sets
        in $\Sigma$,
        $$
          \mu\biggl(\bigcup_{k=1}^\infty E_k\biggr)=\sum_{k=1}^\infty\mu(E_k)
        $$
\end{itemize}

\Definition{Signed measure}\label{e8f6d6e}

A signed measure is a \href{f7f1db2}{measure} with the condition on
non-negativity dropped.

\Definition{Outer measure}\label{ae9b304}

Given a non-empty set $X$, an \textit{outer measure} on $X$ is map
$\mu:\mathcal P(X)\to[0,\infty]$ with the following properties:
\begin{itemize}
  \item[(\textbf{O1})] $\mu(\emptyset)=0$.
  \item[(\textbf{O2})] \textit{(Monotonocity)} If $A,B\in\mathcal P(X)$ are such
        that $A\subseteq B$, then $\mu(A)\leq\mu(B)$.
  \item[(\textbf{O3})] \textit{($\sigma$-subaddivitiy)} For any $A\in\mathcal
        P(X)$, and $\{A_n\}$ is a sequence in $\mathcal P(X)$ with
        $A\subseteq\bigcup_{n=1}^\infty A_n$. It follows that
        $\mu(A)\leq\sum_{n=1}^\infty\mu(A_n)$
\end{itemize}

Note that by and (\textbf{O3}) and using the sequence
$\{A_1,A_2,\ldots,A_{n-1},A_n,\emptyset,\emptyset,\ldots\}$, $\mu$ is
automatically subadditive. i.e. whenever $A,\iter{A_1}{A_n}\in\mathcal P(X)$
are such that $A\subseteq\bigcup_{i=1}^nA_i$, it follows that
$$
  \mu(A)\leq\mu(A_1)+\ldots+\mu(A_n)
$$
