\chapter{\texttt{defs::topology}}\label{f64f6ad}

\begin{toc}
  \citem{a199974} % Fundamentals
  \citem{febb684} % Order Topology
  \citem{e0a3ba0} % Product Topology
  \citem{a8be694} % Subspace topology
  \citem{b65da14} % Closed sets
\end{toc}

\subsection{Fundamentals}\label{a199974}

\Definition{Topology}\label{cc8eb8b}

A \textbf{topology} on a set $X$ is a collection $\mathcal T$ of subsets of $X$
having the following properties:
\begin{itemize}
  \item[(\textbf{T1})] $\emptyset$ and $X$ are in $\mathcal T$.
  \item[(\textbf{T2})] The union of any subcollection of $\mathcal T$ is in
        $\mathcal T$.

        Let $t_1,t_2,\ldots\in\mathcal T$ (then $t_1,t_2,\ldots\subseteq X$),
        then we have $t_1\cup t_2\cup\ldots\in\mathcal T$.
  \item[(\textbf{T3})] The intersection of any finite subcollection of $\mathcal
        T$ is in $\mathcal T$.

        Let $\iter{t_1}{t_n}\in\mathcal T$, then $t_1\cap\ldots\cap
        t_n\in\mathcal T$.
\end{itemize}

\Definition{Topological space}\label{de3c1b1}

A \textbf{topological space} is an ordered pair $(X,\mathcal T)$ consisting of
a set $X$ and a \href{cc8eb8b}{topology} $\mathcal T$.

\Definition{Open set – topology}\label{c4490f8}

If $(X,\mathcal T)$ is a \href{de3c1b1}{topological space}, we say that a
subset $U$ of $X$ is an \textbf{open set} of $X$ (or \textbf{open} in $X$) if
$U$ belongs to the collection of $\mathcal T$.

To be completely precise, we say that $U$ is an open set of $(X,\mathcal T)$,
or open in $(X,\mathcal T)$.

\Definition{Discrete topology}\label{a7c8870}

If $X$ is any set, the collection of \textit{all} subsets of $X$ is a
\href{cc8eb8b}{topology} on $X$. This collection is called the \textbf{discrete
topology}.

\Definition{Indiscrete topology}\label{fae8227}
%+Trivial topology

If $X$ is any set, the collection $\{\emptyset,X\}$ is a
\href{cc8eb8b}{topology} on $X$. This collection is called the
\textbf{indiscrete topology} (or the \textbf{trivial topology}).

\Definition{Fine/coarse topology}\label{e04d168}

Suppose that $\mathcal T_1$ and $\mathcal T_2$ are two
\href{cc8eb8b}{topologies} on a given set $X$.

\begin{itemize}
  \def\T{\mathcal T}
  \item $\T_1\subseteq\T_2$ : we say that $\T_2$ is \textbf{finer} than $\T_1$,
        and $\T_1$ is \textbf{coarser} than $\T_2$.
  \item $\T_1\subsetneq\T_2$ : we say that $\T_2$ is \textbf{strictly} finer
        than $\T_1$, and $\T_1$ is \textbf{strictly} coarser than $\T_2$.
  \item We say that $\mathcal T_1$ is comparable to $\mathcal T_2$ if either
        $\mathcal T_1\subseteq\mathcal T_2$ or $\mathcal T_2\subseteq\mathcal
        T_1$.
\end{itemize}

This terminology is suggested by thinking of topological space as being like a
truckload full of gravel–the pebbles and all the unions of collections of
pebbles being the open sets. If now we smash the pebbles into smaller ones, the
collection of open sets has been \textit{enlarged}, and the topology, like the
gravel, is said to have been made \textit{finer} by the operation.

\Definition{Basis for a topology}\label{e896402}

Let $X$ be a given set. A \textbf{basis} for a \href{cc8eb8b}{topology} on $X$
is a collection $\mathcal B$ of subsets of $X$ (called \textbf{basis elements})
such that
\begin{itemize}
  \item For each $x\in X$, there is at least one basis element $B\in\mathcal B$
        containing $x$.
  \item If $x$ belongs to the intersection of two basis elements $B_1$ and
        $B_2$, then there is a basis element $B_3$ containing $x$ such that
        $B_3\subseteq B_1\cap B_2$.
\end{itemize}

\Definition{Topology generated by a basis}\label{e6b5306}
%+Generating a topology with a basis
%+Using a basis to generate a topology
%+Basis elements are open in the topology it generates

Let $X$ be a given set and let $\mathcal B$ be a \href{e896402}{basis for a
topology} on $X$. We define the \href{cc8eb8b}{topology} $\mathcal T$ generated
by $\mathcal B$ as follows: a subset $U$ of $X$ is said to be
\href{c4490f8}{open} in $X$ (that is, to be an element of $\mathcal T$) if for
each $x\in U$, there is a basis element $B\in\mathcal B$ such that $x\in B$ and
$B\subseteq U$.

Note that each basis element is itself an element in $\mathcal T$.

\Definition{Standard topology on $\mathbb R$}\label{ad37a51}

If $\mathcal B$ is the collection of all open \href{c65e94a}{intervals} in the
real line:
$$
  \mathcal B:=\Set{(a,b)}{a,b\in\R}
$$

The \href{cc8eb8b}{topology} \href{e6b5306}{generated} by $\mathcal B$ is
called the \textbf{standard topology} on $\R$. Whenever we consider $\R$, we
shall suppose it is given this topology unless we specifically state otherwise.

\Definition{Lower limit topology on $\mathbb R$}\label{cefed4f}

If $\mathcal B$ is the collection of all intervals of the form $[a,b)$ in the
real line:
$$
  \mathcal B:=\Set{[a,b)}{a,b\in\R,\ a<b}
$$

The \href{cc8eb8b}{topology} \href{e6b5306}{generated} by $\mathcal B$ is
called the \textbf{lower limit topology} on $\R$.

When $\R$ is given the lower limit topology, we denote it by $\R_\ell$.

\Definition{$K$-topology on $\mathbb R$}\label{ee5d783}

Let $K\subset\Q$ denote the set of all numbers of the form $1/n$, for
$n\in\Z_+$, and let $\mathcal B$ be the collection of all open
\href{c65e94a}{intervals} $(a,b)$ on $\R$, along with all the sets of the form
$(a,b)\setminus K$. The \href{cc8eb8b}{topology} \href{e6b5306}{generated} by
$\mathcal B$ is called the $K$\textbf{-topology} on $\R$. When $\R$ is given
this topology, we denote it by $\R_K$.

\Definition{Subbasis for a topology}\label{aba7b48}

A \textit{subbasis} $\mathcal S$ for a \href{cc8eb8b}{topology} on $X$ is a
collection of subsets of $X$ whose union equals $X$.

\Definition{Topology generated by a subbasis}\label{d1d3329}
%+Generating a topology with a subbasis
%+Using a subbasis to generate a topology

The \href{cc8eb8b}{topology} \textbf{generated} by the \href{aba7b48}{subbasis}
$\mathcal S$ is defined to be the collection $\mathcal T$ of all unions of
finite intersections of elements of $\mathcal S$.

\subsection{Order Topology}\label{febb684}

\Definition{Order Topology}\label{aaff6da}

Let $X$ be a set with a simple order relation $<$, and assume that $X$ has more
than one element. Let $\mathcal B$ be the collection of all sets of the
following types:
\begin{enumerati}
  \def\smol{[a_0,b)}\def\big{(a,b_0]}
  \item All open \href{c65e94a}{intervals} $(a,b)$ in $X$.
  \item All intervals of the form $\smol$, where $a_0$ is the smallest element
        (if any) of $X$.
  \item All intervals of the form $\big$, where $b_0$ is the largest element
        (if any) of $X$.
\end{enumerati}

The collection $\mathcal B$ is a \href{e896402}{basis} for a
\href{cc8eb8b}{topology} on $X$, which is called the \textbf{order topology}.

\Definition{Ray}\label{e91efe4}

If $X$ is an ordered set, and $a\in X$, there are four subsets of $X$ that are
called the \textbf{rays} determined by $a$:
\begin{align*}
  (a,+\infty) &=\Set{x}{x>a},     \\
  (-\infty,a) &=\Set{x}{x<a},     \\
  [a,+\infty) &=\Set{x}{x\geq a}, \\
  (-\infty,a] &=\Set{x}{x\leq a}.
\end{align*}

Sets of the first two types are called \textbf{open rays}, and sets of the last
two types are called \textbf{closed rays}. We formalize these definitions
\href{b1745d9}{here} and \href{a52dbf3}{here}.

\Definition{Open ray}\label{b1745d9}

If $X$ is an ordered set, and $a\in X$, then elements of the following set are
defined as \textbf{open rays} on $X$:
$$
  \Set{(a,+\infty)}{a\in X}\cup
  \Set{(-\infty,a)}{a\in X}
$$

\Definition{Closed ray}\label{a52dbf3}

If $X$ is an ordered set, and $a\in X$, then elements of the following set are
defined as \textbf{closed rays} on $X$:
$$
  \Set{[a,+\infty)}{a\in X}\cup
  \Set{(-\infty,a]}{a\in X}
$$

\subsection{Product Topology}\label{e0a3ba0}

\Definition{Product topology}\label{be6372e}

Let $X,Y$ be \href{de3c1b1}{topological spaces}. The \textbf{product topology}
on $X\times Y$ is the \href{e6b5306}{topology having as basis} the collection
$\mathcal B$ of all sets of the form $U\times V$, where $U$ is an open subset
of $X$ and $V$ is an open subset of $Y$.

\Definition{Projection onto $n$-th factor – topology}\label{cad7d68}

Let $\pi_1:X\times Y\to X$ be defined by the equation
$$
  \pi_1(x,y)=x,
$$

and $\pi_2:X\times Y\to Y$ be defined by the equation
$$
  \pi_2(x,y)=y.
$$

The maps $\pi_1$ and $\pi_2$ are called the \textbf{projections} of $X\times Y$
onto its first and second factors, respectively.

We use the word ``onto" intentionally, to mean that $\pi_1$ and $\pi_2$ are
\href{bd75843}{surjective} (unless one of the spaces $X$ or $Y$ happens to be
empty, in which case $X\times Y$ is empty and our whole discussion is empty as
well).

Observe that $\pi_1^{-1}(U)=U\times Y$, and that $\pi_2^{-1}(V)=X\times V$.

\subsection{Subspace topology}\label{a8be694}

\Definition{Subspace topology}\label{cddfbd8}

Let $(X,\mathcal T)$ be a \href{de3c1b1}{topological space}. If $Y$ is a subset
of $X$, the collection
$$
  \mathcal T_Y:=\Set{Y\cap U}{U\in\mathcal T}
$$

is a \href{cc8eb8b}{topology} on $Y$, called the \textbf{subspace topology} (or
the topology that $Y$ inherits as a subspace of $X$). With this topology, $Y$
is called a \textbf{subspace} of $X$; its open sets consist of all
intersections of open sets of $X$ with $Y$.

To be completely precise, we have to say that $(Y,\mathcal T_Y)$ is a subspace
of $(X,\mathcal T)$.

\Definition{Convex sets (1-D, ordered)}\label{c19c232}

Given an ordered set $X$, let us say that a subset $Y$ of $X$ is
\textbf{convex} if for each pair of points $a<b$ of $Y$, the entire
\href{c65e94a}{interval} $(a,b)$ of points of $X$ lies in $Y$.

\subsection{Closed sets}\label{b65da14}

\Definition{Closed set – topology}\label{dd23fec}

A subset $A$ of a \href{de3c1b1}{topological space} $X$ is said to be closed if
the set $X\setminus A$ is \href{c4490f8}{open}.

Note that a set can be both closed and open at the same time.

\Definition{Interior of a set}\label{c15654b}

Given a subset $A$ of a \href{de3c1b1}{topological space} $X$, the
\textbf{interior} of $A$ is defined as the union of all \href{c4490f8}{open}
sets contained in $A$.

The interior of $A$ is denoted by $\Int A$. \href{cc8eb8b}{Clearly}, $\Int A$
is an open set. Furthermore,
$$
  \Int A\subseteq A
$$

If $A$ is open, then $\Int A=A$.

\Definition{Closure of a set – topology}\label{abdd5f2}

Given a subset $A$ of a \href{de3c1b1}{topological space} $X$, the
\textbf{closure} of $A$ is defined as the intersection of all
\href{dd23fec}{closed} sets containing $A$.

The closure of $A$ is denoted by $\Cl A$ or by $\bar A$.
\href{e48d738}{Clearly}, $\bar A$ is a closed set. Furthermore,
$$
  A\subseteq\bar A
$$

If $A$ is closed, then $A=\bar A$.

\Definition{Neighborhood – topology}\label{de512d5}

Let $X$ be a \href{de3c1b1}{topological space}. We say that $U$ is a
\textbf{neighborhood} of $x\in X$ if $U$ is \href{c4490f8}{open} in $X$ that
contains $x$.

\Definition{Cluster point}\label{b0219cd}
%+Limit point
%+Point of accumulation
%+Accumulation point

Let $A$ be a subset of a \href{de3c1b1}{topological space} $X$. A point $x$ in
$X$ is a \textbf{cluster point} of the set $A$ if every
\href{de512d5}{neighborhood} of $x$ contains at least one point of $A$ other
than $x$ itself.

A cluster point is also called a \textbf{limit point} or \textbf{accumulation
point}.

In real analysis, $c\in\R$ is a cluster point of a non-empty set $A\subseteq\R$
if for every $\epsilon>0$ there exists a point $x\in A\sans{c}$ such that
$x\in(c-\epsilon,c+\epsilon)$.

In complex analysis, $c\in\C$ is a cluster point of a non-empty set
$A\subseteq\C$ if for every $\epsilon>0$ there exists a point $z\in A\sans{c}$
such that $z\in B_\epsilon(c)$.

In real/complex analysis, we use \href{ba35f12}{this specialization} of the
definition of a neighborhood.

\Definition{Convergence – topology}\label{c8f4bbb}

Let $X$ be a \href{de3c1b1}{topological space} and let $\{x_n\}_{n\in\N}$ be a
sequence of points in $X$. Then we say that the sequence $\{x_n\}$
\textbf{converges} to the point $x$ of $X$ if for each
\href{de512d5}{neighborhood} $U$ of $x$ there is an $K=K(U)\in\N$ (we write
$K(U)$ to mean that $K$ depends on the choice of $U$) such that $x_n\in U$ for
all $n\geq K$.

\Definition{Hausdorff Space}\label{e8a8d91}

A \href{de3c1b1}{topological space} $X$ is called a \textbf{Hausdorff space} if
for each pair $x_1,x_2$ of distinct points of $X$, there exist
\href{de512d5}{neighborhoods} $U_1$ of $x_1$ and $U_2$ of $x_2$ that are
disjoint.

\Axiom{$T_1$ axiom}\label{f294751}

Let $X$ be a set. If $X$ satisfies the $T_1$ axiom, then all finite point sets
in $X$ are closed.
